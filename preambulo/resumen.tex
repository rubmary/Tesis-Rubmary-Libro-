\begin{resumen}

La teoría de juegos se encarga de estudiar la toma de decisiones estratégicas de individuos racionales en estructuras denominadas juegos. Los juegos no deterministas con información incompleta son aquellos que tienen azar y hay información oculta para los jugadores, como por ejemplo el juego de póker. En este trabajo de grado se estudian dos modelos diferentes para este tipo de juegos: forma normal y forma extensiva. La forma normal se utiliza para representar juegos en los cuales cada jugador elige una única acción en forma simultánea y la forma extensiva se utiliza para representar juegos secuenciales donde los jugadores toman decisiones por turnos. Además, se limita la investigación a juegos de dos jugadores con suma cero y se utiliza el equilibrio de Nash como principal concepto de solución. Los juegos en forma normal presentados fueron: piedra, papel o tijera, \textit{matching pennies}, ficha vs. dominó y coronel Blotto, para los cuales se calculó una aproximación a un equilibrio de Nash mediante el algoritmo de \textit{Regret Matching}. Los juegos en forma extensiva estudiados fueron: \textit{One Card Póker} (OCP), dudo (un juego de dados), y dominó para dos personas. Estos juegos fueron parametrizados según el número de cartas, dados o piezas, entre otros elementos; obteniendo múltiples instancias para cada uno de ellos. Se utilizó el algoritmo de \textit{Conterfactual Regret Minimization} para aproximar un equilibrio de Nash y se midió el error mediante la explotabilidad. En el juego OCP se resolvieron instancias de hasta $5.000$ cartas. En el juego dudo se resolvieron instancias con dados de hasta $6$ caras y $2$ dados por jugador. En el juego de dominó se resolvieron instancias con fichas que tienen hasta $3$ puntos por lado y una distribución inicial de $4$ fichas por jugador.

\vfill
\textbf{Palabras claves:} juegos, forma normal, forma extensiva, no determinismo, información incompleta, estrategias.
\end{resumen}