\chapter*{Conclusiones y Recomendaciones}

En esta investigación se estudiaron los modelos para representar juegos no deterministas con información incompleta, siendo la forma normal el modelo utilizado para representar juegos de una única acción y la forma extensiva el modelo utilizado para representar juegos secuenciales. Además, se estudiaron diferentes conceptos de solución y se utilizó el equilibrio de Nash, como el concepto de solución principal en el caso de juegos de dos jugadores con suma cero.

Para encontrar equilibrios de Nash en juegos en forma normal se utilizaron procedimientos de \textit{regret matching} y se evaluaron las estrategias mediante la explotabilidad. Los procedimientos fueron probados en $4$ juegos diferentes: piedra, papel o tijera, \textit{matching pennies}, ficha vs. dominó y una instancia del juego coronel Blotto. En todos los juegos se encontraron aproximaciones al equilibrio de Nash con una explotabilidad menor que $0.012$.

Sin embargo, los juegos de mayor interés son aquellos en forma extensiva, pues la cantidad de estados suele ser mucho mayor y son más complejos de estudiar. Para este tipo de juegos se utilizó el algoritmo \textit{Counterfactual Regret Minimization} (CFR), que permite calcular un equilibrio de Nash para juegos de dos jugadores con suma cero y \textit{perfect recall} (recuerdo perfecto). En particular, se utilizó una variación de la familia \textit{Monte Carlo Counterfactual Regret Minimization} (MCCFR), en la que se muestrean los nodos de azar. Por otra parte, se implementó una clase abstracta que proporciona una interfaz para recorrer el árbol de forma ímplicita y obtener los conjuntos de información, así como las instancias concretas para cada uno de los juegos presentados.

El algoritmo de CFR fue probado en $3$ clases de juegos: \textit{One Card Poker} (OCP), una versión simplificada del juego de poker, dudo, un juego de dados y una versión del juego de dominó para $2$ personas. El juego OCP fue parametrizado según el número de cartas inicialmente, de esta forma OCP$(N)$ significa el juego de OCP con $N$ cartas al inicio del juego. El juego dudo fue parametrizado por el número de dados de cada jugador y el número de caras del dado, siendo Dudo$(K, D_1, D_2)$ el juego en el cual el primer jugador tiene $D_1$ dados, el segundo jugador tiene $D_2$ dados y cada dado tiene $K$ caras (con los números del $1$ al $K$). Finalmente el juego de dominó fue parametrizado por el máximo número de puntos que puede tener un lado de una ficha de dominó ($M$) y por la cantidad de fichas iniciales repartidas a cada jugador $N$, haciendo referencia a esta instancia del juego como Domino$(M, N)$.

Se probaron varias instancias por cada una de las clases propuestas y se utilizó CFR para encontrar los equilibrios de Nash, con un total de aproximadamente $10$ horas de entrenamiento por cada instancia. Una instancia fue considerada resuelta si la explotabilidad de la estrategia obtenida fue menor que el $1\%$ de la mínima ganancia positiva posible, que en todos los juegos presentados, esto es igual a $1$.

El juego OCP fue probado con instancias con $N \in \{ 3, 12, 50, 200, 1000, 5000\}$. En todos los casos fue posible obtener una aproximación a un equilibrio de Nash con una explotabilidad menor que el $0.025\%$ de la mínima ganancia positiva posible. Note que este juego es sencillo comparado a los otros $2$ y la instancia con $5000$ cartas tiene únicamente $20.000$ conjuntos de información.

Con respecto al juego Dudo, se probaron instancias con Dados entre $4$ y $6$ caras, entregando $1$ o $2$ dados a cada jugador. En este juego fue posible resolver todas las instancias con $K = 3$ y $D_i \leq 2$. Cuando $K \in \{4, 5\}$ no fue posible resolver las instancias Domino$(K, 2, 2)$ y con $K = 6$ no fue posible resolver ninguna de las instancias (con el tiempo de entrenamiento propuesto). La instancia con la mayor explotabilidad fue Dudo$(5, 2, 2)$ con una explotabilidad mayor que el $15\%$, aunque no fue intentada la instancia Dudo$(6, 2, 2)$ debido a que no se obtuvieron buenos resultados para las instancias Dudo$(6, 1, 2)$ y Dudo$(6, 2, 1)$.

Para la versión planteada del juego de Dominó, se probaron instancias con $M = 2$ y $M = 3$. Fue posible resolver las instancias Domino$(2, 2)$, Domino$(3, 2)$, Domino$(3, 3)$, obteniendo una explotabilidad menor que el $0.05\%$ en cada una de ellas. Para la instancia Domino$(3, 4)$ se obtuvo una explotabilidad mayor que el $1.3\%$ y por lo tanto no se consideró resuelta.

Para todas las instancias se creó una gráfica del \textit{regret} promedio con respecto al número de iteraciones. Se puede observar que, debido a que el \textit{regret} es inicializado con $0$, a medida que se descubren los conjuntos de información esta métrica empieza a aumentar, para posteriormente converger a cero. Se observa que los juegos que no fueron resueltos presentan una gráfica con una convergencia inconclusa.

Futuras investigaciones pueden enfocarse en el juego de dominó en particular intentando resolver instancias más grandes, siendo ésta la primera vez que se estudia este juego. Para esto, se recomienda utilizar abstracciones (juegos más pequeños), que pueden ser obtenidos mediante la unión de varios conjuntos de información en uno solo. La idea es unir conjuntos de información similares o idénticos, tales que la información que se pierda no sea importante en cuanto a las estrategias \cite[pp.~71-72]{bib:thesis-marc-lanctot}. Cabe destacar, que estas abstracciones pueden o no tener \textit{perfect recall}, y CFR no garantiza calcular una aproximación a un equilibrio de Nash en un juego con \textit{imperfect recall}. Sin embargo, aun sin las garantías teóricas, esta técnica ha sido utilizada para calcular fuertes estrategias en versiones del juegos de póker, que pueden incluso superar sus contrapartes con \textit{perfect recall} \cite{bib:imperfect-recall}.

Posibles abstracciones que pueden ser utilizadas en el juego de dominó consisten en considerar únicamente la secuencia final después de cada jugada y no el orden específico en que las fichas fueron colocadas; o simplemente considerar dicha secuencia como un conjunto, sabiendo cuales son los números en los extremos y la información sobre las fichas jugadas por cada jugador, incluso se puede considerar una abstracción donde no se distingan qué fichas fueron jugadas por cada jugador. La motivación para proponer estas abstracciones consiste en que las jugadas posibles dependen únicamente de las manos actuales, las fichas restantes y las fichas en los extremos. Sin embargo, se pierde información sobre las secuencia de las jugadas del oponente que puede ser relevante en ciertas circunstancias.

Finalmente, otro posible enfoque para este juego consiste en intentar resolver el juego de dominó para $4$ jugadores con equipos de $2$ personas, según las reglas cĺásicas venezolanas, considerando cada equipo como un sólo jugador con \textit{imperfect recall} y observar si es posible calcular una aproximación a un equilibrio de Nash con el algortimo CFR (o alguna variación del mismo).