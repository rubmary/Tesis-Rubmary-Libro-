\section{Evaluación de Estrategias y Explotabilidad}
\label{section:explotabilidad}

En un juego de suma cero el \textbf{valor del juego} es igual a la ganancia esperada del primer jugador cuando los jugadores utilizan un Equilibrio de Nash $\sigma^* = (\sigma^*_1, \sigma^*_2)$. Es decir, el valor del juego es igual a $u = u_1(\sigma^*)$. Supongamos ahora que el jugador $1$  usa una estrategia $\sigma_1$, que es una ligera modificación de $\sigma^*$, entonces el jugador $2$ puede usar una estrategia que sea mejor respuesta a $\sigma_1$, digamos $\sigma^{\prime}_2$. Luego, se tiene que:

\begin{alignat}{1}
u_2(\sigma_1, \sigma^{\prime}_2) \geq u_2(\sigma_1, \sigma^*_2) \geq u_2(\sigma^*_1, \sigma^*_2)
\end{alignat}

La primera desigualdad se obtiene porque $\sigma^{\prime}_2)$ es mejor respuesta a $\sigma_1$ y la segunda desigualdad ocurre porque $\sigma^*_1$ es mejor respuesta a $\sigma^*_2$. Luego $u_2(\sigma_1, \sigma^{\prime}_2) = u_2(\sigma^*_1, \sigma^*_2) + \varepsilon_1$ para algún $\varepsilon_1 \geq 0$. Por lo tanto, la estrategia del jugador $1$ se volvió \textit{explotable} por una cantidad $\varepsilon_1$. De forma análoga se puede obtener que, si el jugador $2$ utiliza una estrategia $\sigma_2$ ligeramente alejada del equilibrio de Nash, esta estrategia será explotable por una cantidad no negativa $\varepsilon_2$.

Dada la estrategia $\sigma = (\sigma_1, \sigma_2)$, la \textbf{explotabilidad} de $\sigma$, $\varepsilon_{\sigma} = \varepsilon_1 + \varepsilon_2$, es usada frecuentemente para medir la distancia de una estretegia al equilibrio de Nash \cite[p. 7]{bib:thesis-marc-lanctot}. Por otra parte, sea $v_i = u_i(\sigma_i, \sigma^{\prime}_{-i})$, Por lo dicho anteriormente $v_i = u_i(\sigma^*) + \varepsilon_i$. Note que $v_1 = u_1(\sigma^*) + \varepsilon_1 = u + \varepsilon_1$ y $v_2 = u_2(\sigma^*) + \varepsilon_2 = -u + \varepsilon_2$, obteniendo que $v_1 + v_2 = u + \varepsilon_1 - u + \varepsilon_2 = \varepsilon_{\sigma}$.

Por último, se quiere encontrar una forma sencilla de calcular $v_i$. Para lograr esto se utiliza el hecho que, para cualquier estrategia de cualquier jugador, siempre existe una mejor respuesta cuyo soporte tiene un único elemento (Corolario del Teorema \ref{theo:mejor-respuesta}), obteniendo la siguiente ecuación:
\begin{alignat}{1}
v_i = \max_{s_{-i} \in S_{-i}} u_i(\sigma_i, s_{-i})
\end{alignat}

Usando las fórmulas anteriores se calculó la explotabilidad para cada estrategia obtenida al ejecutar cada procedimiento en cada uno de los juegos en forma normal que se utilizaron para los experimentos.
