\chapter*{Conclusiones y Recomendaciones}

En esta investigación se estudiaron los modelos para representar juegos no deterministas con información incompleta, siendo la forma normal el modelo utilizado para representar juegos de una única acción y la forma extensiva el modelo utilizado para representar juegos secuenciales. Además, se estudiaron diferentes conceptos de solución y se utilizó el equilibrio de Nash, como el concepto de solución principal en el caso de juegos de dos jugadores con suma cero, que fueron los considerados en esta investigación.

Para encontrar equilibrios de Nash en juegos en forma normal se utilizaron procedimientos de \textit{Regret Matching} y se evaluaron las estrategias mediante la explotabilidad. Los procedimientos fueron probados en 4 juegos diferentes: piedra, papel o tijera, \textit{matching pennies}, ficha vs.\ dominó y una instancia del juego coronel Blotto. En todos los juegos se encontraron aproximaciones al equilibrio de Nash con una explotabilidad no mayor que 0,010 (lo que representa el  1\% de la mínima ganancia positiva en todos los casos) por lo que se consideran resueltos.

Para encontrar equilibrios de Nash en juegos en forma extensiva se utilizó el algoritmo \textit{chance-sampled} CFR. Los juegos estudiados presentan \textit{perfect recall}, condición que garantiza la convergencia en el algoritmo en juegos de dos jugadores de suma cero. Este algoritmo fue probado en tres clases de juegos: \textit{One Card Poker} (OCP), dudo (un juego de dados) y una versión del juego de dominó para dos personas.

El juego OCP fue parametrizado por el número de cartas iniciales, representándose con OCP($N$). En este juego todas las instancias probadas (usando entre 3 y 5.000 cartas) se consideraron resueltas.

El juego dudo, representado con Dudo$(K, D_1, D_2)$ fue parametrizado por la cantidad de caras de los dados ($K$) y el número de dados de cada jugador ($D_1, D_2$). Fue posible resolver todas las instancias con dados de hasta 5 caras y 2 dados por jugador, i.e., $K \leq 4$ y $D_1, D_2 \leq 2$, con excepción de las instancias  Dudo$(4, 2, 2)$ y Dudo$(5, 2, 2)$, en esta última instancia la explotabilidad fue mayor que 15\%. Por otra parte, para $K = 6$ sólo fue posible resolver la instancia Dudo$(6, 1, 1)$.

El juego de dominó fue parametrizado por el mayor doble presente en el mazo $M$ y la cantidad de fichas repartidas a cada jugador al inicio del juego $N$, representándose con Domino$(M, N)$. Se consideraron la instancia Domino$(2, 2)$ y las instancias Domino$(3, N)$ con $N \leq 4$. Se resolvieron todas las instancias propuestas con la excepción de la instancia Domino$(3, 4)$ donde se obtuvo una explotabilidad de 1,4871\%.

Futuras investigaciones pueden estar enfocadas en el juego de dominó en particular intentando resolver instancias más grandes, siendo la primera vez que se estudia este juego en particular. Para esto, se recomienda utilizar abstracciones (juegos más pequeños), que pueden ser obtenidos mediante la unión de varios conjuntos de información en uno solo. La idea es unir conjuntos de información similares o idénticos, tales que la información que se pierda no sea importante en cuanto a las estrategias \cite[pp.~71-72]{bib:thesis-marc-lanctot}. Cabe destacar que estas abstracciones pueden o no tener \textit{perfect recall}, y CFR no garantiza calcular una aproximación a un equilibrio de Nash en un juego con \textit{imperfect recall}. Sin embargo, aún sin las garantías teóricas, esta técnica ha sido utilizada para calcular estrategias fuertes en versiones del juegos de póker que pueden incluso superar sus contrapartes con \textit{perfect recall} \cite{bib:imperfect-recall}.

Posibles abstracciones que pueden ser utilizadas en el juego de dominó consisten en considerar únicamente la secuencia final después de cada jugada y no el orden específico en que las fichas fueron colocadas; o simplemente considerar dicha secuencia como un conjunto, sabiendo cuales son los números en los extremos y la información sobre las fichas jugadas por cada jugador. Incluso se puede considerar una abstracción donde no se distingan qué fichas fueron jugadas por cada jugador. La motivación para proponer estas abstracciones consiste en que las jugadas posibles dependen únicamente de las manos actuales, las fichas restantes y las fichas en los extremos. Sin embargo, se pierde información sobre las secuencia de las jugadas del oponente que puede ser relevante en ciertas circunstancias.

Finalmente, otro posible enfoque para este juego consiste en intentar resolver el juego de dominó para 4 jugadores con equipos de 2 personas, según las reglas cĺásicas venezolanas, considerando cada equipo como un sólo jugador con \textit{imperfect recall} y observar si es posible calcular una aproximación a un equilibrio de Nash con el algoritmo CFR (o alguna variación del mismo).
