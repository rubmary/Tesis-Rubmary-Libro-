\documentclass[pregrado]{tesis-usb}

% paquetes
\usepackage[utf8]{inputenc}
\usepackage[spanish]{babel}
\usepackage{verbatim}
\usepackage{acronym}
\usepackage{amsmath}
\usepackage{amsthm}
\usepackage{amsfonts}
\usepackage{amssymb}
\usepackage{enumitem}
\usepackage{chngcntr}
\usepackage{babelbib}
\usepackage{multirow}
\usepackage{mathtools}
\usepackage{graphicx}
\usepackage{dirtytalk}
\usepackage{pifont}
\allowdisplaybreaks

\newtheorem{definition}{Definición}
\newtheorem{theorem}[definition]{Teorema}
\newtheorem{lemma}[definition]{Lema}
\newtheorem{proposition}[definition]{Proposición}
\newtheorem{example}[definition]{Ejemplo}
\renewcommand*{\proofname}{Prueba}
\counterwithin{definition}{chapter}

% estilo de las referencias
\bibliographystyle{babunsrt}

\autor{Rubmary Rojas Linárez}
\autori{R. Rojas Linárez}
\usbid{13-11263}
\titulo{Algoritmos para Juegos con Información Incompleta y No Determinismo}
\fecha{Septiembre~de~2019}
\agno{2019}
%\fechadefensa{31~de~abril~de~2015}
\tutor{Blai Bonet}
%\usarcotutor
%\cotutor{Carolina Chang} 
\trabajo{Trabajo de Grado}
\coord{Ingeniería de Computación} %Coloca la coordinación
\grado{Ingeniero de Computación}
\carrera{Ingeniería de Computación}
\programa{Ingeniería de Computación}
%\juradouno{Ren\'e Escalante}
%\juradodos{Johana Figueroa \mbox{(UC)}}
%\juradotres{Irene Garc\'ia}
%\juradocuatro{José Luis Palacios}

% Cambia comillas simple por comilla cerrada en ambiente verbatim 
\makeatletter
\let \@sverbatim \@verbatim
\def \@verbatim {\@sverbatim \verbatimplus}
{\catcode`'=13 \gdef \verbatimplus{\catcode`'=13 \chardef '=13 }} 
\makeatother

\newcommand{\Blai}[1]{\textcolor{red}{#1}}
\newcommand{\cmark}{\ding{51}}
\newcommand{\xmark}{\ding{55}}

\begin{document}
\frontmatter
\maketitle
\chapter*{Dedicatoria}

Dedicado a mis tutores, jurados y todos los que quieran leer (aunque sea parcialmente) este libro. Espero que lo disfruten tanto como yo lo hice escribiéndolo.

\chapter*{Agradecimientos}

Agradezco a mis padres Mirna Linárez y Rubén Rojas, quienes me han brindado su apoyo incondicional y me han enseñado los valores que rigen mi vida. También agradezco a mis hermanas Rubmir Rojas y Rubdary Rojas, quienes siempre me motivan a seguir adelante y a cumplir mis sueños. Agradezco a mi hermano Rubén quien fue un gran soporte durante mi carrera universitaria. Gracias a ellos he podido seguir adelante, superando las dificutades y logrando mis metas.

Gracias a todos los amigos y compañeros que he conocido y me han apoyado durante mi carrera universitaria. Agradezco especialmente a Elizabeth Acosta y Verónica Viera, a quienes conocí al inicio de la carrera y se convirtieron en mis mejores amigas. Gracias a Samuel Nacache, quien me ha acompañado durante los últimos años y me ha apoyado cuando lo necesito.

Gracias a mis tutores, los profesores Blai Bonet y Carolina Chang, por las revisiones, consejos y correcciones durante el desarrollo de este trabajo, y por su apoyo constante a lo largo de mis estudios. Gracias especiales al profesor Alfredo Ríos, que despertó mi amor por la universidad, y al profesor Ricardo Monascal que me llevó al mundo oscuro pero interesante de la programación competitiva. Gracias a los profesores que me han dado clases, particularmente a (además de los que mencioné previamente): Ivette Carolina Martínez, Marlene Goncalves, Yudith Cardinale, Aurora Olivieri, Vicente Yriarte, Minaya Villasana, Ángela Di Serio, Ildemaro García y Ernesto Hernández-Novich; cuyas clases fueron sumamente importantes en mi desarrollo académico.

Gracias a todos los profesores que, aun con todas las dificultades actuales, continúan dando clases en la universidad. Ustedes son verdaderos héroes.

Gracias a todos los que me han apoyado y me han ayudado a crecer como persona.
\begin{resumen}
    La teoría de juegos se encarga de estudiar la toma de decisiones estratégicas de individuos racionales en estructuras denominadas juegos. Los juegos no deterministas con información incompleta son aquellos que tienen azar y hay información oculta para los jugadores, como por ejemplo el juego de póker. En este trabajo de grado se estudian los diferentes modelos para este tipo de juegos, los conceptos de solución y los algoritmos para resolverlos. Se limita la investigación a juegos de dos jugadores con suma cero, para los cuales algoritmos como \textit{Regret Matching} y \textit{Conterfactual Regret Minimization} (CFR) permiten calcular un equilibrio de Nash, el principal concepto de solución utilizado. Estos algoritmos fueron implementados para diferentes juegos captados por el modelo. Los juegos en forma normal presentados fueron: piedra, papel o tijera, \textit{matching pennies}, ficha vs. dominó y coronel Blotto, para los cuales se calculó una aproximación a un equilibrio de Nash mediante el algoritmo de \textit{Regret Matching}. Los juegos en forma extensiva estudiados fueron: \textit{One Card Póker} (OCP), dudo (un juego de dados), y dominó para dos personas. Estos juegos fueron parametrizados según el número de cartas, dados o piezas, entre otros elementos; obteniendo múltiples instancias para cada uno de ellos. Para cada instancia se calculó una aproximación a un equilibrio de Nash y se midió el error mediante la explotabilidad. Un juego fue considerado resuelto si la explotabilidad de la estrategia obtenida fue menor que un límite establecido. En el juego OCP fue posible resolver todas las instancias planteadas (utilizando hasta 5000 cartas). En el juego dudo se resolvieron instancias con dados de hasta 4 caras y 2 dados por jugador. Por último, la instancia más grande resuelta en el juego de dominó incluye fichas que tienen hasta 3 puntos por lado y una distribución inicial de 3 fichas por jugador.
    
    \vfill
    \textbf{Palabras claves:} juegos, forma normal, forma extensiva, no determinismo, información incompleta, estrategias.
\end{resumen}
\tableofcontents
\listoffigures
\listoftables
\useacronyms
%\chapter*{Notación matemática}
\begingroup
\renewcommand{\arraystretch}{1.5}
\begin{tabular}{l p{12cm}}
$S_i$ & Conjunto de estrategias puras del jugador $i$ \\
$s$ & Perfil estratégico puro \\
$s_i$ & Estrategia pura del jugador $i$ \\
$s_{-i}$ & Perfil estratégico $s$ excluyendo la estrategia del jugador $i$ \\
$\Delta(A)$ & Conjunto de distribuciones de probabilidad del conjunto $A$ \\
$|A|$ & Cardinalidad (número de elementos) del conjunto A \\
$\Delta^n$ & Simplex $n$-dimensional \\
$\sigma_i$ & Estrategia mixta del jugador $i$ \\
$\sigma_i(s_i)$ & Probabilidad de elegir $s_i$ dada la estrategia mixta $\sigma_i$ \\
$\sigma$ & Perfil estratégico mixto \\
$\sigma(s)$ & Probabilidad de que se elija el perfil estratégico $s$ bajo $\sigma$ \\
$u_i$ & Función de pago (utilidad) del jugador $i$ \\
$u_i(\sigma)$ & Ganancia esperada del jugador $i$ dado $\sigma$ \\
$h \sqsubset h'$ & La historia (secuencia) $h$ es prefijo de la historia $h'$ \\ 
$\pi^{x}(h)$ & Probabilidad de alcanzar la historia $h$ dado $x$ \\
$\pi^c(h)$ & Probabilidad de que ocurra $h$ dado que todos los jugadores juegan para alcanzar $h$ \\
$\pi^x(I)$ & Probabilidad de alcanzar el conjunto de información $I$ dado $x$ \\
$\pi^x(h, h')$ & Probabilidad de ir de la historia $h$ a la historia $h'$ dado $x$ \\
$\varepsilon_{\sigma}$ & Explotabilidad del perfil estratégico mixto $\sigma$ \\
$R_i^T$ & \textit{Regret} promedio general del jugador $i$ a tiempo $T$ \\
$\bar\sigma^T_i$ & Estrategia promedio del tiempo $1$ al tiempo $T$ del jugador $i$ \\
$u_i(\sigma, I)$ & Utilidad contrafactual del jugador $i$, dado el conjunto de información $I$ y el perfil estratégico $\sigma$ \\
$R^T_{i, imm}(I)$ & Regret contrafactual inmediato del conjunto de información $I$
\end{tabular}
\endgroup

\mainmatter
\chapter*{Introducción}

La teoría de juegos puede ser definida como el estudio de modelos matemáticos de conflicto y cooperación entre agentes que deben tomar decisiones de forma racional e inteligente \cite[p.~1]{bib:game-theory-book}; estos modelos se denominarán \textbf{juegos}. Esta disciplina tiene aplicaciones en diversas áreas, incluyendo ciencias sociales, economía, matemática y ciencias de la computación.

Uno de los principales pioneros de esta disciplina fue John von Neumann, con su publicación \textit{Zur Theorie der Gesellschaftsspiele} (Sobre la Teoría de Juegos) en el año 1928 \cite{bib:von-neumann}. Asimismo, Jonh Forbes Nash Jr. con su publicación \textit{Non-cooperative Games} (Juegos no Cooperativos) en el año 1951 \cite{bib:nash}, introduce importantes conceptos, entre los cuales se encuentra el concepto de solución que hoy en día se conoce como equilibrio de Nash.

Aunque hay diferentes tipos de juegos, este trabajo se enfoca en juegos no deterministas con información incompleta. Con no determinismo se hace referencia a que los juegos incluyen incertidumbre probabilística, esta incertidumbre puede ocurrir, por ejemplo, al lanzar una moneda, repartir cartas de forma aleatoria o lanzar dados. Por otra parte, un juego con información incompleta permite modelar situaciones donde los jugadores tienen información parcial sobre algunas de las acciones que ya han sido tomadas \cite[p.~199]{bib:course-game-theory}.

El juego de póker (con sus diferentes versiones) es uno de los juegos más estudiados en esta categoría. Note que es un juego no determinista ya que se reparten cartas de forma aleatoria al inicio del mismo. Por otra parte, cada jugador desconoce las cartas que poseen los demás jugadores, por lo que poseen información parcial de la distribución inicial de las cartas. En contraste, juegos como el ajedrez, las damas o \textit{go}, son todos juegos deterministas (no hay elementos de azar) y además con información completa, pues todos los jugadores saben lo que ha ocurrido durante el juego y no hay información oculta entre ellos.

Uno de los retos para esta categoría de modelos consiste en determinar qué significa que un juego sea resuelto o que un jugador juegue de forma óptima. Para esto es necesario introducir el concepto de estrategias, las cuales indican las acciones o planes de acción que tomarán los jugadores en un momento determinado \cite[p.~24]{bib:teoria-juegos-es}. Luego, resolver un juego puede tener diferentes significados acorde al concepto de solución que se utilice, siendo el equilibrio de Nash uno los más importantes y el utilizado en el presente trabajo. Es importante destacar que en un equilibrio de Nash las acciones de los jugadores no son necesariamente deterministas, es decir, un jugador puede tomar decisiones diferentes ante el mismo escenario.

Por otra parte, Hart y Mas-Colell (2000) introducen el concepto de \textit{regret matching} \cite{bib:correlated-equilibrium}, en el cual los jugadores alcanzan un equilibrio teniendo en cuenta el \say{arrepentimiento} de sus jugadas previas, el cual se mide con una métrica denominada \textit{regret}, y haciendo las futuras jugadas proporcionales al \textit{regret} positivo. Este concepto es la base para el algoritmo \textit{Counterfactual Regret Minimization} (CFR), propuesto por Zinkevich, Johanson, Bowling y Piccione (2007) que permite encontrar una aproximación del equilibrio de Nash en cierto tipo de juegos con información incompleta, que sean de dos jugadores con suma cero \cite{bib:cfr}.

Dentro de este contexto, el objetivo de este proyecto de grado es comprender los conceptos en el área de juegos de dos personas que involucran información incompleta y no determinismo, así como implementar los algoritmos de \textit{Regret Matching} y CFR, realizando experimentos sobre distintos juegos que son capturados por el modelo. Con el fin de alcanzar el objetivo general se proponen los siguiente objetivos específicos:
\begin{itemize}
    \item Comprender los diferentes modelos de juegos y los elementos que los componen. Incluyendo juegos en forma normal y juegos en forma extensiva.
    \item Comprender los diferentes conceptos de solución para el tipo de juegos, como equilibrio correlacionado y equilibrio de Nash.
    \item Comprender los resultados teóricos más relevantes, y sus demostraciones, en relación a los modelos de juegos estudiados y los algoritmos implementados.
    \item Implementar los algoritmos \textit{Regret Matching} y \textit{Conterfactual Regret Minimization} que permiten encontrar equilibrios de Nash para el tipo de juego planteado.
    \item Implementar una clase general que permita representar los juegos que se quieren estudiar (independientemente de las reglas específicas de cada juego), así como diferentes juegos concretos que sean captados por el modelo.
    \item Realizar experimentos sobre los juegos propuesto utilizando los algoritmos implementados.
    \item Evaluar las estrategias obtenidas en cada uno de los juegos implementados.
\end{itemize}

Este libro se estructura en 5 capítulos. El Capítulo \ref{chapter:forma-normal} contiene el marco teórico de los juegos en forma normal o estratégica. Se presenta una definición formal de este tipo de juegos y los elementos que los componen. También se presentan dos conceptos de solución importantes: equilibrio de Nash y equilibrio correlacionado. El Capítulo \ref{chapter:juegos-forma-extensiva} contiene el marco teórico de los juegos en forma extensiva, se presentan los elementos en este tipo de juegos y se comparan con los juegos en forma normal. Además, se introduce una clasificación dentro de este tipo de juegos: juegos con \textit{perfect recall} o con \textit{imperfect recall}. Ambos capítulos contienen diversos ejemplos que ilustran los conceptos introducidos.

El Capítulo \ref{chapter:explotabilidad} presenta las propiedades que tienen los juegos de dos jugadores de suma cero, y explica por qué el equilibrio de Nash es importante en este tipo de juegos. Además, introduce dos nuevos conceptos de solución: estrategias \textit{minimax} y \textit{maximin}. Por último, se explica el concepto de explotabilidad que es la métrica que se utiliza para evaluar las estrategias obtenidas de forma experimental en los juegos.

El Capítulo \ref{chapter:regret-matching} presenta tres procedimientos que utilizan \textit{Regret Matching}, los cuales conducen a un equilibrio de Nash cuando los juegos son de dos jugadores de suma cero. Además, se presentan 4 juegos en forma normal y los resultados experimentales que se obtienen al aplicar los procedimientos sobre ellos. El Capítulo \ref{chapter:cfr} presenta el algoritmo CFR y una familia de este tipo de algoritmo, denominada \textit{Monte Carlo CFR} (MCCFR). Este capítulo también incluye 3 clases de juegos en forma extensiva y los resultados obtenidos al aplicar una versión de MCCFR sobre ellos. Finalmente, se presentan las conclusiones y las recomendaciones de este proyecto para las investigaciones futuras en el área.

La implementación de los procedimientos de \textit{Regret Matching}, junto con los resultados experimentales reportados en esta tesis se encuentran de forma pública \url{https://github.com/rubmary/regret-matching}. Similarmente, la implementación de los juegos y el algoritmo CFR, junto a las estrategias obtenidas y resultados experimentales se encuentran en \url{https://github.com/rubmary/cfr}.

Adicionalmente se desarrolló una aplicación web que permite observar la estrategia obtenida, la implementación se encuentra en \url{https://github.com/rubmary/domino-app}. Es importante destacar y agradecer la contribución de Samuel Nacache para el desarrollo de la interfaz gráfica de la aplicación.
%\chapter{Marco teórico}
%En este capítulo presentamos los conceptos básicos de juegos en forma normal y forma extensa, y los conceptos de solución considerados en esta tesis.
%\section{Juegos en Formal Normal o Estratégica}
\label{section:forma-normal}

En un juego en formal normal los jugadores eligen una única acción (o estrategia) de forma simultánea, obteniendo un pago de acuerdo a las acciones realizada por cada uno de ellos. Estos juegos también se llaman frecuentemente \textit{``one-shot game"}   (juegos de un sólo disparo), ya que cada uno de los jugadores realiza una única acción \cite{bib:introductionCFR}. El ejemplo clásico es \textit{piedra, papel o tijera}. En este juego cada jugador elige una de las tres opciones mediante un gesto con sus manos: piedra (con un puño cerrado), papel (con la mano extendida) o tijera (con los dedos índice y medio levantados en forma de ``V"). La piedra gana contra la tijera, la tijera gana contra el papel y el papel gana contra la piedra. Si los jugadores eligen la misma opción, entonces es un empate.

%El concepto de un juego en forma normal se formaliza en la Definición \ref{def:forma-normal} \cite{bib:correlated-equilibrium}.

\begin{definition}[\cite{bib:correlated-equilibrium}]
\label{def:forma-normal}
Un juego de N personas en \textbf{forma normal} (o estratégica) es una tupla $\Gamma = (N, (S_i)_{i \in N}, (u_i)_{i \in N})$, donde:
	\begin{itemize}[noitemsep]
		\item $N = \{1, 2, 3, \dots, N\}$ es el conjunto de jugadores.
		\item  $S_i$ es el conjunto de \textbf{estrategias puras} (o acciones) del jugador $i$.
		\item $u_i : \Pi _{i \in N} S_i \rightarrow \mathbb{R}$ es la función de pago del jugador $i$.
	\end{itemize}
\end{definition}

Otros conceptos básicos como: estrategias, perfiles estratégicos, soporte de una estrategia, ganancias esperada y mejor respuesta, definidos en \cite{bib:tutorial-existence-nash}, son presentados a lo largo de la sección.

\begin{definition} Un \textbf{perfil estratégico} (o perfil de acción) es una $N$-tupla formada por una estrategia para cada jugador. $S = \Pi_{i \in N}S_i$ es el conjunto de perfiles estratégicos y $s = (s_i)_{i \in N}$ representa un elemento genérico de $S$.  
\end{definition}

Se denotará con $s_{-i}$ la combinación de las estrategias de todos los jugadores excepto la del jugador $i$, es decir, $s_{-i} = (s_{i'})_{i' \neq i}$.

\textit{Piedra, papel o tijera}, es un juego para dos jugadores y las acciones (o estrategias puras) son las mismas para cada jugador: piedra (R), papel (P), tijera (S). Es decir, $S_1 = S_2 = \{R, P, S \}$. Estos juegos pueden representarse como una tabla $n$-dimensional, donde cada dimensión está asociada a un jugador y sus filas/columnas corresponden a las acciones de su jugador correspondiente. Cada una de las entradas de la tabla corresponden a un único perfil estratégico (pues representan la intersección de una única acción de cada jugador) y éstas contienen un vector de pagos para cada jugador \cite{bib:introductionCFR}. La Tabla \ref{table:pago-RPS} es la tabla de pagos correspondiente al juego piedra, papel o tijera.

\begin{table}[h]
\begin{center}
\caption[Tabla de pagos de piedra, papel o tijera]{Tabla de pagos de piedra, papel o tijera. \Blai{**** Usar captions con explicación para que se entienda el contenido sin tener que ir al texto ****}}
\label{table:pago-RPS}
\begin{tabular}{c | c | c | c |}
  & R & P & S \\ \hline
R & $0,0$ & $-1,1$ & $1,-1$ \\ \hline
P & $1,-1$ & $0,0$ & $-1,1$ \\ \hline
S & $-1,1$ & $1,-1$ & $0,0$ \\ \hline
\end{tabular}
\end{center}
\end{table}

%Sin embargo, 
En vez de realizar siempre la misma acción, un jugador puede elegir su jugada de acuerdo a una distribución de probabilidad, la cual se denominará \textbf{estrategia mixta}. Dado un conjunto finito $A$, se denotará con $\Delta(A)$ al conjunto de distribuciones de probabilidad sobre $A$, es decir $\Delta(A) = \{ (x_a)_{a \in A} : x_a \geq 0, \sum_{a \in A} x_a = 1\}$. Se presentan a continuación, definiciones formales para estrategia mixta, soporte de una estrategia mixta, perfil estratégico mixto y ganancia esperada.

\begin{definition} Una \textbf{estrategia mixta} del jugador $i$, denotada con $\sigma_i$ es una distribución de probabilidad sobre el conjunto $S_i$, es decir $\sigma_i \in \Delta(S_i)$. Denotaremos con $\sigma_i(s_i)$ la probabilidad que el jugador $i$ elija la acción $s_i \in S_i$. 
\end{definition}

\begin{definition}
El \textbf{soporte} (support) de una estrategia mixta $\sigma_i \in \Delta(S_i)$ del jugador $i$ es el conjunto de estrategias puras con una probabilidad positiva de ser elegidas, es decir:
\begin{alignat}{1}
\text{support}(\sigma_i) = \{s_i | \sigma_i(s_i) > 0 \} \,.
\end{alignat}
\Blai{*** conjunto por extensi\'on denotado con el s\'{\i}mbolo `:' y `$|$': elegir una misma forma de hacerlo a lo largo de la tesis ***}
\end{definition}

\begin{definition}
Un \textbf{perfil estratégico mixto} consiste en una estrategia mixta para cada jugador; es decir, $\sigma \in \Pi_{i \in N} \Delta(S_i)$ es una tupla $\sigma=(\sigma_i)_{i \in N}$.
\end{definition}

Para $\sigma = (\sigma_i)_{i \in N}$ y $s = (s_i)_{i \in N}$, $\sigma(s)$ denota la probabilidad que el perfil estratégico mixto elija la estrategia mixta $s$; i.e., $\sigma(s)=\prod_{i\in N} \sigma_i(s_i)$. Para un perfil $\sigma$ y jugador $i$, descomponemos $\sigma$ en $(\sigma_i,\sigma_{-i})$ como la combinaci\'on de la estrategia para el jugador $i$ y el perfil $\sigma_{-i}$ para el resto de los jugadores. Similarmente, $\sigma_{-i}(s_{-i})=\prod_{j\in N,j\neq i}\sigma_j(s_j)$ denota la probabilidad de que los jugadores diferentes al jugador $i$ elijan las estrategias mixtas en el perfil $\sigma_{-i}$. Finalmente, si $x$ es una estrategia pura para el jugador $i$, tambi\'en utilizamos $x$ para denotar la estrategia mixta $\sigma_i$ para el jugador $i$ tal que $\sigma_i(x)=1$.

La ganancia esperada del jugador $i$ asociada al perfil $\sigma$ es

\begin{definition}
\label{def:ganancia-esperada}
La \textbf{ganancia esperada} del jugador $i$ dado un perfil estratégico mixto $\sigma$ viene dada por
\begin{alignat}{1}
	u_i(\sigma)\ =\ \sum_{s \in S} u_i(s) \sigma(s)\ =\ \sum_{s \in S} u_i(s) \prod _{j \in N} \sigma_j(s_j)\ =\ \sum_{s \in S} u_i(s) \sigma_i(s_i) \sigma_{-i}(s_{-i})\,.
\end{alignat}
\end{definition}

\begin{lemma}
\label{lemma:1}
\Blai{Este lemma hace las cosas mas sencillas abajo...}
\begin{alignat}{1}
u_i(\sigma)\ =\ \sum_{x\in S_i} \sigma_i(x) \sum_{s_{-i}\in S_{-i}} \sigma_{-i}(s_{-i}) u_i(x,s_{-i}) \,.
\end{alignat}
\end{lemma}
\begin{proof}
\Blai{terminar statement of lemma, y hacer prueba}
\end{proof}

%La ganancia esperada es la esperanza de la función de pago obtenida al utilizar una estrategia mixta, permitiendo medir la efectividad de la estrategia utilizada. 
También es importante definir el concepto de mejor respuesta; término que se utiliza para caracterizar una estrategia que maximiza la ganancia esperada de un jugador fijo, conociendo las estrategias del resto de los jugadores.

\begin{definition}
\label{def:mejor-respuesta}
Sea $i\in N$ un jugador, $\sigma_i$ una estrategia mixta para el jugador $i$, y $\sigma_{-i}$ un perfil estrat\'egico mixto para el resto de los jugadores. Decimos que $\sigma_i$ es una \textbf{mejor respuesta} con respecto a $\sigma_{-i}$ si y s\'olo si
$u_i(\sigma_i,\sigma_{-i}) \geq u_i(\sigma'_i,\sigma_{-i})$ para toda estrategia mixta $\sigma'_i$ para el jugador $i$.
%Una estrategia $\sigma_i$ es \textbf{mejor respuesta} para el jugador $i$ si, dadas las estrategias de los otros jugadores, la ganancia esperada del jugador $i$ se maximiza con $\sigma_i$. Es decir, $\sigma_i$ es mejor respuesta a $\sigma_{-i}$, si para toda estrategia $\sigma_i' \in \Delta(S_i)$, se cumple que: 
%\begin{alignat}{1}
%	u_i(\sigma_i, \sigma_{-i}) \geq u_i(\sigma_i', \sigma_{-i})
%\end{alignat}
\end{definition}

Una mejor respuesta no es necesariamente única. En efecto, salvo el caso extremo en el que hay una única mejor respuesta, la cual es una estrategia pura, el número de mejores respuestas es infinito. Cuando el soporte de una estrategia mixta que es mejor respuesta incluye más de dos acciones (estrategias puras), el agente debe ser indiferente a cualquiera de éstas y cualquier mezcla de estas acciones también será mejor respuesta \cite{bib:tutorial-existence-nash}. %Esto se prueba en el Teorema \ref{theo:mejor-respuesta}.

\begin{lemma}
\label{lemma:2}
Sea $\sigma^*_i$ una estrategia mixta para el jugador $i$ que es mejor respuesta a $\sigma_{-i}$, y sea $x\in S_i$ una estrategia pura para el jugador $i$. Entonces, para toda estrategia pura $y\in S_i$ diferente de $x$,
\begin{alignat}{1}
  \sigma^*_i(x) \sum_{s_{-i}} \mu_i(x,s_{-i}) \sigma_{-i}(s_{-i})\ \geq\ \sigma^*_i(x) \sum_{s_{-i}} \mu_i(y,s_{-i}) \sigma_{-i}(s_){-i}) \,.
\end{alignat}
\end{lemma}

\begin{proof}
Considere la estrategia mixta $\sigma'_i$ definida por:
\begin{alignat}{1}
	\sigma'_{i}(s_i)\ =\  
	\begin{cases}
		0 &  \text{si } s_i = x \\
		\sigma^*_i(x) + \sigma^*_i(y) & \text{si } s_i = y \\
		\sigma^*_i(s_i) & \text{en otro caso} 
	\end{cases}
\end{alignat}
Utilizando el Lema~\ref{lemma:1} y el hecho que $\sigma^*_i$ es mejor respuesta a $\sigma_{-i}$:
\begin{alignat}{1}
  u_i(\sigma^*_i, \sigma_{-i})\ 
    &\geq\ u_i(\sigma'_i, \sigma_{-i}) \\
    &=\ \sum_{z \in S_i} \sigma'_i(z) \sum_{s_{-i}} u_i(z,s_{-i}) \sigma_{-i}(s_{-i}) \\
    &=\ \sum_{z\neq x} \sigma^*_i(z) \sum_{s_{-i}} u_i(z,s_{-i}) \sigma_{-i}(s_{-i}) + \sigma^*_i(x)\sum_{s_{-i}} u_i(y,s_{-i})\sigma_{-i}(s_{-i}) \,.
\end{alignat}
Por el Lema~\ref{lemma:1},
$u_i(\sigma^*_i, \sigma_{-i})=\sum_{z \in S_i} \sigma^*_i(z) \sum_{s_{-i}} u_i(z,s_{-i}) \sigma_{-i}(s_{-i})$. Entonces,
\begin{alignat}{1}
  \label{eq:ineq-ganancias}
  \sigma^*_i(x) \sum_{s_{-i}} u_i(x,s_{-i}) \sigma_{-i}(s_{-i})\
    &\geq\ \sigma^*_i(x)\sum_{s_{-i}} u_i(y,s_{-i}) \sigma_{-i}(s_{-i}) \,.
\end{alignat}
\end{proof}

\begin{theorem}
\label{theo:mejor-respuesta}
Sea $\sigma^*_i$ una estrategia mixta para el jugador $i$ que es mejor respuesta a $\sigma_{-i}$. Cualquier estrategia mixta $\sigma_i$ para el jugador $i$ cuyo soporte sea un subconjunto del soporte de $\sigma^*_i$ es tambi\'en una mejor respuesta a $\sigma_{-i}$
\end{theorem}

\begin{proof}
%Partiendo de la Definición \ref{def:mejor-respuesta} se obtiene
%\begin{alignat}{1}
%	u_i(\sigma)\  &=\ %\sum_{s \in S} u_i(s) \prod_{j \in N} \sigma_j(s_j)\ % \\
%	\sum_{s \in S} u_i(s) \sigma_i(s_i) \sigma_{-i}(s_{-i})\ =\ \sum_{s_i\in S_i} \sigma_i(s_i) \sum_{s_{-i}\in S_{-i}} \sigma_{-i}(s_{-i}) u_i(s_i,s_{-i})  \\ % \prod_{\substack{j \in N \\ j \neq i}}	\sigma_j(s_j) \\
%	&=\ \sum_{s \in S} \frac{1}{T}\sum_{t = 1}^{T}u_i(s) \sigma_i(s_i) \sigma_{-i}(s_{-i}) \\
%	&=\ \sum_{x \in S_i} \sigma_i(x) \sum_{ \substack{s \in S \\ s_i = x} } u_i(s) \sigma_{-i}(s_{-i})
%\end{alignat}
%
Sea $x \in S_i$ una \emph{estrategia pura} perteneciente al soporte de $\sigma^*_i$, y sea $y \in S_i$ una estrategia mixta \emph{diferente} de $x$. 
%Considere la estrategia mixta $\sigma'_i$ definida por:
%\begin{alignat}{1}
%	\sigma'_{i}(s_i)\ =\  
%	\begin{cases}
%		0 &  \text{si } s_i = x \\
%		\sigma^*_i(x) + \sigma^*_i(y) & \text{si } s_i = y \\
%		\sigma^*_i(s_i) & \text{en otro caso} 
%	\end{cases}
%\end{alignat}
%Como $\sigma^*_i$ es mejor respuesta,
%\begin{alignat}{1}
%  u_i(\sigma^*_i, \sigma_{-i})\ 
%    &\geq\ u_i(\sigma'_i, \sigma_{-i}) \\
%    &=\ \sum_{z \in S_i} \sigma'_i(z) \sum_{s_{-i}} u_i(z,s_{-i}) \sigma_{-i}(s_{-i}) \\
%    &=\ \sum_{z\neq x} \sigma^*_i(z) \sum_{s_{-i}} u_i(z,s_{-i}) \sigma_{-i}(s_{-i}) + \sigma^*_i(x)\sum_{s_{-i}} u_i(y,s_{-i})\sigma_{-i}(s_{-i}) \,.
%\end{alignat}
%Como $u_i(\sigma^*_i, \sigma_{-i})=\sum_{z \in S_i} \sigma^*_i(z) \sum_{s_{-i}} u_i(z,s_{-i}) \sigma_{-i}(s_{-i})$,
%\begin{alignat}{1}
%  \label{eq:ineq-ganancias}
%  \sigma^*_i(x) \sum_{s_{-i}} u_i(x,s_{-i}) \sigma_{-i}(s_{-i})\
%    &\geq\ \sigma^*_i(x)\sum_{s_{-i}} u_i(y,s_{-i}) \sigma_{-i}(s_{-i}) \,.
%\end{alignat}
%Por lo tanto, ya que $\sigma^*_i(x)>0$ puesto que $x\in support(\sigma^*_i)$,
Por el Lema~\ref{lemma:2},
\begin{alignat}{1}
  \sum_{s_{-i}} u_i(x,s_{-i}) \sigma_{-i}(s_{-i})\ \geq\  \sum_{s_{-i}} u_i(y,s_{-i}) \sigma_{-i}(s_{-i}) \,.
\end{alignat}
%\begin{alignat}{2}
%	&&\ u_i(\sigma^*_i, \sigma_{-i}) &\geq\ u_i(\sigma'_i, \sigma{-i}) \\
%	&\Rightarrow\quad
%	&\sum_{z \in S_i} \sigma^*_i(z) \sum_{s_{-i}} u_i(z,s_{-i}) \sigma_{-i}(s_{-i}) &\geq\ \sum_{z \in S_i} \sigma'_i(z) \sum_{s_{-i}} u_i(z,s_{-i}) \sigma_{-i}(s_{-i}) \\
	%
%	&\Rightarrow\quad
%	&\sum_{z \in S_i} \sigma^*_i(z) \sum_{s_{-i}} u_i(z,s_{-i}) \sigma_{-i}(s_{-i}) &\geq\ \sum_{z\neq x} \sigma^*_i(z) \sum_{s_{-i}} u_i(z,s_{-i}) \sigma_{-i}(s_{-i}) + \sigma^*_i(x)\sum_{s_{-i}} u_i(y,s_{-i})\sigma_{-i}(s_{-i}) \\
	%
	%& \Rightarrow\quad & \sum_{z \in S_i} \sigma_i(z) \sum_{\substack{s \in S \\ s_i = z}} u_i(s) \sigma_{-i}(s_{-i}) &\geq\ \sum_{\substack{z \in S_i \\ z \neq x}} \sigma_i(z) \sum_{\substack{s \in S \\ s_i = z}} u_i(s) \sigma_{-i}(s_{-i}) + \sigma_i(x)\sum_{\substack{s \in S \\ s_i = y}} u_i(s) \sigma_{-i}(s_{-i}) \\
    %\label{eq:ineq-ganancias}
%	&\Rightarrow\quad
%	&\sigma^*_i(x) \sum_{s_{-i}} u_i(x,s_{-i}) \sigma_{-i}(s_{-i}) &\geq\ \sigma^*_i(x)\sum_{s_{-i}} u_i(y,s_{-i}) \sigma_{-i}(s_{-i}) \\
	%
%	&\Rightarrow\quad
%	&\sum_{s_{-i}} u_i(x,s) \sigma_{-i}(s_{-i}) &\geq\  \sum_{s_{-i}} u_i(y,s) \sigma_{-i}(s_{-i}) \,.
%\end{alignat}
En particular, si $x$ y $x'$ son distintos, y ambos pertenecen al soporte de $\sigma_i$,
\begin{alignat}{1}
\sum_{s_{-i}} u_i(x,s_{-i}) \sigma_{-i}(s_{-i})\ =\ \sum_{s_{-i}} u_i(x',s_{-i}) \sigma_{-i}(s_{-i})\ =\ C
\end{alignat}
donde $C$ es una constante que s\'olo depende de $\sigma_{-i}$.
Luego, para cualquier estrategia $\sigma_i$, tal que $support(\sigma_i) \subseteq support(\sigma^*_i)$, se tiene:
\begin{alignat}{1}
u_i(\sigma_i, \sigma_{-i})\ &=\ \sum_{x \in S_i} \sigma_i(x) \sum_{s_{-i}} u_i(x,s_{-i}) \sigma_{-i}(s_{-i})\ =\ \sum_{x \in S_i} \sigma_i(x) C\ =\ C \,.
% &=\ c \sum_{x \in S_i} \sigma^*_i(x) \\
% &=\ c \\
% &=\ u_i(\sigma_i, \sigma_{-i})
\end{alignat}
En particular, $u_i(\sigma^*_i,\sigma_{-i})=C$, y $\sigma_i$ es también mejor respuesta a $\sigma_{-i}$.
\end{proof}

Cuando cada jugador juega con una mejor respuesta frente a las estrategias del resto de los jugadores, se obtiene un Equilibrio de Nash (Definición \ref{def:equilibrio-nash}). En un Equilibrio de Nash ningún jugador puede mejorar su ganancia esperada cambiando su estrategia de forma aislada. Por otra parte, si el juego es finito, siempre existe, al menos, un equilibrio de Nash (Teorema \ref{theo:existencia-nash}). Un juego es finito se el n\'umero de jugadores es finito, y si el conjunto de estrategias puras para cada jugador es tambi\'en finito. E, concepto de equilibrio de Nash es uno de los conceptos de solución más importantes en el \'area de juegos no cooperativos, y es el principal concepto de solución utilizado en el presente trabajo.

\begin{definition}
\label{def:equilibrio-nash} Un perfil estratégico mixto $\sigma$ es un \textbf{equilibrio de Nash} si y s\'olo si para todo jugador $i$, la estrategia $\sigma_i$ es mejor respuesta para $\sigma_{-i}$.
\end{definition}

\begin{theorem}[\cite{bib:tutorial-existence-nash}]
\label{theo:existencia-nash}
Todo juego finito tiene al menos un equilibrio de Nash.
\end{theorem}
 % REORGANIZED
%\input{capitulo-1/forma-extensa} % REORGANIZED
%\section{Equilibrio correlacionado}

Aunque el equilibrio de Nash es uno de los principales conceptos de solución, es importante destacar que éste no garantiza el mejor resultado si los jugadores toman sus decisiones en conjunto. Si los jugadores pueden correlacionar sus acciones, pueden existir estrategias con mayores ganancias para ellos. 
Esto puede ser obtenido mediante un equilibrio correlacionado (Definición \ref{def:equilibrio-correlacionado}, \cite{bib:correlated-equilibrium}), una generalización del equilibrio de Nash. Todo equilibrio de Nash es un equilibrio correlacionado, pero este último permite otras soluciones importantes \cite{bib:correlated-equilibrium}. La relación entre los conceptos de equilibro de Nash y correlacionado se muestra en los Teoremas \ref{theo:nash-correlacionado} y \ref{theo:correlacionado-nash}.

\begin{definition}
\label{def:equilibrio-correlacionado}
Una distribución $\psi\in\Delta(S)$ es un \textbf{equilibrio correlacionado} si y sólo si para cualquier jugador $i$, y para cualesquiera estrategias puras $x, y \in S_i$,
\begin{alignat}{1}
\label{eq:equilibrio-correlacionado}
\sum_{s_{-i}\in S_{-i}} \psi(x,s_{-i}) [ u_i(x,s_{-i}) - u_i(y,s_{-i})]\ \geq\ 0 \,.
\end{alignat}
\end{definition}

Si en la desigualdad \eqref{eq:equilibrio-correlacionado} se cambia el $0$ por un $\epsilon > 0$ se obtiene la definición de $\epsilon$-equilibrio correlacionado.


\begin{theorem}
\label{theo:nash-correlacionado}
Si $\sigma$ es un equilibrio de Nash, entonces $\sigma$ es un equilibrio correlacionado.
\end{theorem}
\begin{proof}
Sea $\sigma$ un equilibrio de Nash, sean $x,y\in S_i$ estrategias puras distintas cualesquiera para el jugador $i$, y sea $\sigma'_i$ una estrategia mixta cualquiera para el jugador $i$. Por el Lema~\ref{lemma:2},
\begin{alignat}{1}
  \sigma_i(x) \sum_{s_{-i}} u_i(x,s_{-i}) \sigma_{-i}(s_{-i})\ \geq\  \sigma_i(x)\sum_{s_{-i}} u_i(y,s_{-i}) \sigma_{-i}(s_{-i}) \,.
\end{alignat}
Es decir,
\begin{alignat}{1}
  0\ \leq\ \sigma_i(x) \sum_{s_{-i}} \sigma_{-i}(s_{-i}) [u_i(x,s_{-i}) - u_i(y,s_{-i})]\ =\ \sum_{s_{-i}} \sigma(x,s_{-i}) [u_i(x,s_{-i}) - u_i(y,s_{-i})] \,.
\end{alignat}
%\begin{alignat}{2}
%  &\Rightarrow\quad
%  &&\sigma_i(x)\sum_{\substack{s \in S \\ s_i = y}} u_i(s) \sigma_{-i}(s_{-i})  \leq\ \sigma_i(x) \sum_{\substack{s \in S \\ s_i = x}} u_i(s) \sigma_{-i}(s_{-i}) \\
%	& \Rightarrow\quad & 0\ &\leq\ \sigma_i(x) \sum_{\substack{s \in S \\ s_i = x}} u_i(s) \sigma_{-i}(s_{-i}) -  \sigma_i(x)\sum_{\substack{s \in S \\ s_i = y}} u_i(s) \sigma_{-i}(s_{-i}) \\
%	& \Rightarrow\quad & 0\ &\leq\  \sum_{\substack{s \in S \\ s_i = x}} u_i(s) \sigma(s) -  \sum_{\substack{s \in S \\ s_i = y}} u_i(s) \sigma(s) \\
%	& \Rightarrow\quad & 0\ &\leq\  \sum_{\substack{s \in S \\ s_i = x}} \sigma(s) [u_i(s)  -  u_i(y, s)]
%\end{alignat}
Luego, $\sigma$ es un equilibrio correlacionado.
\end{proof}

\begin{theorem}
\label{theo:correlacionado-nash}
Sea $\psi\in\Delta(S)$ un equilibrio correlacionado. Si $\psi$ se factoriza como $\psi=\prod_{i\in N} \sigma_i$ donde $\{\sigma_i\}_{i\in N}$ es un conjunto de estrategias mixtas para cada jugador (i.e., $\psi(s)=\prod_{i \in N} \sigma_i(s_i)$ para todo $s\in S$), entonces $\psi$ es un equilibrio de Nash.
\end{theorem}

\begin{proof}
Sea $\psi= \prod_{i \in N} \sigma_i$ un equilibrio correlacionado en forma factorizada. Debemos mostrar que para cualquier jugador $i$ y estrategia mixta $\sigma'_i$ para el jugador $i$, se cumple $u_i(\sigma) \geq u_i(\sigma'_i, \sigma_{-i})$.

Sean $x$ y $y$ estrategias puras para el jugador $i$.
Como $\sigma$ es un equilibrio correlacionado,
\begin{alignat}{1}
\label{eq:1:theo:correlacionado-nash}
0\ \leq\ %\sum_{\substack{s \in S \\ s_i = x}} \sigma(s)[u_i(s) - u_i(y, s_{-i})] = 
\sigma_i(x) \sum_{s_{-i}} \sigma_{-i}(s_{-i})[u_i(x, s_{-i}) - u_i(y, s_{-i})] \,.
\end{alignat}
Al sumar sobre $x\in S_i$ obtenemos, 
\begin{alignat}{2}
\label{eq:2:theo:correlacionado-nash}
0\ \leq\ \sum_{x\in S_i} \sum_{s_{-i}} \sigma(x,s_{-i}) [u_i(x, s_{-i}) - u_i(y, s_{-i})]\ =\ \sum_s \sigma(s) [u_i(s) - u_i(y, s_{-i})] \,.
\end{alignat}
Si $x^* \in S_i$ es tal que $\sigma_i(x^*)>0$, obtenemos de \eqref{eq:1:theo:correlacionado-nash} al multiplicar por $\sigma'_i(y)$ y sumar sobre $y\in S_i$:
\begin{alignat}{1}
\label{eq:3:theo:correlacionado-nash}
\sum_{y \in S_i} \sigma'_i(y) \sum_{s_{-i}} \sigma_{-i} (s_{-i}) [u_i(x^*, s_{-i}) - u_i(y, s_{-i})]\ =\ \sum_{s} \sigma'(s) [u_i(x^*, s_{-i}) - u_i(s)]\ \geq\ 0
\end{alignat}
donde $\sigma'$ denota la estrategia $\sigma'=(\sigma'_i,\sigma_{-i})$. 
Al sumar \eqref{eq:2:theo:correlacionado-nash} y
\eqref{eq:3:theo:correlacionado-nash}, obtenemos que para cualquier $y$ y $x^*$ tal que $\sigma_i(x^*)>0$:
\begin{alignat}{1}
\label{eq:4:theo:correlacionado-nash}
\sum_{s \in S} u_i(s) [\sigma(s) - \sigma'(s)] - \sum_{s \in S} \sigma(s)u_i(y, s_{-i}) + \sum_{s \in S} \sigma'(s) u_i(x^*, s_{-i})\ \geq\ 0\ \,.
\end{alignat}
Por otra parte, note que:
\begin{alignat}{1}
  \sum_{s \in S} \sigma(s) u_i(x^*,s_{-i}) - \sum_{s \in S} &\sigma'(s) u_i(x^*,s_{-i}) \\
    &\qquad=\ \sum_{s_{-i}} u_i(x^*,s_{-i}) \sigma_{-i}(s_{-i}) \sum_{z\in S_i} [\sigma_i(z) - \sigma'_i(z)] \\
    &\qquad=\ \sum_{s_{-i}} u_i(x^*,s_{-i}) \sigma_{-i}(s_{-i}) \biggl[ \sum_{z\in S_i} \sigma_i(z) - \sum_{z\in S_i} \sigma'_i(z) \biggr] \\
    &\qquad=\ 0 \,.
\end{alignat}
Luego, al tomar $y=x^*$ en \eqref{eq:4:theo:correlacionado-nash},
\begin{alignat}{1}
 \sum_{s \in S} u_i(s) [\sigma(s) - \sigma'(s)]\ =\ \sum_{s \in S} u_i(s)\sigma(s) - \sum_{s \in S} u_i(s)\sigma'(s)\ =\ u_i(\sigma) - u_i(\sigma'_i, \sigma_{-i})\ \geq\ 0 \,.
\end{alignat}
Como $\sigma'_i$ es una estrategia cualquiera para el jugador $i$, $\sigma$ es un equilibrio de Nash.
\end{proof}

A diferencia del conjunto de equilibrios de Nash, el cual es un conjunto matemáticamente complejo (un conjunto de puntos fijos), el conjunto de equilibrios correlacionados en un conjunto bastante simple. En particular, el conjunto de equilibrios correlacionado es un politopo (generalización de un polígono en $\mathbb{R}^N$) convexo. Por lo tanto puede esperarse que existan procedimientos simples para calcular equilibrios correlacionados \cite{bib:correlated-equilibrium}.

\begin{theorem}
Sean $\sigma$ y $\sigma'$ dos equilibrios correlacionados, y $\alpha$ un número real en $(0,1)$. Entonces, la distribucion $\alpha\sigma + (1-\alpha)\sigma'$ es un equilibrio correlacionado.
\end{theorem}
\begin{proof}
\Blai{hacer demostracion}
\end{proof}

\subsection{Blackwell's Approachability Theorem}
Antes de presentar los procedimientos que llevan a equilibrios correlacionados, es importante enunciar el Teorema de Aproximación de Blackwell, \textit{Blackwell's Approachability Theorem}, base para la obtención de los procedimientos que calculan equilibrios correlacionados. El enunciado del teorema, las definiciones utilizadas y los procedimientos mostrados son presentados en \cite{bib:correlated-equilibrium}.

El marco teórico en el cual se aplica el teorema está conformado por: (1)~un \textbf{decididor} $i$ que toma decisiones de un conjunto finito de acciones $S_i$, (2)~un \textbf{oponente} $-i$ que toma decisiones de un conjunto finito de acciones $S_{-i}$, (3)~un \textbf{conjunto indexado} denotado por $L$, y (4)~un \textbf{vector de pagos} $v(s_i, s_{-i}) \in \mathbb{R}^{|L|}$.
El decididor y oponente toman decisiones $s_t=(s^t_i,s^t_{-i})\in S_i\times S_{-i}$ indexadas en tiempo $t\geq 1$. El problema planteado consiste en ver si el decididor puede garantizar que el promedio de pagos $D_t$ a tiempo $t$, definido por $D_t=\frac{1}{t}\sum_{\tau=1}^t v(s_\tau)=\frac{1}{t}\sum_{\tau=1}^t v(s^\tau_i,s^\tau_{-i})$, \emph{alcanza} el conjunto $\mathbb{R}^{|L|}$.
Antes de enunciar el teorema es necesario presentar las definiciones de distancia de un punto a un conjunto (Definición \ref{def:distancia}), un conjunto alcanzable (Definición \ref{def:alcanzable}) y de función de soporte (Definición \ref{def:funcion-soporte}).

\begin{definition}
\label{def:distancia}
Sea $A$ un conjunto cerrado y convexo en $\mathbb{R}^n$, y $x \in \mathbb{R}^n$ un punto cualquiera. La \textbf{distancia} de $x$ a $A$ es definida por
\begin{alignat}{1}
\text{dist}(x, A)\ =\ \min\{ \|x - a\| : a \in A \}
\end{alignat}
donde $\|\cdot\|$ denota la distancia euclidiana en $\mathbb{R}^n$.
\end{definition}

\begin{definition}
\label{def:alcanzable}
Sea $\mathcal{C}$ un conjunto convexo y cerrado en $\mathbb{R}^{|L|}$. El conjunto $\mathcal{C}$ es \textbf{alcanzable} por el decididor $i$ si hay un procedimiento para $i$ que garantiza que $D_t$ alcanza a $\mathcal{C}$; es decir. $dist(D_t, \mathcal{C}) \rightarrow 0$ (a.s.) sin importar la elección del oponente $-i$.
\end{definition}

\begin{definition}
\label{def:funcion-soporte}
Sea $\mathcal{C} \in \mathbb{R}^n$ un conjunto. La \textbf{función de soporte} $w_{\mathcal{C}}$ para el conjunto $\mathcal{C}$, es definida por
\begin{alignat}{1}
	w_{\mathcal{C}}(\lambda)\ =\ \sup\{\lambda \cdot c : c \in \mathcal{C} \}
\end{alignat}
donde $\cdot$ denota el producto interno en $\mathbb{R}^n$.
\end{definition}

Dado un conjunto convexo y cerrado $\mathcal{C}$ denotaremos con $F(x)$ el punto (único) más cercano a $x$ de $C$, y con $\lambda(x)= x - F(x)$ el \emph{vector normal exterior} a $\mathcal{C}$ en el punto $x$ \Blai{*** def? ***}.
El Teorema de Blackwell (Te)orema~\ref{theo:blackwell} establece una condición necesaria y suficiente para el problema planteado previamente.

\begin{theorem}
\label{theo:blackwell}
Sea $\mathcal{C} \subseteq \mathbb{R}^{|L|}$ un conjunto convexo y cerrado con función de soporte $w_{\mathcal{C}}$. Entonces, $\mathcal{C}$ es alcanzable por $i$ si y sólo si para todo $\lambda \in \mathbb{R}^{|L|}$, existe una estrategia mixta $q_{\lambda} \in \Delta(S_i)$ para el decididor $i$ tal que para todo $s_{-i}\in S_{-i}$:
\begin{alignat}{1}
  \lambda \cdot v(q_{\lambda}, s_{-i})\ \leq\ w_{\mathcal{C}}(\lambda) \,.
\end{alignat}
En esta expresión, $v(q, s_{-i})$ denota $\sum_{s_i \in S_i} q(s_i)u_i(s_i, s_{-i})$. 
Además, el siguiente procedimiento garantiza que $dist(D_t, \mathcal{C}) \rightarrow 0$ (a.s.) cuando $t \rightarrow \infty$: en el tiempo $t+1$, jugar $q_{\lambda(D_t)}$ si $D_t \notin \mathcal{C}$, y jugar arbitrariamente si $D_t \in \mathcal{C}$.
\end{theorem}

\Blai{**** Algún ejemplo? ****}

\subsection{Regret Matching}
Una vez enunciado el Teorema de Aproximación de Blackwell, es posible plantear la siguiente pregunta ?`Hay procedimientos adaptativos simples que siempre lleven a un equilibrio correlacionado? A continuación se describen tres procedimientos, dos de los cuales llevan a equilibrios correlacionados \cite{bib:correlated-equilibrium}.

\subsubsection{Procedimiento A: Regret condicional}

Sea $\Gamma$ un juego en forma normal el cual es jugado repetidamente a través del tiempo $t = 1, 2, \ldots $. 
Sea $h_t = (s^\tau)_{\tau = 1}^t \in \prod_{\tau = 1}^{t} S$ la historia del juego al inicio del tiempo $t+1$. El jugador $i \in N$ elige su estrategia con una distribución de probabilidad $p_{t+1}^i \in \Delta(S_i)$, definida de la siguiente manera.

Para cada par de estrategias $j, k \in S_i$, supongamos que el jugador $i$ remplaza la estrategia $j$ (cada vez que la jugó en el pasado) por la estrategia $k$. Luego, su ganancia a tiempo $1\leq \tau \leq t$ hubiera sido:
\begin{alignat}{1}
W_i^{\tau}(j,k)\ =\ 
\begin{cases}
u_i(k, s_{-i}^{\tau}) &\text{ si } s_i = j \,, \\
u_i(s^\tau) & \text{en otro caso.} 
\end{cases}
\end{alignat}
La diferencia resultante en el promedio de la función de pago, denotada con $D_i^t(j, k)$, para el jugador $i$ sería:
\begin{alignat}{1}
  D_i^t(j, k)\ 
    =\ \frac{1}{t} \sum_{\tau = 1}^{t} W_i^{\tau}(j, k) - \frac{1}{t} \sum_{\tau = 1}^{t} u_i(s^{\tau})\ 
	=\ \frac{1}{t} \sum_{\substack{1\leq \tau \leq t \\s^\tau_i = j}} u_i(k, s_{-i}^{\tau}) - u_i(s^{\tau}) \,.
\end{alignat}
Finalmente, definimos
\begin{alignat}{1}
\label{eq:regret}
R_i^t(j, k)\ =\ [D_i^t(j, k)]^+\ =\ \max(0, D_i^t(j, k)) \,.
\end{alignat}

La expresión \eqref{eq:regret} se puede interpretar como una medida de ``arrepentimiento'' del jugador $i$ de haber elegido la acción $j$ en vez de la acción $k$ en el pasado, y por lo tanto, dicha medida es denominada \textit{regret}.

Fijemos un número $\mu > 0$ suficientemente grande. Sea $j \in S_i$ la última estrategia jugada por el jugador $i$, es decir $j = s_i^t$. Luego, la distribución de probabilidad $p_{t+1}^i \in \Delta(S_i)$ usada por el jugador $i$ a tiempo $t+1$ es definida como:
\begin{alignat}{1}
\label{eq:proc-A}
  \begin{cases}
    p_{t+1}^i(k)\ :=\  \frac{1}{\mu} R_t^i(j, k) & \text{ si } k \neq j \,, \\
    p_{t+1}^i(j)\ :=\ 1 - \sum_{k \in S_i, k \neq j} p_{t+1}^i(k) \,.
  \end{cases}
\end{alignat}
La distribución inicial $p_{1}^i \in \Delta(S_i)$, a tiempo $t=1$, es elegida de forma arbitraria.

Para cada tiempo $t$, sea $z_t \in \Delta(S)$ la distribución empírica de las $N$-tuplas jugadas hasta tiempo $t$, es decir:
$z_t(s) := \frac{1}{t} |\{1\leq\tau \leq t : s^{\tau} = s \}|$. El siguiente teorema enuncia que el procedimiento arriba descrito produce un equilibrio correlacionado.

\begin{theorem}[\cite{bib:correlated-equilibrium}]
\label{theo:conv-proc-A}
Si cada jugador juega de acuerdo al procedimiento descrito por \eqref{eq:proc-A}, entonces la distribución empírica del juego $z_t$ converge (a.s.) cuando $t \rightarrow \infty$ al conjunto de equilibrios correlacionado del juego $\Gamma$.
\end{theorem}

Es importante destacar que $z_t$ no tiene que converger necesariamente a un punto equilibrio correlacionado. El Teorema~\ref{theo:conv-proc-A} es equivalente al siguiente enunciado: para todo $\varepsilon > 0$, existe un tiempo $T_0 = T_0(\varepsilon)$ tal que para todo $t \geq T_0$ podemos encontrar un equilibrio correlacionado $\psi_t$ que está distancia menor que $\varepsilon$ de $z_t$.

En el procedimiento descrito cada jugador tiene dos opciones en cada período: continuar jugando con la última estrategia, o cambiarla por otra estrategia cuyas probabilidades son proporcionales a cuanto mayor hubiese sido su ganancia acumulada si hubiese hecho ese cambio en el pasado. El procedimiento planteado es simple, tanto de entender y explicar, como de implementar. Además en cada período no sólo se elige la mejor respuesta, todas las respuestas mejores a la actual pueden ser escogidas con probabilidades que son proporcionales a sus ganancias aparentes (medidas por el \textit{regret}). Este tipo de procedimientos son llamados procedimientos de \textit{regret matching}. Por último, el procedimiento tiene inercia: la estrategia jugada previamente importa, siempre hay una probabilidad positiva de continuar jugando la misma estrategia, y más aún, sólo se cambiará de estrategia si hay una razón para hacerlo.

La siguiente proposición conecta el conjunto de equilibrios correlacionados con la noción de regret: \Blai{*** La relevancia de la siguiente proposición, así como su prueba, no están claras. Revisar esto. Me imagino que lo que se quiere establecer es que el regret tiendo a cero cuando se utiliza el procedimiento, pero esto no esta claro ***}

\begin{proposition}
\label{prop:no-regret}
Sea $(s_t)_{t = 1, 2, ...}$ una secuencia de juegos de $\Gamma$, y sea $\varepsilon \geq 0$ un número no negativo.
Entonces, $\limsup_{t \rightarrow \infty} R_i^t(j, k) \leq \varepsilon$ para cada $i$ y cada $j, k \in S_i$, con $j \neq k$, si y sólo si la secuencia de distribuciones empíricas $z_t$ converge al conjunto de $\varepsilon$-equilibrio correlacionado.
\end{proposition}

\begin{proof}
Note que:
\begin{alignat}{1}
  D_i^t(j, k)\ 
    &=\ \frac{1}{t} \sum_{\substack{1\leq \tau \leq t \\ s_i^{\tau}=j}} u_i(k, s_{-i}^{\tau}) - u_i(s^{\tau}) \\
    &=\ \sum_{ \substack{s \in S \\ s_i = j}} \frac{1}{t} |\{1\leq\tau \leq t : s^{\tau} = s\}|\,[u_i(k, s_{-i}) - u_i(s)] \\
    &=\ \sum_{ \substack{s \in S \\ s_i = j}} z_t(s)\,[u_i(k, s_{-i}) - u_i(s)] \,.
\end{alignat}
Entonces, para cualquier subsecuencia $z_{t'}$ de $z_t$ tal que $z_{t'} \rightarrow \psi \in \Delta(s)$ (es decir, converge) se tiene que:
\begin{alignat}{1}
	D'_{t'}(j, k)\ \longrightarrow\ \sum_{\substack{s\in S \\ s_i = j}}\psi(s)[u_i(k, s_{-i}) - u_i(s)] \,.
\end{alignat}
Luego, el resultado es inmediato de la definición de $\varepsilon$-equilibrio correlacionado y de la definición de $R_i^t(j, k)$.
\end{proof}

\subsubsection{Procedimiento B: Vector invariante de probabilidad}

Este procedimiento es una variación del anterior. Sin embargo, a tiempo $t+1$ las probabilidades de transición de la estrategia utilizada por el jugador $i$ son determinadas por la matriz estocástica (derecha) $M^i_t$ definida en \eqref{eq:proc-A}; i.e., $M^i_t(j,k)=\frac{1}{\mu}R^i_t(j,k)$ si $k\neq j$, y $M^i_t(j,j)=1-\frac{1}{\mu}\sum_{k\in S_i,k\neq j} R^i_t(j,k)$.

Considere un vector (fila) invariante de probabilidad $q^i_t$, donde $q^i_t\in \Delta(S_i)$, para la matriz $M^t$. Es decir, $q^i_t$ satisface $q^i_t \times M^i_t = q^i_t$ (dicho vector simpre existe):
\begin{alignat}{1}
  q^i_t(j)\ 
    =\ \sum_{k\in S_i} q^i_t(k) M^i_t(k,j)\ 
    =\ \bigg[\sum_{k \in S_i, k \neq j} q^i_t(k)\frac{1}{\mu}R^i_t(k,j)\bigg] + q_i^t(j)\biggl[1 - \frac{1}{\mu}\sum_{k \in S_i, k \neq j} R^i_t(j,k)\biggr]
\end{alignat}
para todo $j \in S_i$. Definamos $R_t^i(j, j) = 0$, luego:
\begin{alignat}{3}
  &
  & q^i_t(j)\ &=\ \biggl[\sum_{k \in S_i} q^i_t(k)\frac{1}{\mu}R^i_t(k,j)\biggr] + q^i_t(j)\biggl[1 - \sum_{k \in S_i} \frac{1}{\mu} R^i_t(j,k)\biggr] \\
  &\Rightarrow\quad
  &\mu q_t^i(j)\ &=\ \biggl[\sum_{k \in S_i}q^i_t(k)R^i_t(k,j)\biggr] + q^i_t(j)\biggl[\mu - \sum_{k \in S_i} R^i_t(j, k)\biggr] \\
  &\Rightarrow\quad
  &\mu q^i_t(j)\ & =\ \biggl[\sum_{k \in S_i}q^i_t(k)R^i_t(k,j)\biggr] + \mu q^i_t(j) - q^i_t(j)\sum_{k\in S_i} R^i_t(j,k) \,.
\end{alignat}
Por lo tanto,
\begin{alignat}{1}
\label{eq:proc-B}
q^i_t(j)\sum_{k \in S_i} R^i_t(j,k)\ =\ \sum_{k \in S_i} q_t^i(k)R_i^t(k,j) \,.
\end{alignat}

\begin{theorem}
Supongamos que a cada período $t+1$, el jugador $i$ elige las estrategias acorde a un vector de distribución de probabilidad $q_t^i$ que satisface \eqref{eq:proc-B}. Entonces, $R^i_t(j, k)$ converge a cero (a. s.) para todo $j, k \in S_i$ con $j \neq k$.
\end{theorem}

\begin{proof}
La prueba es una aplicación directa del Teorema de Aproximación de Blackwell con $L$, $v$ y $\mathcal{C}$ definidos de la siguiente manera:
\begin{itemize}[noitemsep]
  \item $L = \{ (j, k) \in S_i \times S_{i}  : j \neq k \}$
  \item $v(s_i, s_{-i}) \in \mathbb{R}^L$ dado por
    \begin{alignat}{1}
      [v(s_i, s_{-i})](j, k)\ =\  
        \begin{cases}
          u_i(k, s_{-i}) - u_i(j, s_{-i}) & \text{si } s_i = j \\
          0 & \text{en otro caso,}
        \end{cases}
    \end{alignat}
  \item $\mathcal{C} = \mathbb{R}^L_{-} = \{x \in \mathbb{R}^L : x_i \leq 0\ \forall i \in L \}$ es decir, el ortante negativo.
\end{itemize}
Demostraremos que $\mathcal{C}$ es alcanzable por $i$.
Note que:
\begin{alignat}{1}
	w_{\mathcal{C}}(\lambda)\ =\ \sup\{\lambda \cdot c : c \in \mathcal{C} \}\ =\ \sup \{\sum_{i \in L} \lambda_i c_i : c_i \leq 0 \} \,.
\end{alignat}
Luego, si $\lambda_i \geq 0$, $\forall i \in L$, entonces $\lambda \cdot c \leq 0$ para todo $c \in \mathcal{C}$, y $w_{\mathcal{C}}(\lambda) = 0$. Por otra parte, si $\lambda_i < 0$ para algún $i\in N$, entonces $c_i \lambda_i$ no está acotado superiormente y  $w_{\mathcal{C}}(\lambda) = \infty$. Luego,
\begin{alignat}{1}
  w_{\mathcal{C}}\ =\  
	\begin{cases}
	  0 & \text{si } \lambda \in \mathbb{R}^L_+ \,, \\
	  \infty & \text{en caso contrario.}
	\end{cases}
\end{alignat}
Por otra parte, se tiene que:
\begin{alignat}{1}
	\lambda \cdot v(q_{\lambda}, s_{-i})\ 
	  &=\ \sum_{(j,k) \in L} \lambda(j,k) \cdot [v(q_{\lambda}, s_{-i})](j, k) \\
	&=\ \sum_{(j,k) \in L} \lambda(j, k)\left[\sum_{s_i \in S_i} q_{\lambda}(s_i) v(s_i, s_{-i}) \right](j, k) \\
	&=\ \sum_{(j,k) \in L} \lambda(j, k) q_{\lambda}(j) [v(j, s_{-i})](j, k) \\
	&=\ \sum_{(j,k) \in L} \lambda(j, k) q_{\lambda}(j) [u_i(k, s_{-i}) - u_i(j, s_{-i})] \\
	&=\ \sum_{(j,k) \in L} \lambda(j, k) q_{\lambda}(j)u_i(k, s_{-i}) - \sum_{(j,k) \in L} \lambda(j, k) q_{\lambda}(j)u_i(j, s_{-i}) \\
	&=\ \sum_{k \in S_i} u_i(k, s_{-i}) \sum_{j \in S_i} \lambda(j, k) q_{\lambda}(j) - \sum_{j \in S_i} q_{\lambda}(j)u_i(j, s_{-i}) \sum_{k \in S_i} \lambda(j, k) \\
	&=\ \sum_{j \in S_i} u_i(j, s_{-i}) \sum_{k \in S_i} \lambda(k, j) q_{\lambda}(k) - \sum_{j \in S_i} q_{\lambda}(j)u_i(j, s_{-i}) \sum_{k \in S_i} \lambda(j, k) \\
	&=\ \sum_{j \in S_i} u_i(j, s_{-i}) \left[ \sum_{k \in S_i} \lambda(k, j) q_{\lambda}(k) - q_{\lambda}(j) \sum_{k \in S_i} \lambda(j, k) \right] \,.
\end{alignat}
Defina
\begin{alignat}{1}
  \alpha(j)\ =\ \sum_{k \in S_i} \lambda(k, j) q_{\lambda}(k) - q_{\lambda}(j) \sum_{k \in S_i} \lambda(j, k) \,.
\end{alignat}
Entonces, $\lambda \cdot v(q_{\lambda}, s_{-i}) = \sum_{j \in S_i} u_i(j, s_{-i}) \alpha(j)$.
Luego, en este caso, la condición del Teorema \ref{theo:blackwell} es equivalente a:
\begin{alignat}{1}
	\sum_{j \in S_i} u_i(j, s_{-i}) \alpha(j)\ \leq\ 0 \,.
\end{alignat}
Si se elige $q_{\lambda}$ que cumpla:
\begin{alignat}{1}
  q_{\lambda}(j) \sum_{k \in S_i} \lambda(j, k)\ =\ \sum_{k \in S_i} \lambda(k, j) q_{\lambda}(k)
\end{alignat}
para todo $j \in S_i$, entonces $\alpha(j)=0$ para $j\in S_i$, y la condición del Teorema~\ref{theo:blackwell} se cumple como igualdad cuando $\mathcal{C} = \mathbb{R}^{L}_-$.

Por otra parte, sea $D_t=\frac{1}{t}\sum_{\tau=1}^t v(s_\tau)$ el promedio de los vectores de pago a tiempo $t$. Entonces,
\begin{alignat}{1}
  D_t[j, k]\ &=\ \sum_{\tau=1}^{t} v(s_{\tau})[j,k]\
	=\ \sum_{1\leq\tau \leq t, s_i^{\tau} = j} u_i(k, s_{-i}^{\tau}) - u_i(j, s_{-i}^{\tau})\
	=\ D_i^t(j, k) \,.
\end{alignat}
Para $x \notin \mathbb{R}^-$, $F(x) = x^-$ y $\lambda (x) = x - x^- = x^+$, obteniendo 
\begin{alignat}{1}
	\lambda(D_t)\ =\ (R_t^i(j, k))_{(j, k) \in L} \,.
\end{alignat}
Luego, usar una estrategia que cumpla
\begin{alignat}{1}
	q_{\lambda}(j) \sum_{k \in S_i} \lambda(j, k)\ =\  \sum_{k \in S_i} q_{\lambda}(k) \lambda(k, j)
\end{alignat}
cuando $\lambda(j, k) = [D_i^t(j, k)]^+ = R_t^i(j, k)$ es equivalente que la estrategia $p_{t+1}^i \in \Delta(S_i)$ cumpla con
\begin{alignat}{1}
	p_{t+1}^i (j) \sum_{k \in S_i} R_i^t(j, k)\ =\ \sum_{k \in S_i} R_i^t(k, j) p_{t+1}^i(k)
\end{alignat}
Aplicando el Teorema \ref{theo:blackwell} se tiene que al usar dicha estrategia, $D_t$ alcanza a $\mathbb{R}^-$ que es equivalente a que $R_i^t(j, k) \rightarrow 0$ para todo $j, k \in S_i$.
\end{proof}

\subsubsection{Procedimiento C: Regret incondicional}

El tercer procedimiento no conduce necesariamente a un equilibrio correlacionado. Sin embargo es considerado ``universalmente consistente'' (Definición \ref{def:proc-univ-consistente}). En este procedimiento, el pago promedio del jugador $i$, en el límite, no es peor a el pago si él hubiese jugado cualquier estrategia constante $k$, para todo $\tau \leq t$.

\begin{definition}
\label{def:proc-univ-consistente}
Un procedimiento adaptativo es \textbf{universalmente consistente} para el jugador $i$ si:
\begin{alignat}{1}
	\limsup_{t \rightarrow \infty } \left[ \max_{k \in S_i} \frac{1}{t} \sum_{\tau = 1}^{t} u_i(k, s_{-i}^{\tau}) - \frac{1}{t} \sum_{\tau = 1}^{t} u_i(s_{\tau}) \right]\ \leq\ 0\quad (a. s.)
\end{alignat}
\end{definition}
El procedimiento es definido a continuación. A tiempo $t$, definimos
\begin{alignat}{1}
D_i^t(k)\ &=\ \frac{1}{t} \sum_{\tau = 1}^{t} u_i(k, s_{-i}^{\tau}) - u_i(s_{\tau}) \,, \\
R_i^t(k)\ &=\ [D_i^t(k)]^+\ =\ \max(0, D_i^t(k)) \,.
\end{alignat}
Luego, la distribución de probabilidad a tiempo $t+1$, $p_{t+1}^i \in \Delta(S_i)$, es definida como sigue:
\begin{alignat}{1}
\label{eq:proc-C}
  p_{t+1}^i(k)\ =\ \frac{R_i^t(k)}{\sum_{k'\in S_i} R_i^t(k')}
\end{alignat}
si el denominador es positivo, y de forma arbitraria en caso contrario. Note que las probabilidades son elegidas de forma proporcional a $R_i^t(k)$ que será denominado \textit{regret} incondicional (en contraste al \textit{regret} condicional definido previamente).

\begin{theorem}
\label{theo:conv-proc-C}
El procedimiento adaptativo definido en \eqref{eq:proc-C} es universalmente consistente para el jugador $i$.
\end{theorem}

\begin{proof}
La prueba es similar a la del procedimiento anterior. Se definen $L$, $v$ y $\mathcal{C}$ del Teorema \ref{theo:blackwell} de la siguiente manera:
\begin{itemize}[noitemsep]
  \item $L = S_i$,
  \item $v = v(s_i, s_{-i}) \in \mathbb{R}^L$ dada por:
    $[v(s_i, s_{-i})](k)=u_i(k, s_{-i}) - u_i(s_i, s_{-i})$,
  \item $\mathcal{C} = \mathbb{R}^L_- = \{x \in \mathbb{R}^L : x_i \leq 0\ \forall i \in L \}$ (i.e.\ el ortante negativo).
\end{itemize}
Se demostrará que $\mathcal{C}$ es alcanzable por $i$. Al igual que antes, se tiene que:
\begin{alignat}{1}
  w_{\mathcal{C}}\ =\ 
    \begin{cases}
	  0 & \text{si } \lambda \in \mathbb{R}^L_+ \,, \\
      \infty & \text{en caso contrario.}
    \end{cases}
\end{alignat}
Por otra parte,
\begin{alignat}{1}
	\lambda \cdot v(q_{\lambda}, s_{-i})\ 
	&=\ \sum_{k \in L} \lambda(k) \cdot [v(q_{\lambda}, s_{-i})](k) \\
	&=\ \sum_{k \in S_i} \lambda(k) \cdot \sum_{j \in S_i} q_{\lambda}(j)[v(j, s_{-i})] (k) \\
	&=\ \sum_{k \in S_i} \lambda(k) \cdot \sum_{j \in S_i} q_{\lambda}(j) [u_i(k, s_{-i}) - u_i(j, s_{-i})] \\
	&=\ \sum_{\substack{k \in S_i \\ j \in S_i}} \lambda(k) q_{\lambda}(j) [u_i(k, s_{-i}) - u_i(j, s_{-i})] \\
	&=\ \sum_{\substack{k \in S_i \\ j \in S_i}} \lambda(k) q_{\lambda}(j) u_i(k, s_{-i}) - \sum_{\substack{k \in S_i \\ j \in S_i}} \lambda(k) q_{\lambda}(j) u_i(j, s_{-i}) \\
	&=\ \sum_{\substack{j \in S_i \\ k \in S_i}} u_i(j, s_{-i}) \lambda(j) q_{\lambda}(k)  - \sum_{\substack{j \in S_i \\ j \in S_i}} u_i(j, s_{-i}) \lambda(k) q_{\lambda}(j) \\
	&=\ \sum_{\substack{j \in S_i \\ k \in S_i}} u_i(j, s_{-i}) [\lambda(j) q_{\lambda}(k)  - \lambda(k) q_{\lambda}(j)] \\
	&=\ \sum_{j \in S_i} u_i(j, s_{-i}) \left[ \lambda(j) \sum_{k \in S_i} q_{\lambda}(k)  - q_{\lambda}(j) \sum_{k \in S_i} \lambda(k)  \right] \\
	&=\ \sum_{j \in S_i} u_i(j, s_{-i}) \left[ \lambda(j) - q_{\lambda}(j) \sum_{k \in S_i} \lambda(k) \right] \,.
\end{alignat}
La última igualdad porque $\sum_{k\in S_i} q_{\lambda}(k)=1$. Luego, si se define:
\begin{alignat}{1}
	\alpha(j)\ =\ \lambda(j) - q_{\lambda}(j) \sum_{k \in S_i} \lambda(k) \,,
\end{alignat}
obtenemos $\lambda \cdot v(q_{\lambda}, s_{-i}) = \sum_{j \in S_i} u_i(j, s_{-i})\alpha(j)$.
Note que si $q_{\lambda(j)} = \frac{\lambda(j)}{\sum_{k \in S_i} \lambda(k)}$, entonces $\alpha(j) = 0$ para todo $j \in S_i$ y se cumple la condición del Teorema \ref{theo:blackwell} en forma de igualdad. Además, para $D_t=\frac{1}{t} \sum_{\tau = 1}^{t} v(s_{\tau})$, tenemos
\begin{alignat}{1}
	D_t[k]\ =\ \sum_{\tau = 1}^{t} v(s_{\tau})[k]\ =\ \sum_{\tau \leq t }[u_i(k, s_{-i}^{\tau}) - u_i(s_{\tau})]\ =\  D_i^t(k) \,.
\end{alignat}
Luego $F(D_t) = D_t^-$ y $\lambda(D_t) = D_t^+ = (R_i^t(k))_{k \in S_i}$, obteniendo:
\begin{alignat}{1}
	q_{\lambda(D_t)}\ =\ \frac{[\lambda(D_t)](j)}{\sum_{k \in S_i}[\lambda(D_t)](k)}\ =\ \frac{R_i^t(j)}{\sum_{k \in S_i} R_i^t(k)}
\end{alignat}
Al elegir $p_{t+1}(j) = q_{\lambda(D_t)}(j) = \frac{R_i^t (j)}{\sum_{k \in S_i} R_i^t(k)}$, se obtiene que $D_t$ alcanza a $\mathbb{R^-}$, lo cual es equivalente a que $R_i^t(j) \rightarrow 0$ para todo $j \in S_i$.
\end{proof}
 % SUBSUMED BY forma-normal
%\input{capitulo-1/regret-matching.tex} % REORGANIZED
%\section{Equilibrio de Nash en el Juego Kunh Poker}

La descripción del juego se encuentra en la sección \ref{section:kuhn-poker}. Con respecto a la solución, se tiene que el jugador $2$ tiene una ganancia esperada de $\frac{1}{18}$ por mano, como se prueba en \cite{bib:kuhn-poker}, si ambos jugadores juegan de forma óptima (es decir, acorde a un equilibrio de Nash). El equilibrio de Nash se resume en la Tabla \ref{tab:estrategia-kuhn-poker}, donde los conjuntos de información fueron enumerados en un orden de búsqueda por profundidad (dfs).

\begin{table}[ht]
    \centering
    \begin{tabular}{c|r r}
        \hline
        I & \multicolumn{2}{|c}{Equilibrio de Nash}  \\ \hline
         $1$ & $1-\alpha$ & $\alpha$ \\
         $2, 3, 6, 10$ & $1$ & $0$ \\
         $4$ & $\frac{2}{3}$ & $\frac{1}{3}$ \\
         $5, 7, 12$ & $0$ & $1$ \\
         $8$ & $\frac{2}{3}$ & $\frac{1}{3}$ \\
         $9$ & $\frac{2}{3} - \alpha$ & $\alpha + \frac{1}{3}$ \\
        $11$ & $1 - 3 \alpha$ & $3 \alpha$ \\ \hline
    \end{tabular}
    \caption{Equilibrio de Nash para el juego de Kuhn Poker}
    \label{tab:estrategia-kuhn-poker}
\end{table}


El primer jugador tiene infinitas estrategias óptimas, las cuales pueden ser representadas por la elección de un parámetro $\alpha \in [ 0, \frac{1}{3} ]$. Una vez elegido este parámetro, el primer jugador en su primera jugada debe apostar con probabilidad $\alpha$ cuando su carta tenga el número $1$, apuestar con una probabilidad $3 \alpha$ cuando tenga el número $3$ y pasar siempre cuando tenga el número $2$. Si el primer jugador tiene un segundo turno, debe pasar siempre que tenga el número $1$, apostar cuando tiene el número $3$, y en el caso que tenga el número $2$ debe apostar con probabilidad $\alpha + \frac{1}{3}$.

El segundo jugador tiene una única estrategia mixta óptima: apostar siempre que tenga el número $3$. Cuando tenga el número $1$, pasar siempre que el primer jugador haya apostado y apostar con probabilidad $\frac{1}{3}$ en caso contrario. Cuando tenga el número $2$, debe pasar cuando el oponente haya pasado previamente y apostar con probabilidad $\frac{2}{3}$ en caso contrario. La figura \ref{fig:kunh-poker-estrategias} muestra el árbol con las distribuciones de probabilidad de las estrategias previamente descritas en cada uno de los nodos alcanzables en el juego.

\begin{figure}[ht]
\begin{center}
\caption{Equilibrio de Nash del juego de Kunh poker}
\label{fig:kunh-poker-estrategias}
\begin{tikzpicture}[
chance/.style={circle, draw=black, fill=gray, thick, minimum size = 6mm},
player1/.style={circle, draw=black, solid, thick, minimum size = 4mm},
player2/.style={circle, thick, minimum size=4mm},
terminal/.style={rectangle, draw=black, solid, fill=black, thick, minimum size=2mm},
level 1/.style={sibling distance=26mm, level distance=30mm},
level 2/.style={sibling distance=13mm, level distance=20mm},
level 3/.style={sibling distance=6.5mm, level distance=15mm},
]
\node[chance] [label=above:{Nodo de azar}] {} {
	child {node [player1] (P1) [label=above:{$(1, 2)$}]{}
		child { node [player2] [draw=red, fill=red] {} 
				child { node [terminal] [label=below:{$-1$}] {}
						edge from parent [dashed] node [left, draw=none] {\tiny{$1$}} {} }
				child { node [player1] (A) {} 
						child { node [terminal] [label=below:{$-1$}] {}
								edge from parent [dashed] {} }
						child { node [terminal] [label=below:{$-2$}] {} 
								edge from parent [solid, double] {} }
						edge from parent [solid, double] node [right, draw=none] {\tiny{$0$}}  {}
				}
				edge from parent [dashed] node [left, draw=none] {\tiny{$1-\alpha$}}  {}
		}
		child { node [player2] [draw=red] {} 
				child { node [terminal] [label=below:{$1$}] {}
						edge from parent [dashed] node [left, draw=none] {\tiny{$\frac{2}{3}$}} {} }
				child { node [terminal] [label=below:{$-2$}] {}
						edge from parent [solid, double] node [right, draw=none] {\tiny{$\frac{1}{3}$}} {} }
				edge from parent [solid, double] node [right, draw=none] {\tiny{$\alpha$}}{}
		}
	}
	child {node [player1] (P2) [label=above:{$(1, 3)$}]{}
		child { node [player2] [draw=blue, fill=blue] {} 
				child { node [terminal] [label=below:{$-1$}] {}
						edge from parent [dashed] node [left, draw=none] {\tiny{$0$}} {} }
				child { node [player1] (B) {} 
						child { node [terminal] [label=below:{$-1$}] {}
								edge from parent [dashed] node [left, draw=none] {\tiny{$1$}} {} }
						child { node [terminal] [label=below:{$-2$}] {}
								edge from parent [solid, double] node [right, draw=none] {\tiny{$0$}} {} }
						edge from parent [solid, double] node [right, draw=none] {\tiny{$1$}} {}
				}
				edge from parent [dashed] node [left, draw=none] {\tiny{$1-\alpha$}} {}
		}
		child { node [player2] [draw=blue] {} 
				child { node [terminal] [label=below:{$1$}] {}
						edge from parent [dashed] node [left, draw=none] {\tiny{$0$}} {} }
				child { node [terminal] [label=below:{$-2$}] {}
						edge from parent [solid, double] node [right, draw=none] {\tiny{$1$}} {} }
				edge from parent [solid, double] node [right, draw=none] {\tiny{$\alpha$}} {}
		}
	}
	child {node [player1] (P3) [label=135:{$(2, 1)$}]{}
		child { node [player2] [draw=green, fill=green] {} 
				child { node [terminal] [label=below:{$1$}] {}
						edge from parent [dashed] node [left, draw=none] {\tiny{$\frac{2}{3}$}} {} }
				child { node [player1] (C) {} 
						child { node [terminal] [label=below:{$-1$}] {}
								edge from parent [dashed] node [left, draw=none] {\tiny{$\frac{2}{3}-\alpha$}} {} }
						child { node [terminal] [label=below:{$2$}] {} 
                        		edge from parent [solid, double] node [right, draw=none] {\tiny{$\alpha+\frac{1}{3}$}} {} }
						edge from parent [solid, double] node [right, draw=none] {\tiny{$\frac{1}{3}$}} {}
				}
				edge from parent [dashed] node [left, draw=none] {\tiny{$1$}} {}
		}
		child { node [player2] [draw=green] {} 
				child { node [terminal] [label=below:{$1$}] {}
						edge from parent [dashed] {} }
				child { node [terminal] [label=below:{$2$}] {}
						edge from parent [solid, double] {} }
				edge from parent [solid, double] node [right, draw=none] {\tiny{$0$}} {}
		}
	}
	child {node [player1] (P4) [label=45:{$(2, 3)$}]{}
		child { node [player2] [draw=blue, fill=blue]{} 
				child { node [terminal] [label=below:{$-1$}] {}
						edge from parent [dashed] node [left, draw=none] {\tiny{$0$}} {} }
				child { node [player1] (D) {} 
						child { node [terminal] [label=below:{$-1$}] {}
								edge from parent [dashed] node [left, draw=none] {\tiny{$\frac{2}{3}-\alpha$}} {} {} }
						child { node [terminal] [label=below:{$-2$}] {}
								edge from parent [solid, double] {} node [right, draw=none] {\tiny{$\alpha+\frac{1}{3}$}} {} }
						edge from parent [solid, double] node [right, draw=none] {\tiny{$1$}} {}
				}
				edge from parent [dashed] node [left, draw=none] {\tiny{$1$}} {}
		}
		child { node [player2] [draw=blue] {} 
				child { node [terminal] [label=below:{$1$}] {}
						edge from parent [dashed] {} }
				child { node [terminal] [label=below:{$-2$}] {}
						edge from parent [solid, double] {} }
				edge from parent [solid, double] node [right, draw=none] {\tiny{$0$}} {}
		}
	}
	child {node [player1] (P5) [label=above:{$(3, 1)$}]{}
		child { node [player2] [draw=green, fill=green] {} 
				child { node [terminal] [label=below:{$1$}] {}
						edge from parent [dashed] node [left, draw=none] {\tiny{$\frac{2}{3}$}} {} }
				child { node [player1] (E) {} 
						child { node [terminal] [label=below:{$-1$}] {}
								edge from parent [dashed] node [left, draw=none] {\tiny{$0$}} {} }
						child { node [terminal] [label=below:{$2$}] {}
								edge from parent [solid, double] node [right,draw=none] {\tiny{$1$}} {} }
						edge from parent [solid, double] node [right, draw=none] {\tiny{$\frac{1}{3}$}} {}
				}
				edge from parent [dashed] node [left, draw=none] {\tiny{$1-3 \alpha$}} {}
		}
		child { node [player2] [draw=green] {} 
				child { node [terminal] [label=below:{$1$}] {}
						edge from parent [dashed] node [left, draw=none] {\tiny{$1$}} {} }
				child { node [terminal] [label=below:{$2$}] {}
						edge from parent [solid, double] node [right, draw=none] {\tiny{$0$}} {} }
				edge from parent [solid, double] node [right, draw=none] {\tiny{$3 \alpha$}} {}
		}
	}
	child {node [player1] (P6) [label=above:{$(3, 2)$}]{}
		child { node [player2]  [draw=red, fill=red] {} 
				child { node [terminal] [label=below:{$1$}] {}
						edge from parent [dashed] node [left, draw=none] {\tiny{$1$}} {} }
				child { node [player1] (F) {} 
						child { node [terminal] [label=below:{$-1$}] {}
								edge from parent [dashed] {}}
						child { node [terminal] [label=below:{$2$}] {}
								edge from parent [solid, double] {} }
						edge from parent [solid, double] node [right, draw=none] {\tiny{$0$}} {}
				}
				edge from parent [solid, double] node [left, draw=none] {\tiny{$1-3\alpha$}} {}
		}
		child { node [player2]  [draw=red] {} 
				child { node [terminal] [label=below:{$1$}] {}
						edge from parent [dashed] node [left, draw=none] {\tiny{$\frac{2}{3}$}} {} }
				child { node [terminal] [label=below:{$2$}] {}
						edge from parent [solid, double] node [right, draw=none] {\tiny{$\frac{1}{3}$}} {} }
				edge from parent [solid, double] node [right, draw=none] {\tiny{$3\alpha$}} {}
		}
	}
};
\end{tikzpicture}
\end{center}
\end{figure}
 % REORGANIZED: movido adentro de forma-extensa.tex
%\section{Counterfactual Regret Minimization}

El objetivo de esta sección es presentar un algoritmo que permita encontrar un equilibrio de Nash en un juego en forma extensiva no determinista con información incompleta. Aunque todo juego en forma extensiva puede ser representado en forma normal, esto no es de mucho interés, pues la forma normal puede tener un tamaño exponencialmente más grande al tamaño del árbol. Se verá como el concepto de \textit{regret minimization} puede ser extendido a juegos secuenciales, sin necesidad de la forma normal explícita. Los conceptos, procedimientos y teoremas mostrados en esta sección, son presentados en \cite{bib:cfr}.

\subsection{Regret Minimization}

La primera definición clave es el \textit{regret}. Para esto, es necesario considerar jugar repetidamente un juego en forma extensiva. Sea $\sigma_i^t$ la estrategia usada por el jugador $i$ a tiempo $t$. La Definición \ref{def:average-overall-regret}, presenta el concepto de \textit{average overall regret}.

\begin{definition}
\label{def:average-overall-regret}
El \textbf{average overall regret} del jugador $i$ a tiempo $T$ es:
\begin{alignat}{1}
R_i^T = \frac{1}{T} \max_{\sigma^* in B_i} \sum_{t = 1}^T [u_i(\sigma_i^*, \sigma_{-i}^t) - u_i(\sigma^t)]
\end{alignat}
\end{definition}

Se denotará con $\bar{\sigma}_i^{T}$ la estrategia promedio del jugador $i$ del tiempo $1$ al tiempo $T$. En particular, para cada conjunto de información $I \in \mathcal{I}_i$ y para cada acción $a \in A(I)$ se define:
\begin{alignat}{1}
\bar{\sigma}_i^{T}(I)(a) = \frac{\sum_{t = 1}^T \pi^{\sigma^t_i}(I)\sigma^t_i(I)(a)}{\sum_{t = 1}^T \pi^{\sigma_i^t}(I)}
\end{alignat}

Esta estrategia es el promedio ponderado de las probabilidades $\sigma^t(I)(a)$ con respecto a que tan probable es alcanzar $I$ dado $\sigma_i^t$.

La relación entre el \textit{average overall regret} y el concepto de solución se muestra en el Teorema \ref{theo:regret-nash} \cite{bib:cfr}.
\begin{theorem}
\label{theo:regret-nash}
En un juego de $2$ jugadores de suma cero si el average overall regret a tiempo $T$ es menor que $\varepsilon$ entonces $\sigma^{-T}$ es un $2\varepsilon$-equilibrio de Nash
\end{theorem}
\begin{proof}

Se probará que la probabilidad de alcanzar $z$ bajo $\bar{\sigma}_i^T$ viene dada por el promedio de alcanzar $z$ en cada estrategia. Sean $h_1 \sqsubset h_2 \sqsubset h_3 \sqsubset ... \sqsubset h_m \sqsubset z$ todos los prefijos de $z$ correspondientes al jugador $i$, es decir $P(h_k) = i\ \forall k : 1 \leq k \leq m$ y sean $a_1, a_2, ..., a_m$ las acciones correspondientes en $z$ en cada historia respectiva. Luego:
\begin{alignat}{2}
\pi^{\bar{\sigma}_i^T}(z)\ &=\ \prod_{k = 1}^m \bar{\sigma}_i^T(I(h_k))(a_k) \\	&=\ \prod_{k = 1}^m \frac{\sum_{t = 1}^{T}  \pi^{\sigma_i^t}(I(h_k))\sigma^t_i(I(h_k))(a)}{\sum_{t = 1}^T \pi^{\sigma_i^t}(I(h_k))}
\end{alignat}

Por otra parte, note que $\pi^{\sigma_i^t}(I)\sigma_i^t(I(h_k))(a_k) = \pi^{\sigma_i^t}(I(h_{k+1}))$. Entonces:
\begin{alignat}{1}
	\pi^{\bar{\sigma}_i^T}(z)\ &=\ \frac{\sum_{t = 1}^T\pi^{\sigma_i^t}(I_m) \sigma_i^t(I_m)(a_m)}{\sum_{t = 1}^T \pi^{\sigma_i^t}(I_1)} \\
	&=\ \frac{\sum_{t = 1}^T \pi^{\sigma_i^t}(z)}{\sum_{t = 1}^T 1} \\
	&=\ \frac{1}{T} \sum_{t = 1}^T \pi^{\sigma_i^t}(z)
\end{alignat}

Además, se tiene que, para cualquier jugador $i$ y cualquier estrategia de $\sigma_i$:
\begin{alignat}{2}
\frac{1}{T} \sum_{t = 1}^T u_i(\sigma'_i, \sigma_{-i}^t)\ &=\ \frac{1}{T} \sum_{t = 1}^T \left( \sum_{z \in Z} \pi^{\sigma'_i}(z) \pi^{\sigma_{-i}^t}(z) \pi^c(z) \right) \\
	&=\ \sum_{z \in Z} u_i(z) \pi^{\sigma'_i}(z) \pi^c(z) \left( \frac{1}{T} \sum_{t = 1}^T \pi^{\sigma_{-i}^t}(z) \right) \\
	&=\ \sum_{z \in Z} u_i(z) \pi^{\sigma'_i}(z) \pi^{\bar{\sigma}_i^T}(z) \pi^c(z) \\
	&=\ u_i(\sigma'_i, \bar{\sigma}_{-i}^T)
\end{alignat}

Por otra parte, como $R_2^T \leq \varepsilon$, para todo $\sigma'_2 \in B_2$ se tiene que:
\begin{alignat}{2}
	& \frac{1}{T} \sum_{t = 1}^T[u_2(\sigma_1^t, \sigma'_2) - u_2(\sigma^t)]\  \leq\  \varepsilon\\
	\Rightarrow\ & \frac{1}{T} \sum_{t = 1}^T u_2(\sigma^t) + \varepsilon\  \geq\  \frac{1}{T} \sum_{t = 1}^T u_2(\sigma_1^t, \sigma'_2) \\
	\Rightarrow\ & \frac{1}{T} \sum_{t = 1}^T u_2(\sigma^t) + \varepsilon\ \geq\ u_2(\bar{\sigma}_1^T, \sigma'_2)
\end{alignat}

Luego, como se cumple para cualquier $\sigma'_2$, se cumple para $\sigma_2^t$ para $t = 1, 2, ..., T$, obteniendo:
\begin{alignat}{2}
\frac{1}{T} \sum_{t = 1}^T u_2(\sigma^t) + \varepsilon\ & \geq\ \frac{1}{T} \sum_{t = 1}^T u_2(\bar{\sigma}_1^T, \sigma_2^t) \\
&=\ \frac{1}{T} \sum_{t = 1}^T \sum_{z \in Z} u_2(z) \pi^{\bar{\sigma}_1^T}\pi^{\sigma_2^t}(z) \pi^c(z) \\
&=\  \sum_{z \in Z} u_2(z) \pi^{\bar{\sigma}_1^T} \pi^c(z) \left( \frac{1}{T} \sum_{t = 1}^T \pi^{\sigma_2^t}(z) \right)\\
&=\  \sum_{z \in Z} u_2(z) \pi^{\bar{\sigma}_1^T} \pi^{\bar{\sigma}_2^T} \pi^c(z) \\
&=\ u_2(\bar{\sigma}^T)
\end{alignat}

Como $\Gamma$ es un juego de suma cero, se tiene que $u_2 = -u_1(\sigma)$ para toda estrategia $\sigma$, luego:
\begin{alignat}{2}
	& \frac{1}{T} \sum_{t = 1}^T u_2(\sigma^t) + \varepsilon\ \geq\  u_2(\bar{\sigma}^T) \\
	\Rightarrow\ & \frac{1}{T} \sum_{t = 1}^T -u_1(\sigma^t) + \varepsilon\ \geq\  -u_1(\bar{\sigma}^T) \\
	\Rightarrow\ &  u_1(\bar{\sigma}^T)\ + \varepsilon\ \geq\   \frac{1}{T} \sum_{t = 1}^T u_1(\sigma^t)  \\
	\Rightarrow\ &  u_1(\bar{\sigma}^T)\ + 2\varepsilon\ \geq\   \frac{1}{T} \sum_{t = 1}^T u_1(\sigma^t) + \varepsilon
\end{alignat}

Por otra parte, como $R_i^t \leq \varepsilon$ se tiene que, para cualquier $\sigma'_1 \in B_1$:
\begin{alignat}{1}
	\frac{1}{T} \sum_{t = 1}^T u_1(\sigma^t) + \varepsilon\ \geq\ \frac{1}{T} \sum_{t = 1} u_1(\sigma'_1, \sigma_2^t)\ =\ u_1(\sigma'_1, \bar{\sigma}_2^T)
\end{alignat}

Luego, se obtiene que:
\begin{alignat}{1}
	& u_1(\bar{\sigma}^T)\ + 2\varepsilon\ \geq\   \frac{1}{T} \sum_{t = 1}^T u_1(\sigma^t) + \varepsilon\ \geq\ u_1(\sigma'_1, \bar{\sigma}_2^T)\\
	\Rightarrow\ & u_1(\bar{\sigma}^T)\ + 2\varepsilon\ \geq\ u_1(\sigma'_1, \bar{\sigma}_2^T)
\end{alignat}

Análogamente, se demuestra que:
\begin{alignat}{1}
	u_2(\bar{\sigma}^T)\ + 2\varepsilon\ \geq\ u_2( \bar{\sigma}_1^T, \sigma'_2)
\end{alignat}

Concluyendo que $\bar{\sigma}^T$ es un $2\varepsilon$-equilibrio correlacionado.
\end{proof}

Como consecuencia del teorema anterior se obtiene que un algoritmo que lleve el \textit{average overall regret} a cero, conducirá a un equilibrio de Nash. La idea fundamental del enfoque presentado a continuación, propuesto en \cite{bib:cfr}, consiste en descomponer el \textit{average overall regret} en un conjunto de términos aditivos de \textit{regret} que puedan ser minimizados independientemente. En particular es necesario introducir un par de nuevo concepto, la utilidad contrafactual (Definición \ref{def:utilidad-contrafactual}) y el \textit{regret} contrafactual inmediato (Definición \ref{def:regret-inmediato}).

\begin{definition}
\label{def:utilidad-contrafactual}
La \textbf{utilidad contrafactual} es la ganancia esperada dado que el conjunto $I$ es alcanzado y todos los jugadores juegan con la estrategia $\sigma$ con excepción del jugador $i$ que juega para alcanzar $I$. Formalmente, si $\pi^{\sigma}(h, h')$ es la probabilidad de ir de la historia $h$ a la historia $h'$, entonces:
\begin{alignat}{1}
u_i(\sigma, I) = \frac{\sum_{h \in H, z \in Z} \pi^{\sigma_{-i}(h)1} \pi^{\sigma}(h, z) u_i(z)}{\pi^{\sigma_{-i}}(I)}
\end{alignat}
\end{definition}

Para toda acción $a \in A(I)$, se define $\sigma|_{I \rightarrow a}$ como el perfil estratégico idéntico a $\sigma$ excepto que el jugador $i$ siempre elige $a$ en el conjunto de información $I$.

\begin{definition}
\label{def:regret-inmediato}
El \textbf{regret contrafactual inmediato} es:
\begin{alignat}{1}
R_{i, \text{imm}}^T (I) = \frac{1}{T} \max_{a \in A(I)} \sum_{t = 1}^T \pi^{\sigma_{-i}^t}(I)[u_i(\sigma^t|_{I \rightarrow a}, I) - u_i(\sigma^t, I)] ]
\end{alignat}
\end{definition}

Intuitivamente, el \textit{regret} contrafactual inmediato es el arrepentimiento del jugador $i$ en su decisión en el conjunto de información $I$, en términos de la utilidad contrafactual, con un término de ponderación adicional para la probabilidad contrafactual que $I$ alcanzaría en esa ronda si el jugador hubiera intentado hacer eso. Usualmente, es de mayor interés el \textit{regret} cuando es positivo, por lo que se define $R_{i, \text{imm}}^{T, +} (I) = max(R^T_{i, \text{imm}} (I), 0)$. Luego, se tiene el siguiente resultado:

\begin{theorem}
\begin{alignat}{1}
R_i^T \leq \sum_{I \in \mathcal{I}_i} R_{i, \text{imm}}^{T, +}(I)
\end{alignat}
\end{theorem}

Debido a que minimizar cada \textit{regret} contrafactual inmediato minimiza el \textit{average overall regret} arrepentimiento general promedio, es posible enfocarse en minimizar los primeros para obtener un equilibrio de Nash.

\subsection{Counterfactual Regret Minimization}

Antes de mostrar el algoritmo principal, denominado \textit{Counterfactual Regret Minimization} para los juegos en forma extensiva, es necesario introducir el algoritmo de \textit{Regret Matching} general. Este algoritmo puede ser descrito en un dominio donde hay un conjunto fijo de acciones $A$, una función $u^t : A \rightarrow \mathbb{R}$ y en cada ronda una distribución de probabilidad $p^t$ es elegida.

\begin{definition}
\label{def:regret}
El \textit{regret} de no haber elegido la acción $a \in A$ hasta tiempo $T$, se define como:
\begin{alignat}{1}
R_i^T(a) = \frac{1}{T} \sum_{t = 1}^T \left[u_i(a) - \sum_{a' \in A}p^t(a)u^t(a)\right]
\end{alignat}
\end{definition}

Se define $R^{t, +}(a) = \max(R^t(a), 0)$. Luego la distribución $p^{t+1}$ es elegida de la siguiente manera:
\begin{alignat}{1}
p^t(a) = 
\begin{cases}
\frac{R_i^{t, +}}{\sum_{a' \in A} R^{t, +}(a)}\ & \text{si }\ \sum_{a' \in A} R^{t, +}(a) > 0\\
\frac{1}{A}\ & \text{en otro caso}
\end{cases}
\end{alignat}

\begin{theorem}
Si $|u| = \max_{t \in \{1, 2, ... T\}} \max_{a, a' \in A}(u^t(a) - u^t(a'))$ entonces el regret del algoritmo de regret matching está acotado por:
\begin{alignat}{1}
max_{a \in A}R^t(a) \leq \frac{|u| \sqrt{|A|}}{\sqrt T}
\end{alignat}
\end{theorem}

Luego, el algoritmo de \textit{Counterfactual Regret Minimization} es una aplicación del algoritmo \textit{Regret Minimization} de forma independiente a cada conjunto de información. En particular, se mantiene, para cada $I \in \mathcal{I}_i$ y para todo $a \in A(I)$:
\begin{alignat}{1}
R_i^T(I, a) = \frac{1}{T} \sum_{t = 1}^T \pi^{\sigma^t_{-i}}(I)[u_i(\sigma^t|_{I \rightarrow a}, I) - u_i(\sigma^t, I)]
\end{alignat}

Se define $R_i^{T, +}(I, a) = \max(R_i^T(I, a), 0)$, luego a tiempo $T+1$ la estrategia elegida es:
\begin{alignat}{1}
\sigma_i^{T+1}(I)(a) =
\begin{cases}
\frac{R_i^{T, +}(I, a)}{\sum_{a' \in A(I) R_i^{T, +}(I, a')}}\ & \text{si } \sum_{a' \in A(I) R_i^{T, +}(I, a')} > 0 \\
\frac{1}{|A(I)|}\ & \text{en otro caso} 
\end{cases}
\end{alignat}

Este algoritmo consiste en seleccionar las acciones de forma proporcional a la cantidad del \textit{regret} contrafactual positivo de no haber elegido esa acción. Si ninguna de estas cantidades es positiva, entonces la acción se elige con una distribución uniforme. Luego, como cota de convergencia, se tiene el siguiente teorema:
\begin{theorem}
Si el jugador $i$ selecciona las acciones de acuerdo al procedimiento anterior, entonces $R^T_{i, \text{imm}}(I) \leq \Delta_{u, i} \frac{\sqrt{|A_i|}}{\sqrt{T}}$ y por lo tanto $R_i^T \leq \Delta_{u, i} |\mathcal{I}_i| \frac{|A_i|}{\sqrt T}$, donde $|A_i| = max_{h : P(h) = 1}{|A(h)|}$
\end{theorem} % REORGANIZED: moved to REORGANIZED as new chapter
%\chapter{Regret Matching}
%\section{Regret Matching y Equilibrio de Nash}
En el capítulo anterior se describieron procedimientos universalmente consistente, algunos de los cuales permiten obtener equilibrios correlacionados. Sin embargo, éstos no garantizan obtener un equilibrio de Nash, surgiendo la siguiente interrogante: ?`Bajo que condiciones se puede garantizar que un procedimiento universalmente consistente conduce a un equilibrio de Nash? El Teorema \ref{UC-EN} responde esta pregunta.

\begin{theorem}
\label{UC-EN}
Sea $\Gamma$ un juego de dos jugadores de suma cero y sea $(s^t)_{t=1,2,..., T}$ una secuencia de juegos de $\Gamma$, tales que, para todo $s_i \in S_i$, para todo $i \in {1, 2}$:
\begin{alignat}{1}
\frac{1}{T}\sum_{t = 1}^{T}u_i(s_i, s_{-i}^t) - \frac{1}{T} \sum_{t = 1}^T u_i(s^t) \leq \varepsilon
\end{alignat}
para algún $\varepsilon > 0$. Sea $\bar{\sigma}^T = (\bar{\sigma_1}^T, \bar{\sigma_2}^T)$, donde:
\begin{alignat}{1}
\bar{\sigma}_i^T(s_i) = \frac{ |\{ 1 \leq T : s_i^t = s_i\}|}{T} = \frac{\#(s_i)}{T}
\end{alignat}
es decir, $\bar{\sigma}^T$, es la distribución empírica de probabilidad. Entonces $\bar{\sigma}^T$ es un $2\varepsilon$-equilibrio de Nash.
\end{theorem}

\begin{proof}
Por hipótesis del teorema, se tiene que:
\begin{alignat}{1}
\frac{1}{T} \sum_{t = 1}^T u_i(s_i, s_{-i}^t) - \frac{1}{T} \sum_{t = 1}^T u_i(s^t) \leq \varepsilon
\end{alignat}

Reordenado la sumatoria del primer término y utilizando la definición de $\bar\sigma$, se obtiene:
\begin{alignat}{2}
& \frac{1}{T} \sum_{s_{-i} \in S_{-i}} \#(s_{-i})u_i(s_i, s_{-i}) - \frac{1}{T} \sum_{t = 1}^Tu_i(s^t)  & \leq \varepsilon \\
\Rightarrow & \sum_{s_{-i} \in S_{-i}} \bar{\sigma}_{-i}^T(s_{-i})u_i(s_i, s_{-i}) - \frac{1}{T} \sum_{t = 1}^T u_i(s^t) & \leq \varepsilon
\end{alignat}

Sea $\sigma_i \in \Delta(S_i)$ cualquier estrategia del jugador $i$, luego

\begin{alignat}{4}
& & \sum_{s_i \in S_i} \sigma_i(s_i) \left[ \sum_{s_{-i} \in S_{-i}} \bar{\sigma}_{-i}^T(s_{-i})u_i(s_i, s_{-i}) - \frac{1}{T} \sum_{t = 1}^T u_i(s^t) \right] & \leq & \sum_{s_i \in S_i} \sigma_i(s_i) \varepsilon \\
& \Rightarrow & \sum_{s_i \in S_i} \sum_{s_{-i} \in S_{-i}} \sigma_i(s_i)\bar{\sigma}_{-i}^T(s_{-i}) u_i(s_i, s_{-i}) - \sum_{s_i \in S_i} \sigma_i(s_i)u_i(s^t) & \leq & \varepsilon \\
& \Rightarrow & u_i(\sigma_i, \bar{\sigma}_{-i}^T) - \frac{1}{T} \sum_{t = 1}^T u_i(s^t) & \leq & \varepsilon
\end{alignat}

En particular, se tiene que, para estrategias cualesquiera $\sigma_1 \in \Delta(S_1)$ y $\sigma_2 \in \Delta(S_2)$
\begin{alignat}{1}
\label{eq:star1}
u_1(\sigma_1, \bar{\sigma}_2^T) - \frac{1}{T} \sum_{t=1}^T u_1(s^t) \leq \varepsilon \\
u_2(\bar{\sigma}_1^T, \sigma_2) - \frac{1}{T} \sum_{t=1}^T u_2(s^t) \leq \varepsilon
\end{alignat}

Además, como $\Gamma$ es un juego de suma cero, se tiene que $u_2(\bar{\sigma}_1^T, \sigma_2) = -u_1(\bar{\sigma}_1^T, \sigma_2)$ y $u_2(s^t) = -u_1(s^t)$, luego:
\begin{alignat}{1}
u_2(\bar{\sigma}_1^T, \sigma_2) - \frac{1}{T} \sum_{t=1}^T u_2(s^t) = -u_1(\bar{\sigma}_1^T, \sigma_2) - \frac{1}{T} \sum_{t=1}^T -u_1(s^t) \leq \varepsilon
\end{alignat}

En particular, si $\sigma_2 = \bar{\sigma_2}^T$ entonces:
\begin{alignat}{1}
\label{eq:star2}
-u_1(\bar{\sigma}_1^T, \bar{\sigma_2}^T) + \frac{1}{T} \sum_{t=1}^T u_1(s^t) \leq \varepsilon
\end{alignat}

Al sumar las desigualdades \ref{eq:star1} y \ref{eq:star2} se obtiene que:
\begin{alignat}{2}
& u_1(\sigma_1, \bar{\sigma}_2^T) - u_1(\bar{\sigma}_1^T, \bar{\sigma}_2^T) \leq 2\varepsilon \\
\Rightarrow & u_1(\bar{\sigma}^T) + 2\varepsilon \geq u_1(\sigma_1, \bar{\sigma}_2^T)
\end{alignat}

Análogamente se tiene que $u_2(\bar{\sigma}^T) + 2\varepsilon \geq u_2(\bar{\sigma_1}^T, \sigma_2)$, con lo que se concluye que $\bar{\sigma}^T$ es un $2\varepsilon$-equilibrio de Nash.
\end{proof}

Se obtiene entonces que, en juegos de dos jugadores de suma cero, si un procedimiento es universalmente consistente, su distribución empírica llevará a un equilibrio de Nash. Los procedimientos propuestos en la sección anterior son universalmente consistentes, por lo que se pueden utilizar para encontrar un equilibrio de Nash en un juego particular. Estos algoritmos fueron implementados y probados en diferentes juegos de suma cero para encontrar una aproximación a un equilibrio de Nash en cada uno de ellos. % REORGANIZED: copied into regret-matching.tex
%\input{capitulo-2/evaluacion-explotabilidad} % REORGANIZED: moved to REORGANIZED as new chapter
%\section{Descripción de los Juegos}
\label{section:descripcion-juegos-fn}

Los algoritmos fueron probados en cuatro juegos diferentes en forma normal: \textit{matching pennies}, piedra, papel y tijeras, ficha vs dominó y coronel blotto. Estos juegos son de suma cero para dos persona, por lo tanto es suficiente con determinar el pago para del primer jugador para que el juego esté bien definido. Por otra parte, cualquier juego con estas características puede modelarse como un problema de programación lineal \cite[pp. 228-233]{bib:pl-chvatal} y resolverse mediante procedimientos destinados para esto.

Cada uno de los juegos es descrito mediante sus reglas. Además, si el juego tiene un tamaño fijo y es lo suficientemente pequeños, se muestra su matriz de pagos y el problema de programación lineal asociado con una solución.

\subsection{Matching Pennies}

En este juego cada jugador tiene una moneda y selecciona cara o sello de forma secreta. Si las elecciones son iguales gana el jugador $1$, en caso contrario gana el jugador $2$. La matriz de pagos de este juego se muestra en la Tabla \ref{table:pagos-matching-pennies}.

\begin{table}[ht]
\begin{center}
\caption[Tabla de pagos del juego matching pennies]{Tabla de pagos del juego \textit{matching pennies}}
\label{table:pagos-matching-pennies}
\begin{tabular}{ c | c | c |}
 & cara & sello  \\ \hline
 cara  &  1 & -1 \\ \hline
 sello & -1 &  1 \\ \hline
\end{tabular}
\end{center}
\end{table}

El problema de programación lineal asociado es el siguiente
\begin{equation}
\begin{array}{r r r r r r r r}
\max  & z &  & & & & &\\
\text{s.a.}  
&   &   & x_1  & + & x_2 & = & 1 \\
& z & - & x_1  & + & x_2 & \leq & 0 \\
& z & + & x_1  & - & x_2 & \leq & 0 \\
&   &   & x_1, &   & x_2 & \geq & 0
\end{array}
\end{equation}
Cuya solución primal viene dada por $(z^*, x_1^*, x_2^*) = (0, \frac{1}{2}, \frac{1}{2})$ y su solución dual por $(w^*, y^*_1, y^*_2) = (0, \frac{1}{2}, \frac{1}{2})$. Luego el equilibrio de Nash se obtiene cuando ambos jugadores eligen cara o sello con probabilidad $\frac{1}{2}$ y el valor del juego es igual a $0$.

\subsection{Piedra, Papel o Tijeras}

Este juego es descrito en la sección \ref{section:forma-normal} y su matriz de pago se muestra en la Tabla \ref{table:pago-RPS}. El problema de programación lineal asociado es el siguiente:
\begin{equation}
\begin{array}{r r r r r r r r r r}
\max  & z &  & & & & &\\
\text{s.a.}  
&   &   & x_1  & + & x_2  & + & x_3 & =    & 1 \\
& z &   &      & + & x_2  & - & x_3 & \leq & 0 \\
& z & - & x_1  &   &      & + & x_3 & \leq & 0 \\
& z & + & x_1  & - & x_2  &   &     & \leq & 0 \\
&   &   & x_1, &   & x_2, &   & x_3 & \geq & 0
\end{array}
\end{equation}
La solución primal y dual de este problema son $(z^*, x_1^*, x_2^*, x_3^*) = (0, \frac{1}{3}, \frac{1}{3}, \frac{1}{3})$ y $(w^*, y_1^*, y_2^*, y_3^*) = (0, \frac{1}{3}, \frac{1}{3}, \frac{1}{3})$, respectivamente. Luego, el equilibrio de Nash se obtiene cuando ambos jugadores eligen cada una de las acciones con probabilidad igual a $\frac{1}{3}$.

\subsection{Ficha vs Dominó}

En este juego cada jugador tiene un tablero de tamaño $2\times 3$. El primer jugador tiene una ficha de dominó (que ocupa dos casillas con un lado en común) que puede colocar de $7$ formas diferentes, cada forma es mostrada en la figura \ref{fig:posiciones-domino}, con su respectiva etiqueta. El segundo jugador posee una ficha que ocupa una sóla casilla de su tablero y la ubica en una de las $6$ casillas, las cuales se numeran en la Figura \ref{fig:posiciones}. Luego se superponen los tableros y si la ficha es cubierta por el dominó entonces el segundo jugador gana, en caso contrario gana el primer jugador \cite[p. 237]{bib:pl-chvatal}.

\begin{figure}[hbt]
\caption{Posibles posiciones de la ficha de dominó}
\label{fig:posiciones-domino}
\centering
\includegraphics[width=1\textwidth]{figuras/posiciones-domino.png}
\end{figure}

\begin{figure}[hbt]
\caption{Posibles posiciones de la ficha del segundo jugador}
\label{fig:posiciones}
\centering
\includegraphics[width=0.4\textwidth]{figuras/posiciones.png}
\end{figure}

\begin{table}[hbt]
\begin{center}
\caption{Matriz de pagos del juego Ficha vs Dominó}
\label{table:pagos-domino}
\begin{tabular}{ c | c | c | c | c | c | c |}
  &  1 &  2 &  3 &  4 &  5 &  6 \\ \hline
A & -1 & -1 &  1 &  1 &  1 &  1 \\ \hline
B &  1 &  1 &  1 & -1 & -1 &  1 \\ \hline
C &  1 & -1 & -1 &  1 &  1 &  1 \\ \hline
D &  1 &  1 &  1 &  1 & -1 & -1 \\ \hline
E & -1 &  1 &  1 & -1 &  1 &  1 \\ \hline
F &  1 & -1 &  1 &  1 & -1 &  1 \\ \hline
G &  1 &  1 & -1 &  1 &  1 & -1 \\ \hline
\end{tabular}
\end{center}
\end{table}

El problema de programación lineal asociado es:
\begin{equation}
\begin{array}{r r r r r r r r r r r r r r r r r r}
max        &z& &   & &   & &   & &   & &   & &   & &   &      & \\
\text{s. a}
	& & &x_1&+&x_2&+&x_3&+&x_4&+&x_5&+&x_6&+&x_7& =    &1 \\
    &z&+&x_1&-&x_2&-&x_3&-&x_4&+&x_5&-&x_6&-&x_7& \leq &0 \\
    &z&+&x_1&-&x_2&+&x_3&-&x_4&-&x_5&+&x_6&-&x_7& \leq &0 \\
    &z&-&x_1&-&x_2&+&x_3&-&x_4&-&x_5&-&x_6&+&x_7& \leq &0 \\
    &z&-&x_1&+&x_2&-&x_3&-&x_4&+&x_5&-&x_6&-&x_7& \leq &0 \\
    &z&-&x_1&+&x_2&-&x_3&+&x_4&-&x_5&+&x_6&-&x_7& \leq &0 \\
    &z&-&x_1&-&x_2&-&x_3&+&x_4&-&x_5&-&x_6&+&x_7& \leq &0 \\
    & & &x_1,&&x_2,&&x_3,&&x_4,&&x_5,&&x_6,&&x_7,&\geq &0 \\
\end{array}
\end{equation}

Este problema no tiene solución única (lo que implica que el juego no tiene un equilibrio de Nash único), una solución viene dada por $(z^*, x^*_1, x^*_2, x^*_4, x^*_5, x^*_6, x^*_6, x^*_7) = (\frac{1}{3}, \frac{1}{3}, \frac{1}{3}, 0, 0, 0, 0, \frac{1}{3})$ y $(w^*, y^*_1, y^*_2, y^*_3,  y^*_4, y^*_5, y^*_6) = (\frac{1}{3}, \frac{1}{3}, 0, \frac{1}{3}, 0, \frac{1}{3}, 0)$, esta solución corresponde a la estrategia en la que el jugador $1$ elige las posiciones A, B y G con probabilidad $\frac{1}{3}$ cada una, y el jugador $2$ elige las posiciones $1$, $3$, y $5$ con probabilidad $\frac{1}{3}$ cada una.

\subsection{Coronel Blotto}

En este juego cada uno de los jugadores tiene $S$ soldados en total que debe ubicar en $N$ campos de batallas. Cada soldado puede ser asignado a un único campo, pero cualquier número de soldados puede ser colocado en cualquier campo, incluyendo el cero. Un jugador obtiene un campo de batalla si asigna más soldados que su oponente en ese campo de batalla. El juego es ganado por el jugador que obtenga un mayor número de campos y su pago es igual a la diferencia entre el número de campos obtenidos por cada uno de los jugadores \cite{bib:blotto-game}.

Formalmente el juego puede ser descrito de la siguiente manera. Cada jugador debe elegir $N$ números enteros, digamos $(a_1, a_2, ..., a_N)$ y $(b_1, b_2, ..., b_N)$, para el jugador $1$ y $2$ respectivamente, tales que $a_1 + a_2 + ... + a_N = S$ y $b_1 + b_2 + ... + b_N = S$, con $N < S$, donde $a_i$ y $b_i$ es la cantidad de soldados ubicados el $i$-ésimo campo por el primer y segundo jugador, respectivamente. Para estas distribuciones, la ganancia del jugador $1$ viene dada por:
\begin{alignat}{1}
|\{ 1 \leq i \leq N : a_i > b_i\}| - |\{ 1 \leq i \leq N : a_i < b_i\}|
\end{alignat}

Este juego depende de dos parámetros: el número de soldados $S$ y el número de campos de batallas $N$, por lo que la matriz de pagos no es constante y por eso no es presentada como en los juegos anteriores. La matriz para de un juego con $S$ soldados y $N$ es una matriz cuadrada de tamaño $\binom{N+S-1}{S-1}$.
 % REORGANIZED: copied into regret-matching.tex
%\section{Detalles de Implementación y Ejecución}

Los algoritmos fueron implementados en el lenguaje de programación C++, utilizando la librería estándar y una librería adicional llamada \textit{Eigen}, para factorizar matrices y resolver sistemas de ecuaciones.

Se implementó una clase para encontrar un equilibrio de Nash mediante el algoritmo de \textit{Regret Matching}. En cada iteración la actualización de las estrategias depende de cada procedimiento según las fórmulas propuestas en la sección anterior.

En el juego Coronel Blotto la matriz de pagos no tiene un tamaño fijo y además no es proporcionada de forma explícita, por lo que es necesario generarla dependiendo de los parámetros. Para esto se creó un programa que, dado el número de campos de batalla ($N$) y el número de soldados ($S$), genera todas las posibles distribuciones de cada uno de los jugadores mediante un algoritmo de \textit{backtracking} y calcula el pago para cada juego posible, obteniendo como salida del programa la matriz deseada. De esta forma se generó la matriz de pagos para este juego cuando $N = 3$ y $S = 5$.

Las ejecuciones de estos algoritmos se realizaron en una máquina personal, con las siguientes características:
\begin{itemize}[noitemsep]
    \item Procesador: Intel\textsuperscript{\textregistered} Core\textsuperscript{\texttrademark} i5-8250U CPU  \makeatletter{@} 1.60GHz
    \item 8CPUs
    \item 8Gb de memoria RAM
    \item Sistema Operativo: Ubuntu 18.04.3 LTS
\end{itemize}

 % REORGANIZED: copied into regret-matching.tex
%\section{Resultados Experimentales}

En esta sección se presenta un resumen de los resultados experimentales obtenidos utilizando el algoritmo \textit{Regret Matching} en los juegos descritos. Cada procedimiento fue probado $10$ veces por juego, finalizando cada corrida cuando se obtenía un regret máximo menor que $0.005$.

La tabla \ref{tab:resumen-resultados-RM} muestra un resumen de los resultados. En esta tabla se muestra, por cada juego, el tamaño de la matriz de pagos, el valor teórico del juego ($u_t$), el valor del juego utilizando la estrategia obtenida en la última corrida del algoritmo ($u_e$) y la explotabilidad ($\varepsilon_{\sigma}$). Las dos últimas métricas se presentan para cada uno de los procedimientos (A, B y C). En todos los casos se observa que la utilidad esperada para las estrategias obtenidas es cercana al valor del juego, además, se obtuvo una explotabilidad menor o igual que $0.011$, por lo que las estrategias obtenidas representan buenas aproximaciones al equilibrio de Nash en cada juego.

\begin{table}[htpb]
    \centering
    \begin{tabular}{l|r|r r r|r r r}
        &  & \multicolumn{3}{|c}{$u_e$} & \multicolumn{3}{|c}{$\varepsilon_{\sigma}$}  \\ \hline
        Juegos & $u_t$ & \multicolumn{1}{|c}{A} & \multicolumn{1}{c}{B} & \multicolumn{1}{c|}{C}
          & \multicolumn{1}{|c}{A} & \multicolumn{1}{c}{B} & \multicolumn{1}{c}{C} \\ \hline
        Matching Pennies
            & $0$ & $0$ & $0$ & $0$ & $0.006$ & $0.006$ & $0.008$ \\
        Piedra, Papel o Tijeras
            & $0$ & $-0.000012$ & $0.000004$ & $0.000022$ & $0.006$ & $0.01$ & $0.009$ \\
        Ficha vs. Dominó
            & $\frac{1}{3}$ & $0.333$ & $0.334$ & $0.334$ & $0.01$ & $0.007$ & $0.004$ \\
        Coronel Blotto
            & $0$ & $0.000219$ & $-0.000502$ & $0.000024$ & $0.01$ & $0.011$ & $0.009$ \\
        \hline
    \end{tabular}
    \caption{Resumen de los resultados y evaluación de las estrategias obtenidas usando el algoritmo de Regret Matching en juegos en forma normal}
    \label{tab:resumen-resultados-RM}
\end{table}

Para evaluar la convergencia se midió el tiempo necesario para alcanzar el regret desea y el número de iteraciones, en la Tabla \ref{tab:resumen-regret-tiempo-RM} se presenta el tiempo ($T$), el número de iteraciones ($I$) y el tiempo por iteración ($T/I$) obtenido para cada uno de los juegos para cada procedimiento. Estos resultados son el promedio de las $10$ corridas realizadas por juego y procedimiento. Además, se crearon gráficas del regret por iteración para observar como disminuye a medida que corre el algoritmo, la  Figura  \ref{fig:regret-blotto} muestra las gráficas para el juego Coronel Blotto y los $3$ procedimientos. Estas gráficas son mostradadas con una escala logarítmica en el eje $x$ para apreciar mejor los resultados. En el Apéndice A se muestran las tablas detalladas con los resultados en cada una de las corridas y las gráficas de cada juego con los $3$ procedimientos.

\begin{figure}[htpb!]
    \caption{Gráficas del regret con respecto al número de iteraciones del juego Coronel Blotto}
    \label{fig:regret-blotto}
    \centering
    \includegraphics[width=0.58\textwidth]{graficas/coronel-blotto/procedimiento-A.png}
    \includegraphics[width=0.58\textwidth]{graficas/coronel-blotto/procedimiento-B.png}
    \includegraphics[width=0.58\textwidth]{graficas/coronel-blotto/procedimiento-C.png}
\end{figure}

\begin{table}[htpb]
    \centering
    \begin{tabular}{l | c |r r r}
        Juegos & Proc. & $T$ & $I$ & $T/I$ \\ \hline
        \multirow{3}{*}{Matching Pennies}
            & A & $10.276$ & $3892550.4$ & $2.64 {\times} 10^{-06}$ \\
            & B &  $0.777$ &   $25616.6$ & $3.03 {\times} 10^{-05}$ \\
            & C &  $0.042$ &   $16260.5$ & $2.58 {\times} 10^{-06}$ \\ \hline
        \multirow{3}{*}{Piedra, Papel o Tijeras}
            & A & $12.198$ & $4519054.1$ & $2.70 {\times} 10^{-06}$ \\
            & B &  $0.345$ &    $6601.3$ & $5.23 {\times} 10^{-05}$ \\
            & C &  $0.049$ &   $19321.1$ & $2.54 {\times} 10^{-06}$ \\ \hline
        \multirow{3}{*}{Ficha vs. Dominó}
            & A & $319.179$ & $108319272.4$ & $2.95 {\times} 10^{-06}$ \\
            & B &  $11.275$ &     $75250.2$ & $1.50 {\times} 10^{-04}$ \\
            & C &   $0.237$ &     $84318.5$ & $2.81 {\times} 10^{-06}$ \\ \hline
        \multirow{3}{*}{Coronel Blotto}
            & A & $875.533$ & $190222305.3$ & $4.60 {\times} 10^{-06}$ \\
            & B &  $79.358$ &     $66378.4$ & $1.20 {\times} 10^{-03}$ \\
            & C &   $0.166$ &     $48613.5$ & $3.41 {\times} 10^{-06}$ \\ \hline
    \end{tabular}
    \caption{Resumen de los resultados y evaluación de las estrategias obtenidas usando el algoritmo de Regret Matching en juegos en forma normal}
    \label{tab:resumen-regret-tiempo-RM}
\end{table}

A continuación se se analiza el desempeño de los procedimientos, comparándolos entre sí, observando la rapidez de convergencia de cada uno de ellos.

\subsection{Complejidad de cada iteración}

Los procedimientos cambian en la forma en que se elige la siguiente estrategia en cada iteración. En los procedimientos A y B se utiliza un regret condicional, en el que se mide el \textit{arrepentimiento} de cambiar una estrategia por otra en específica. Esta métrica se debe mantener a lo largo de todas las iteraciones, por lo que cada iteración necesita memoria adicional de complejidad $\mathcal{O}(N^2 + M^2)$, donde $N$ y $M$ es el número de acciones posibles para el jugador $1$ y $2$, respectivamente. En el procedimiento C se utiliza únicamente el regret incondicional, por lo que la cantidad de memoria adicional es del orden $\mathcal{O}(N + M)$.

Con respecto a la complejidad de tiempo se tiene que los procedimientos de regret condicional e incondicional (A y C), son lineales al número de acciones. Sin embargo, en el procedimiento B es necesario resolver un sistema de ecuaciones lineales para elegir cada estrategia nueva, del tamaño del número de acciones del jugador respectivo, obteniendo que la complejidad total es $\mathcal{O}(N^3 + M^3)$. La Tabla \ref{tab:complejidades-iteraciones} muestra un resumen de la complejidad es tiempo y memoria adicional.

\begin{table}[ht]
    \centering
    \begin{tabular}{c|c|c}
         Procedimiento & Memoria & Tiempo  \\ \hline
         A & $\mathcal{O}(N^2 + M^2)$ & $\mathcal{O}(N + M)$ \\ 
         B & $\mathcal{O}(N^2 + M^2)$ & $\mathcal{O}(N^3 + M^3)$ \\
         C & $\mathcal{O}(N + M)$     & $\mathcal{O}(N + M)$ \\ \hline
    \end{tabular}
    \caption{Complejidad por iteración de cada uno de los procedimientos}
    \label{tab:complejidades-iteraciones}
\end{table}

Por lo anterior, se observa que la velocidad de las iteraciones del procedimiento que calcula el vector invariante de probabilidad es más lenta en todos los casos, estando uno o dos órdenes de magnitud por encima, según el tamaño de la matriz. Por lo que, si la matriz es sumamente grande, el segundo método sería el menos adecuado.

\subsection{Número de iteraciones}

En la Tabla \ref{tab:resumen-regret-tiempo-RM} se puede observar un resumen de las iteraciones promedio de los tres procedimientos en cada uno de los juegos. Se nota que el procedimiento A, regret incondicional es el que necesita muchas más iteraciones para converger. Con respecto a los procedimientos B y C, se observa que en algunos casos el promedio en el procedimiento B fue menor y en otros el promedio del procedimiento C. También es importante destacar que en el juego de piedra, papel o tijera se tienen varios casos donde se obtiene la convergencia en menos de $10$ iteraciones (ver apéndice A), esos son casos donde se obtiene el equilibrio de Nash de forma exacta en pocas iteraciones.

\subsection{Tiempo transcurrido}

Observando el tiempo promedio de los procedimientos en la tabla \ref{tab:resumen-regret-tiempo-RM}, se nota que el procedimiento A es el que emplea más tiempo en todos los casos, esto ocurre porque necesita muchas más iteraciones que los otros dos procedimientos. Por otra parte el procedimiento C es también más rápido que el procedimiento B, ya que la complejidad en cada iteración para resolver el sistema de ecuaciones enlentece el tiempo total necesario, incluso, si la matriz es muy grande este procedimiento podría ser más lento que el procedimiento A y no sería factible.

Aunque el procedimiento donde se aplica regret matching al regret incondicional (procedimiento C), es el más sencillo de implementar y el más rápido en converger, este procedimiento tiene una desventaja con respecto a los otros dos. Al utilizar el regret condicional, los dos primeros procedimientos garantizan que el regret condicional tiende a cero para cualquier par de estrategias de cada jugador y por lo tanto, conducen siempre a un equilibrio correlacionado. El tercer procedimiento sólo minimiza el regret incondicional y por lo tanto, si el juego es de más de dos jugadores o no es de suma cero, entonces ya no es de utilidad para hallar alguna solución del juego.

 % REORGANIZED: copied into regret-matching.tex
\chapter{Conterfactual Regret Minimization}

\section{Monte Carlo Conterfactual Regret Minimization}

En la sección \ref{section:cfr} se explicó el algoritmo de CFR, utilizado para resolver juegos en forma extensiva. Sin embargo, en la versión presentada es necesario recorrer el árbol completo en cada iteración, esta versión suele conocerse como \textit{vanilla} CFR. En \cite{bib:montecarlo-cfr} se describe una familia general de algoritmos CFR (basados en muestreo) denominada \textbf{Monte Carlo Conterfactual Regret Minimization} (MCCRF), para evitar recorrer el árbol completo en cada iteración.

La idea general es restringir los estados terminales alcanzados en cada iteración, pero manteniendo el mismo valor esperado para la utilidad contrafactual. Dada la definición \ref{def:informacion-incompleta}, sea $\mathcal{Q} = \{Q_1, Q_2, ..., Q_r\}$, un conjunto de subconjuntos de $Z$ tal que su unión sea igual a $Z$. Cada uno de estos conjuntos será llamado un bloque. Sea $q_j > 0$ la probabilidad de considerar el bloque $Q_j$ para la iteración actual (donde $\sum_{j = 1}^r {q_j} = 1$). 
Sea $q(z) = \sum_{j | j \in Q_i}$, es decir, $q(z)$ es la probabilidad de considerar $z$ en la iteración actual. La utilidad contrafactual muestreada, cuando se actualiza el bloque $j$ es:

\begin{alignat}{1}
\tilde{u}_i(\sigma, I | j) = \sum_{h \in I, z \in Q_j} \frac{\pi^{\sigma_{-i}}(h) \pi^{\sigma}(h, z) u_i(z)}{q(z) \pi^{\sigma_{-i}}(I)}
\end{alignat}

\begin{lemma}
$E_{j \sim q_j} [\tilde{u}_i(\sigma, I | j)] = u_i(\sigma, I)$
\end{lemma}

\begin{proof}
\begin{alignat}{2}
    E_{j \sim q_j}[\tilde{u}_i(\sigma, I | j)] & = \sum_{j} {q_j u_i(\sigma, I)} \\
    & = \sum_{j} { q_j \frac{\sum_{h \in I, z \in Q_j} \pi^{\sigma_{-i}}(h) \pi^{\sigma}(h, z) u_i(z)}{q(z) \pi^{\sigma_{-i}}(I)}} \\ \label{eq:def-esperanza}
    & = \sum_{j} \sum_{ \substack{h \in I \\ z \in Q_j}} \frac{q_j \pi^{\sigma_{-i}}(h) \pi^{\sigma}(h, z) u_i(z)}{ q(z) \pi^{\sigma_{-i}}(I)} \\ \label{eq:reorder-sum-1}
    & = \sum_{ \substack{h \in I \\ z \in Z} } \sum_{j | z \in Q_j} \frac{q_j \pi^{\sigma_{-i}}(h) \pi^{\sigma}(h, z) u_i(z)}{ q(z) \pi^{\sigma_{-i}}(I)} \\ \label{eq:reorder-sum-2}
    & = \sum_{ \substack{h \in I \\ z \in Z} } \left(\frac{\sum_{j | z \in Q_j} q_j }{q(z)}\right) \frac{\pi^{\sigma_{-i}}(h) \pi^{\sigma}(h, z) u_i(z)}{\pi^{\sigma_{-i}}(I)} \\
    & = \sum_{ \substack{h \in I \\ z \in Z} } \frac{\pi^{\sigma_{-i}}(h) \pi^{\sigma}(h, z) u_i(z)}{\pi^{\sigma_{-i}}(I)}  = u_i(\sigma, I) \label{eq:lemma-final-eq}
\end{alignat}

La ecuación \ref{eq:def-esperanza} se obtiene de la definición de $\tilde{u}_i(\sigma, I | j)$. \ref{eq:reorder-sum-1} y \ref{eq:reorder-sum-2} se obtienen al reordenar las sumatorias y considerando que la unión de los bloques generan a $Z$. La ecuación \ref{eq:lemma-final-eq}
\end{proof}

Si se elige $\mathcal{Q} = {Z}$, es decir un único bloque con todas las historias terminales y $q_1 = 1$, la utilidad contrafactual es igual a la utilidad contrafactual muestreada y se obtiene el algoritmo \textit{vanilla} CFR. Si se eligen los bloques para incluir todas las historias terminales con la misma secuencia de acciones en los nodos de azar se obtiene el \textit{chance-sampled} CFR, siendo esta última versión la utilizada para estudiar los juegos presentados en este trabajo de grado. Se implementa el algoritmo como es detallado en \cite{bib:introductionCFR} que se presenta en el \textit{\textbf{apéndice X}}.
\section{Evaluación de las estrategias y explotabilidad}

Para evaluar la convergencia de los algoritmos y la estrategia obtenida se utilizaron las métricas de \textit{regret} y explotabilidad, respectivamente.

La explotabilidad se obtiene al calcular la mejor respuesta de la estrategia de cada jugador y sumar los resultados, como se explicó en la sección \textit{\textbf{Explicar explotabilidad en el capítulo 2}}. Sin embargo, la diferencia es que en los juegos de forma extensiva no se pueden listar todas las estrategias fácilmente como en los juegos en forma normal, ya que esta tarea es exponencial en el tamaño del árbol.

Para calcular la explotabilidad en estos juegos se utilizó el algoritmo propuesto en \cite{bib:thesis-marc-lanctot}, denominado \textit{Generalized Expectimax Best Response} (GEBR), descrito en el apéndice \textit{\textbf{X}}. La complejidad de este algoritmo es $\mathcal{O}(ND)$ donde $N$ es el número de nodos del árbol y $D$ es la profundidad del árbol. Note que el algoritmo tiene una alta complejidad, por lo que se usará únicamente para calcular la explotabilidad de la estrategia final.
\section{Detalles de implementación}
Los algoritmos y la representación de los juegos fueron implementados en el lenguaje de programación C++. Para la representación de los juegos se utilizó una clase abstracta llamada \textit{Game}, que recibe como template los tipos para el estado, las acciones, las propiedades, los conjunto de información y el Hash del juego específico.

Esta clase contiene las funciones virtuales necesarias para recorrer el árbol del juego de forma \textbf{implícita}, tales como: \textit{actions}, que retornan las acciones del juegos en el estado actual, \textit{update\_state}, que actualiza el estado del juego dada una acción a realizar, \textit{terminal\_state} que indica si un estado es terminal o no, \textit{utiliy} que retorna la utilidad en un estado terminal, entre otras. Los algoritmos CFR y GEBR utilizan esta clase abstracta en su implementación.

Para cada tipo de juego, se creó una clase derivada de la clase \textit{Game}, donde se implementaron las funciones según las reglas de cada juego. De esta forma se puede utilizar la misma implementación de los algoritmos para todos los juegos.

Cabe destacar que todos los juegos fueron representados mediantes árboles con la raíz como único nodo de azar. Algunos juegos tienen esta representación de forma natural, por ejemplo, el Kuhn Poker, ya que las cartas se reparten al inicio y luego se juega acorde a esa distribución, sin volver a introducir ninguna jugada aleatoria. Otros juegos pueden tener nodos de azar distintos a la raíz, sin embargo siempre es posible transformarlos a un árbol que represente el mismo juego donde todos los nodos de azar son condensados en la raíz y cada hijo de la raíz representa una elección por cada uno de los nodos de azar del árbol original. En esta representación se asume que todas las decisiones aleatorias se toman al inicio del juego.

La clase \textit{Game} y todos los algoritmos se implementados suponiendo la raíz como único nodo de azar del juego.
\section{Descripción de los juegos}
\label{section-description-juegos-forma-extensiva}

Para probar los algoritmos se implementaron tres tipos de juegos diferentes: \textit{One Card Poker} (OCP), \textit{Dudo}, un juego de dados, y una versión del juego de dominó para $2$ personas. La descripción detallada de las reglas de los juegos se describen en esta sección.

\subsection{One-Card Poker}
One-Card Poker, abreviado OCP(N), es la versión generalizada del juego Kuhn Póker, explicado en la sección \ref{section:kuhn-poker}. En este juego, cada jugador recibe una carta de un mazo de $N$ cartas, y luego pueden apostar o retirarse según las mismas reglas del Kuhn Póker. Note que OCP(3) es equivalente al Kuhn Poker. El árbol de este juego tiene $9N(N-1)+1$ nodos (incluyendo el nodo inicial, que es el nodo de azar) y hay $4N$ conjuntos de información entre ambos jugadores. 

\subsection{Dudo}
Dudo, también conocido como \textit{Bluff}, \textit{Liar's Dice} o Perudo, es un juego de dados y apuestas. Usualmente se juega entre $2$ y $6$ jugadores. Los jugadores se ubican en forma circular y cada uno de ellos tiene un número de dados. De forma simultánea, todos lanzan sus dados, cada jugador puede ver el resultado de sus propios dados, pero no puede ver el resultado de los dados de los otros jugadores. Una vez hecho esto, los jugadores empiezan a apostar sobre el número de veces que apareció una cara en específico en todos los dados que hay en la mesa.

Una apuesta consiste en decir $2$ números $(x, y)$, esto indica que el jugador apuesta que hay, al menos, $x$ dados cuyo resultado fue el número $y$. El primer jugador (que se elige previamente mediante el lanzamiento de $1$ dado o de alguna otra forma), realiza la primera apuesta y, en sentido horario, cada jugador puede hacer una apuesta más alta o decir \say{dudo} y retar al jugador anterior. Una apuesta es más alta que otra si el número de dados que se anuncian en la apuesta ($x$) es mayor, o si el número de dados es igual, pero la cara apostada ($y$) es mayor. Por ejemplo $(3, 1)$ es mayor que $(2, 5)$, y ambas apuestas son mayores que $(2, 3)$.

Por otra parte, si un jugador reta al jugador previo, se descubren todos los dados de todos los jugadores. Si la cantidad de dados con la cara $y$ es mayor o igual a $x$, donde $(x, y)$ fue la apuesta realizada por el jugador, el jugador que hizo el reto pierde un dado. En caso contrario, el jugador que hizo la apuesta pierde un dado. Luego, todos los jugadores lanzan sus dados nuevamente y una nueva ronda de apuestas empieza por el jugador que perdió la ronda anterior. Un jugador pierde cuando se queda sin dados, el ganador es el último jugador con al menos un dado restante. La figura \ref{fig:dudo} muestra una foto del juego, de \textit{Perudo}, una versión comercial de este juego, que está diseñada para $6$ jugadores, donde cada jugador empieza con $5$ dados. En la figura se observan los vasos que se utilizan para lanzar los dados y evitar que cada jugador vea los dados de los demás.

\begin{figure}[htb]
\caption[Juego Dudo]{Juego Dudo. Los vasos se utilizan para lanzar los dados y evitar que los oponentes vean el resultado}
\label{fig:dudo}
\centering
\includegraphics[width=0.6\textwidth]{figuras/dudo.jpg}
\end{figure}

En este trabajo de grado consideraremos este juego para $2$ jugadores únicamente. Dudo$(K, D_1, D_2)$ hará referencia a una única ronda de apuestas de $2$ jugadores, donde el primer jugador tiene $D_1$ dados, el segundo jugador tiene $D_2$ dados y cada dado tiene $K$ caras. El juego completo consiste en múltiples rondas, donde $D_1$ o $D_2$ disminuye en una unidad al finalizar cada ronda. Cuando uno de los jugadores pierde todos los dados obtiene una utilidad de $-1$, mientras que su oponente obtiene una utilidad de $1$. En este juego cada ronda se considerará un subjuego y se representará con un árbol independiente, donde los valores esperados para los juegos Dudo$(K, D_1 - 1, D_2)$ y Dudo$(K, D_1, D_2 - 1)$ se precalculan y se utilizan como utilidad para las hojas del árbol Dudo$(K, D_1, D_2)$. Note que, en el juego estándar, $K$ siempre tiene un valor de $6$.

Cuando el jugador $i$ lanza $D_i$ dados hay $\binom{D_i+K-1}{K-1}$ resultados posibles diferentes, ya que cada resultado puede ser representado con una tupla $(a_1, a_2, ..., a_k)$, donde $a_j$ representa el número de dados con la cara $j$, por lo que $\sum_j^K = D_i$ y $a_j \geq 0$. Por otra parte cada secuencia de apuestas puede ser representada por una secuencia binaria de longitud $K(D_1 + D_2)$, donde el $i$-ésimo bit es $1$ si la $i$-ésima secuancia más fuerte fue dicha durante la ronda y $0$ en caso contrario. Por ejemplo, si $D_1 = D_2 = 1$, las apuestas $(1, 1)-(1, 3)-(1, 6)-(2, 4)-(2, 5)-(1, 6)$ se representa con la secuencia binaria $101001000110$, por lo que hay $2^{K(D_1 + D_2)}$ secuencias diferentes. Cada secuencia pertenece a un jugador en específico, por lo que si $D_1 = D_2$, el número de conjuntos de información  es igual a  $\binom{D_i+K-1}{K-1}2^{K(D_1 + D_2)}$.

Para contar el número total de nodos, se puede considerar el lanzamiento de los dados de forma independiente, pues las secuencias posibles de apuestas no dependen del resultado de los dados. Por lo expuesto anteriormente el número posible de apuestas es igual a $2^{K(D_1+D_2)}$, pero después de cada secuencia siempre se puede decir \say{dudo}, salvo para la secuencia vacía. Luego el número total de nodos (incluyendo nodos terminales y no terminales) es igual a $\binom{D_1+K-1}{K-1}\binom{D_1+K-1}{K-1}(2^{K(D_1+D_2)+1}-1)+1$.

\subsection{Domino}
En este trabajo se utilizó una versión de este juego para $2$ jugadores. Al inicio del juego cada jugador toma una cantidad específica de piezas de forma aleatoria, las piezas restantes se dejan sin descubrir para ser usadas en turnos posteriores. Como en el juego tradicional de dominó, los jugadores juegan por turnos alternados (el primero jugador se elige de forma arbitraria), cada uno debe colocar una ficha válida acorde a las reglas \textit{estándares} en Venezuela del juego (ver apéndice \textit{\textbf{X}}). Si un jugador no puede colocar una ficha toma una ficha de las que no están descubiertas (si todavía hay disponibles), el jugador verifica si puede colocar la ficha tomada y en caso contrario pasa el turno y juega el oponente.

El juego termina cuando alguno de los jugadores usa todas las piezas o cuando ambos jugadores no pueden jugar ni tomar piezas nuevas, en este último caso se dice que el juego está bloqueado. El ganador es el jugador que se queda sin piezas o, en caso de bloqueo, el jugador que acumule menos puntos en todas las piezas que quedaron en su mano. La utilidad obtenida es el número de puntos que el jugador perdedor acumuló en las piezas que quedaron en su mano (con signo positivo para el jugador ganador y signo negativo para el perdedor). Cabe destacar que sólo se puede tomar una pieza o pasar, si no se puede realizar una jugada con la mano actual.

Usualmente se utilizan $28$ piezas, donde las piezas pueden tener entre $0$ y $6$ puntos en cada extremo, y cada jugador recibe $7$ piezas al inicio del juego. En este trabajo se parametriza el número máximo de puntos que puede tener una ficha, así como la cantidad de piezas repartidas inicialmente. De esta forma se hará referencia a Domino$(M, N)$ an un juego donde las piezas tienen entre $0$ y $M$ puntos (con un total de $M(M+1)/2$ piezas) y cada jugador recibe $N$ piezas al inicio del juego.

En este juego no es fácil calcular el tamaño del árbol y el número de conjuntos de información, principalmente porque las acciones posibles en un estados dependen tanto de la mano del jugador, como de las piezas en la mesa. En el Kuhn Póker simpre hay $2$ acciones posibles ${pasar, apostar}$ y en el Dudo las acciones disponibles dependen únicamente de la última apuesta y no dependen de los dados que tengan los jugadores. Así que se decidió estimar estos parámetros recorriendo el árbol del juego mediante DFS. La tabla \ref{tab:tree-domino}, muestra el número de nodos del árbol y el número de conjuntos de información por cada juego de dominó que se presenta.

\begin{table}[ht]
    \centering
    \begin{tabular}{c|r|r}
        & Conjutos de Información & Nodos \\ \hline
       Domino$(1, 1)$ & $3$ & $13$ \\
       Domino$(2, 2)$ & $441$   & $7321$ \\
       Domino$(3, 2)$ & $844437$   & $46534657$ \\
       Domino$(3, 3)$ & $1082290$   & $246760993$ \\ 
       Domino$(3, 4)$ & $902218$   & $1547645185$ \\ \hline
    \end{tabular}
    \caption{Número de nodos y conjuntos de Información en diferentes juegos de Dominó}
    \label{tab:tree-domino}
\end{table}



\subsection{Resultados experimentales}

Se crearon varias instancias de los $3$ juegos explicados en la sección \ref{section-description-juegos-forma-extensiva} con diferentes parámetros. Para cada instancia se utilizó el algoritmo de CFR y se iteró sobre el árbol durante $10$ (\textit{\textbf{número tentativo}}) horas (se excluye el tiempo que se calcula el regret ya que no forma parte del algoritmo y esto se hace únicamente para obtener las gráficas). Una vez terminado el tiempo asignado se calcula la mejor respuesta para cada jugador y la explotabilidad. Una instancia de un juego se considerará resuelta si la explotabilidad de la estrategia obtenida es menor que el $1\%$ de la mínima unidad de utilidad posible según cada juego.

La tabla \ref{tab:resultados-CFR} resume los resultados, donde $N$ representa el número de nodos del árbol $I$ el número de conjuntos de información, $u_{\sigma}$ el valor del juego usando la estrategia obtenida y $\varepsilon_{\sigma}$ la explotabilidad. También se agrega el número de iteraciones y la última columna indica si el juego fue resuelto o no, según lo establecido en el párrafo anterior.

\begin{table}[ht]
    \centering
    \begin{tabular}{l|r|r|r|r|r|c}
        Juego & $N$ & $I$ & Iteraciones & $u$ & $\varepsilon$ & Resuelto \\ \hline
        OCP$(3)$    &        $55$ &    $12$ & & & & \cmark \\
        OCP$(12)$   &      $1189$ &    $48$ & & & & \cmark \\
        OCP$(50)$   &     $22051$ &   $200$ & & & & \cmark \\
        OCP$(200)$  &    $358201$ &   $800$ & & & & \cmark \\
        OCP$(1000)$ &   $8991001$ &  $4000$ & & & & \cmark \\
        OCP$(4000)$ & $143964001$ & $16000$ & & & & \cmark \\
        \hline
        Dudo$(3, 1, 1)$ &      $1144$ &      $192$ & & & & \cmark \\
        Dudo$(3, 2, 1)$ &     $18415$ &     $2304$ & & & & \cmark \\
        Dudo$(3, 1, 2)$ &     $18415$ &     $2304$ & & & & \cmark \\
        Dudo$(3, 2, 2)$ &    $294877$ &    $24576$ & & & & \cmark \\
        Dudo$(4, 1, 1)$ &      $8177$ &     $1024$ & & & & \cmark \\
        Dudo$(4, 2, 1)$ &    $327641$ &    $28672$ & & & & \cmark \\
        Dudo$(4, 1, 2)$ &    $327641$ &    $28672$ & & & & \cmark \\
        Dudo$(4, 2, 2)$ &  $13107101$ &   $655360$ & & & & \xmark \\
        Dudo$(5, 1, 1)$ &     $51176$ &     $5120$ & & & & \cmark \\
        Dudo$(5, 2, 1)$ &   $4915126$ &   $327680$ & & & & \cmark \\
        Dudo$(5, 1, 2)$ &   $4915126$ &   $327680$ & & & & \cmark \\
        Dudo$(5, 2, 2)$ & $471858976$ & $15728640$ & & & & \xmark \\
        Dudo$(6, 1, 1)$ &    $294877$ &    $24576$ & & & & \cmark \\
        Dudo$(6, 2, 1)$ &  $66060163$ &  $3538944$ & & & & \cmark \\
        Dudo$(6, 1, 2)$ &  $66060163$ &  $3538944$ & & & & \cmark \\
        Dudo$(6, 2, 2)$ &  $14797504071$ &  $352321536$ & & & & \xmark \\
        \hline
        Domino$(2, 2)$ &     $441$ &       $7321$ & & & & \cmark \\
        Domino$(3, 2)$ &  $844437$ &   $46534657$ & & & & \cmark \\
        Domino$(3, 3)$ & $1082290$ &  $246760993$ & & & & \cmark \\
        Domino$(3, 4)$ &  $902218$ & $1547645185$ & & & & \cmark \\
        Domino$(4, 2)$ & & & & & & \xmark \\
        \hline
    \end{tabular}
    \caption{Resultados del algortimo CFR en los diferentes juegos}
    \label{tab:resultados-CFR}
\end{table}

La gráfica \textit{\textbf{Mostrar una o dos gráficas interesantes y decir algo al respecto}}. En el apéndice \textit{\textbf{X}} se pueden observar todas las gráficas del regret con respecto al número de iteraciones, se nota que el regret tiende a $0$ en todos los casos (\textit{\textbf{Agregar algún otro detalle interesante que se vea en las gráficas}})
\chapter*{Conclusiones}

Sugerencias y revisiones de esta clase enviarlos a los correos \texttt{ccontreras@usb.ve} y \texttt{asajo@usb.ve}.
\nocite{*}
\bibliography{epilogo/referencias}
\appendix
\chapter{Pruebas}
\label{apex:chapter:pruebas}

\makeatletter
\newtheorem*{rep@theorem}{\rep@title}
\newcommand{\newreptheorem}[2]{
    \newenvironment{rep#1}[1]
    {
        \def\rep@title{#2 \ref{##1}}
        \begin{rep@theorem}
    }
     {
        \end{rep@theorem}
     }
}
\makeatother

\newreptheorem{theorem}{Teorema}


\section*{Capítulo I}

\begin{reptheorem}{theo:ganancia-esperada}
La ganancia esperada $u_i(\sigma)$ del jugador $i$ dado el perfil estratégico $\sigma$ satisface:
\begin{alignat}{1}
u_i(\sigma)\ =\ \sum_{s_i\in S_i} \sigma_i(s_i) \sum_{s_{-i}\in S_{-i}} \sigma_{-i}(s_{-i}) u_i(s_i,s_{-i}) \,.
\end{alignat}
\end{reptheorem}

\begin{proof}
Partiendo de la Definición \ref{def:ganancia-esperada} se obtiene
\begin{alignat}{1}
	u_i(\sigma)\  &=\ \sum_{s \in S} u_i(s) \sigma_i(s_i) \sigma_{-i}(s_{-i})\ =\ \sum_{s_i \in S_i} \sum_{s_{-i} \in S_{-i}} u_i(s_i, s_{-i}) \sigma_i(s_i) \sigma_{-i}(s_{-i}) \\
	&=\ \sum_{s_i\in S_i} \sigma_i(s_i) \sum_{s_{-i}\in S_{-i}} \sigma_{-i}(s_{-i}) u_i(s_i,s_{-i})\,.
\end{alignat}
\end{proof}

\begin{lemma}
\label{lemma:2}
Sea $\sigma^*_i$ una estrategia mixta para el jugador $i$ que es mejor respuesta a $\sigma_{-i}$, y sea $x\in S_i$ una estrategia pura para el jugador $i$. Entonces, para toda estrategia pura $y\in S_i$ diferente de $x$,
\begin{alignat}{1}
  \sigma^*_i(x) \sum_{s_{-i}} u_i(x,s_{-i}) \sigma_{-i}(s_{-i})\ \geq\ \sigma^*_i(x) \sum_{s_{-i}} u_i(y,s_{-i}) \sigma_{-i}(s_{-i}) \,.
\end{alignat}
\end{lemma}

\begin{proof}
Considere la estrategia mixta $\sigma'_i$ definida por:
\begin{alignat}{1}
	\sigma'_{i}(s_i)\ =\  
	\begin{cases}
		0 &  \text{si } s_i = x \\
		\sigma^*_i(x) + \sigma^*_i(y) & \text{si } s_i = y \\
		\sigma^*_i(s_i) & \text{en otro caso} \,. 
	\end{cases}
\end{alignat}
Utilizando el Teorema~\ref{theo:ganancia-esperada} y el hecho que $\sigma^*_i$ es mejor respuesta a $\sigma_{-i}$:
\begin{alignat}{1}
  u_i(\sigma^*_i, \sigma_{-i})\ 
    &\geq\ u_i(\sigma'_i, \sigma_{-i}) \\
    &=\ \sum_{z \in S_i} \sigma'_i(z) \sum_{s_{-i}} u_i(z,s_{-i}) \sigma_{-i}(s_{-i}) \\
    &=\ \sum_{z\neq x} \sigma^*_i(z) \sum_{s_{-i}} u_i(z,s_{-i}) \sigma_{-i}(s_{-i}) + \sigma^*_i(x)\sum_{s_{-i}} u_i(y,s_{-i})\sigma_{-i}(s_{-i}) \,.
\end{alignat}
Por el Teorema~\ref{theo:ganancia-esperada},
$u_i(\sigma^*_i, \sigma_{-i})=\sum_{z \in S_i} \sigma^*_i(z) \sum_{s_{-i}} u_i(z,s_{-i}) \sigma_{-i}(s_{-i})$. Entonces,
\begin{alignat}{1}
  \label{eq:ineq-ganancias}
  \sigma^*_i(x) \sum_{s_{-i}} u_i(x,s_{-i}) \sigma_{-i}(s_{-i})\
    &\geq\ \sigma^*_i(x)\sum_{s_{-i}} u_i(y,s_{-i}) \sigma_{-i}(s_{-i}) \,.
\end{alignat}
\end{proof}


\begin{reptheorem}{theo:mejor-respuesta}
Sea $\sigma^*_i$ una estrategia mixta para el jugador $i$ que es mejor respuesta a $\sigma_{-i}$. Cualquier estrategia mixta $\sigma_i$ para el jugador $i$ cuyo soporte sea un subconjunto del soporte de $\sigma^*_i$ es también una mejor respuesta a $\sigma_{-i}$.
\end{reptheorem}

\begin{proof}
Sea $x \in S_i$ una \emph{estrategia pura} perteneciente al soporte de $\sigma^*_i$, y sea $y \in S_i$ una estrategia pura \emph{diferente} de $x$. 

Por el Lema~\ref{lemma:2},
\begin{alignat}{1}
  \sum_{s_{-i}} u_i(x,s_{-i}) \sigma_{-i}(s_{-i})\ \geq\  \sum_{s_{-i}} u_i(y,s_{-i}) \sigma_{-i}(s_{-i}) \,.
\end{alignat}

En particular, si $x$ y $x'$ son distintos, y ambos pertenecen al soporte de $\sigma_i$,
\begin{alignat}{1}
\sum_{s_{-i}} u_i(x,s_{-i}) \sigma_{-i}(s_{-i})\ =\ \sum_{s_{-i}} u_i(x',s_{-i}) \sigma_{-i}(s_{-i})\ =\ C \,,
\end{alignat}
donde $C$ es una constante que s\'olo depende de $\sigma_{-i}$.
Luego, para cualquier estrategia $\sigma_i$, tal que $support(\sigma_i) \subseteq support(\sigma^*_i)$, se tiene:
\begin{alignat}{1}
u_i(\sigma_i, \sigma_{-i})\ &=\ \sum_{x \in S_i} \sigma_i(x) \sum_{s_{-i}} u_i(x,s_{-i}) \sigma_{-i}(s_{-i})\ =\ \sum_{x \in S_i} \sigma_i(x) C\ =\ C \,.
\end{alignat}

Luego, $u_i(\sigma^*_i,\sigma_{-i})=C$, y $\sigma_i$ es también mejor respuesta a $\sigma_{-i}$.
\end{proof}

\begin{reptheorem}{theo:nash-correlacionado}
Si $\sigma$ es un equilibrio de Nash, entonces $\sigma$ es un equilibrio correlacionado.
\end{reptheorem}

\begin{proof}
Sea $\sigma$ un equilibrio de Nash, sean $x,y\in S_i$ estrategias puras distintas cualesquiera para el jugador $i$, y sea $\sigma'_i$ una estrategia mixta cualquiera para el jugador $i$. Por el Lema~\ref{lemma:2},
\begin{alignat}{1}
  \sigma_i(x) \sum_{s_{-i}} u_i(x,s_{-i}) \sigma_{-i}(s_{-i})\ \geq\  \sigma_i(x)\sum_{s_{-i}} u_i(y,s_{-i}) \sigma_{-i}(s_{-i}) \,.
\end{alignat}

Es decir,
\begin{alignat}{1}
  0\ \leq\ \sigma_i(x) \sum_{s_{-i}} \sigma_{-i}(s_{-i}) [u_i(x,s_{-i}) - u_i(y,s_{-i})]\ =\ \sum_{s_{-i}} \sigma(x,s_{-i}) [u_i(x,s_{-i}) - u_i(y,s_{-i})] \,.
\end{alignat}
Luego, $\sigma$ es un equilibrio correlacionado.
\end{proof}

\begin{reptheorem}{theo:correlacionado-nash}
Sea $\psi\in\Delta(S)$ un equilibrio correlacionado. Si $\psi$ se factoriza como $\psi=\prod_{i\in N} \sigma_i$ donde $\{\sigma_i\}_{i\in N}$ es un conjunto de estrategias mixtas para cada jugador (i.e., $\psi(s)=\prod_{i \in N} \sigma_i(s_i)$ para todo $s\in S$), entonces $\psi$ es un equilibrio de Nash.
\end{reptheorem}

\begin{proof}
Sea $\psi= \prod_{i \in N} \sigma_i$ un equilibrio correlacionado en forma factorizada. Se debe mostrar que para cualquier jugador $i$ y estrategia mixta $\sigma'_i$ para el jugador $i$, se cumple $u_i(\sigma) \geq u_i(\sigma'_i, \sigma_{-i})$.

Sean $x$ y $y$ estrategias puras para el jugador $i$.
Como $\sigma$ es un equilibrio correlacionado,
\begin{alignat}{1}
\label{eq:1:theo:correlacionado-nash}
0\ \leq\ \sigma_i(x) \sum_{s_{-i}} \sigma_{-i}(s_{-i})[u_i(x, s_{-i}) - u_i(y, s_{-i})] \,.
\end{alignat}

Al sumar sobre $x\in S_i$ se obtiene, 
\begin{alignat}{2}
\label{eq:2:theo:correlacionado-nash}
0\ \leq\ \sum_{x\in S_i} \sum_{s_{-i}} \sigma(x,s_{-i}) [u_i(x, s_{-i}) - u_i(y, s_{-i})]\ =\ \sum_s \sigma(s) [u_i(s) - u_i(y, s_{-i})] \,.
\end{alignat}

Si $x^* \in S_i$ es tal que $\sigma_i(x^*)>0$, se obtiene de \eqref{eq:1:theo:correlacionado-nash} al multiplicar por $\sigma'_i(y)$ y sumar sobre $y\in S_i$:
\begin{alignat}{1}
\label{eq:3:theo:correlacionado-nash}
\sum_{y \in S_i} \sigma'_i(y) \sum_{s_{-i}} \sigma_{-i} (s_{-i}) [u_i(x^*, s_{-i}) - u_i(y, s_{-i})]\ =\ \sum_{s} \sigma'(s) [u_i(x^*, s_{-i}) - u_i(s)]\ \geq\ 0 \,,
\end{alignat}
donde $\sigma'$ denota la estrategia $\sigma'=(\sigma'_i,\sigma_{-i})$. 

Al sumar \eqref{eq:2:theo:correlacionado-nash} y
\eqref{eq:3:theo:correlacionado-nash}, se obtiene que para cualquier $y$ y $x^*$ tal que $\sigma_i(x^*)>0$:
\begin{alignat}{1}
\label{eq:4:theo:correlacionado-nash}
\sum_{s \in S} u_i(s) [\sigma(s) - \sigma'(s)] - \sum_{s \in S} \sigma(s)u_i(y, s_{-i}) + \sum_{s \in S} \sigma'(s) u_i(x^*, s_{-i})\ \geq\ 0\ \,.
\end{alignat}
Por otra parte, note que:
\begin{alignat}{1}
  \sum_{s \in S} \sigma(s) u_i(x^*,s_{-i}) - \sum_{s \in S} &\sigma'(s) u_i(x^*,s_{-i}) \\
    &\qquad=\ \sum_{s_{-i}} u_i(x^*,s_{-i}) \sigma_{-i}(s_{-i}) \sum_{z\in S_i} [\sigma_i(z) - \sigma'_i(z)] \\
    &\qquad=\ \sum_{s_{-i}} u_i(x^*,s_{-i}) \sigma_{-i}(s_{-i}) \biggl[ \sum_{z\in S_i} \sigma_i(z) - \sum_{z\in S_i} \sigma'_i(z) \biggr] \\
    &\qquad=\ 0 \,.
\end{alignat}
Luego, al tomar $y=x^*$ en \eqref{eq:4:theo:correlacionado-nash},
\begin{alignat}{1}
 \sum_{s \in S} u_i(s) [\sigma(s) - \sigma'(s)]\ =\ \sum_{s \in S} u_i(s)\sigma(s) - \sum_{s \in S} u_i(s)\sigma'(s)\ =\ u_i(\sigma) - u_i(\sigma'_i, \sigma_{-i})\ \geq\ 0 \,.
\end{alignat}
Como $\sigma'_i$ es una estrategia cualquiera para el jugador $i$, $\sigma$ es un equilibrio de Nash.
\end{proof}

\begin{reptheorem}{theo:correlacionado-linealidad}
Sean $\sigma$ y $\sigma'$ dos equilibrios correlacionados, y $\alpha$ un número real en $(0,1)$. Entonces, la distribución $\alpha\sigma + (1-\alpha)\sigma'$ es un equilibrio correlacionado.
\end{reptheorem}

\begin{proof}
Como $\sigma$ y $\sigma'$ son equilibrios correlacionados y $\alpha, 1 - \alpha \in (0, 1)$ se cumple que para cualesquiera $x$ e $y$:
\begin{alignat}{2}
 \alpha \sum_{s_{-i} \in S_{-i}} \sigma(x, s_{-i})[u_i(x, s_{-i}) - u_i(y, s_{-i})]\ & \geq\ 0 & \ \text{ y } \\
 (1 - \alpha) \sum_{s_{-i} \in S_{-i}} \sigma'(x, s_{-i})[u_i(x, s_{-i}) - u_i(y, s_{-i})]\ & \geq\ 0 \,.
\end{alignat}

Sumando las ecuaciones anteriores y factorizando se obtiene:
\begin{alignat}{1}
 \sum _{s_{-i} \in S_{-i}} [ \alpha \sigma(x, s_{-i}) +  (1 - \alpha) \sigma'(x, s_{-i})][u_i(x, s_{-i}) - u_i(y, s_{-i})]\ \geq\ 0 \,.
\end{alignat}

Concluyendo que $\alpha \sigma +  (1 - \alpha) \sigma'$ es un equilibrio correlacionado.
\end{proof}

\section*{Capítulo II}

\begin{reptheorem}{theo:comportamiento-a-mixta}
Dado un juego en forma extensa y un jugador $i$, tal que: si $h' \sqsubset h$ y $P(h') = P(h) = i$, entonces $I(h') \neq I(h)$. Luego, para cualquier estrategia de comportamiento $\sigma^b_i \in B^i$, la estrategia mixta $\sigma^m_i$ dada por:
\begin{alignat}{1}
\sigma^m_i(s_i) := \prod_{I_i \in \mathcal{I}_i} \sigma^b_i(I_i)(s_i(I_i)) \label{eq-apex:comportamiento-a-mixta}
\end{alignat}
es equivalente a la estrategia $\sigma^b_i$.
\end{reptheorem}

\begin{proof}
Se quiere probar que para todo $z \in Z$, se tiene que $\pi^{\sigma^m_i}(z) = \pi^{\sigma^b_i}(z)$. 
Para cualquier estrategia se denotará con $\sigma_i(s)$ la probabilidad de elegir la estrategia $s_i$ bajo $\sigma_i$. Además, la probabilidad de elegir una estrategia $s_i$ bajo la estrategia $\sigma^b_i$ es exactamente el lado derecho de la Ecuación \ref{eq:comportamiento-a-mixta}, la cual, por definición es la probabilidad de elegir $s_i$ bajo $\sigma^m_i$. Luego se tiene que $\sigma_i^b(s_i) = \sigma_i^m(s_i)$ para cualquier estrategia pura $s_i \in S_i$.

Por otra parte, como ninguna historia atraviesa más de una vez el mismo conjunto de información, se tiene que para cualquier estrategia $\sigma_i$ (mixta o de comportamiento):

\begin{alignat}{1}
\label{eq:definicion-mixta}
\pi^{\sigma_i}(z)\ =\ \sum_{\substack{s_i \in S_i \\  z \text{ es} \text{alcanzable} \\ \text{por } s_i} } \sigma_i(s_i) \,.
\end{alignat}

Luego, $\pi^{\sigma_i^b}(z) = \pi^{\sigma_i^m}(z)$ para todo $z \in Z$, obteniendo el resultado deseado. 
\end{proof}


\begin{reptheorem}{theo:mixta-a-comportamiento}
Dado un juego finito de $N$ personas en el que el jugador $i$ tiene ``perfect recall''. Entonces, para cada estrategia mixta $\sigma^m_i \in \Delta(S_i)$ del jugador $i$, existe una estrategia de comportamiento $\sigma^b_i \in B^i$, equivalente a $\sigma^m_i$.
\end{reptheorem}

\begin{proof}
Se denotará por $\pi^{\sigma_i}(I_i, a)$ la probabilidad, bajo $\sigma_i$, que $I_i$ sea alcanzable y se elija la acción $a$. De forma más general se denotará con $\pi^{\sigma_i}(I_i, a_1, a_2, ..., a_k)$ la probabilidad de que $I_i$ sea alcanzable y que luego el jugador $i$ elija las acciones $a_1, a_2, ... a_k$. Luego se elige la siguiente estrategia de comportamiento:
\begin{alignat}{1}
\sigma_i^b(I_i)(a)\ &=\ P[\text{se elija}\ a\ \text{bajo}\ \sigma^m_i | I_i\ \text{es alcanzable bajo}\ \sigma^m_i] \\
&=\ \frac{P[I_i\ \text{sea alcanzable bajo}\ \sigma^m_i\ \text{y se elija la opción}\ a\ \text{bajo}\ \sigma^m_i]}{ P[I_i\ \text{es alcanzable bajo}\ \sigma^m_i] } \\
&=\ \frac{\pi^{\sigma^m_i}(I_i, a)}{\pi^{\sigma^m_i}(I_i)} \label{eq:mixta-a-comportamiento} \,.
\end{alignat}
en caso que $\pi^{\sigma^m_i}(I_i) > 0$ y de forma arbitraria en caso contrario.

Se demostrará que $\pi^{\sigma^b_i}(z) = \pi^{\sigma^m_i}(z)$, cuando $\pi^{\sigma^m_i}(z) > 0$.  Dado $z \in Z $, sean $a_1, a_2, ..., a_k$ las acciones elegidas por el jugador $i$ (en ese orden), y sean $I^1_i, I^2_i, ..., I^k_i$ los conjuntos de información respectivos. Note que $\pi^{\sigma^m_i}(I^j_i, a^j_i) = \pi^{\sigma^m_i}(I^{j+1}_i)$, luego:
\begin{alignat}{1}
\pi^{\sigma^b_i}(z)\ &=\ \prod_{j = 1}^k \sigma^b_i(I^j_i)(a^j_i)\ =\ \prod _{j = 1}^k \frac{\pi^{\sigma^m_i}(I^j_i, a_j)}{\pi^{\sigma^m_i}(I^j_i)}\ =\ \frac{\pi^{\sigma^m_i}(I^k_i, a_k)}{\pi^{\sigma^m_i}(I^1_i)}\ =\ \pi^{\sigma^m_i}(I^k_i, a_k) \,.
\end{alignat}

Además, usando inducción, se obtiene que para cualquier $k' < k$ se tiene:
\begin{alignat}{1}
\pi^{\sigma^m_i}(I^k_i, a_k)\ =\ \pi^{\sigma^m_i} (I^{k'}_i, a_{k'}, a_{k'+1}, ..., a_{k})
\end{alignat}

Entonces
\begin{alignat}{1}
\pi^{\sigma^m_i}(I^k_i, a_k)\ =\ \pi^{\sigma^m_i}(I^1_i, a_1, a_2, ..., a_k) = \pi^{\sigma^m_i}(z)
\end{alignat}

Obteniendo $\pi^{\sigma^b_i}(z) = \pi^{\sigma^m_i}(z)$, que era lo que se quería demostrar.
\end{proof}

\section*{Capítulo III}

\begin{reptheorem}{theo:cota-ganancia-esperada}
Sea $\sigma^* = (\sigma^*_1, \sigma^*_2)$ un equilibrio de Nash de un juego de dos jugadores de suma cero, tal que $u_1(\sigma) = u$. Entonces $u_i(\sigma^*) \leq u_i(\sigma^*_i, \sigma_{-i})$, para cualquier estrategia $\sigma_{-i}$.  
\end{reptheorem}

\begin{proof}
Como $\sigma^*$ es un equilibrio de Nash, $\sigma^*_{-i}$ es mejor respuesta a $\sigma^*_i$ y por lo tanto, para cualquier estrategia $\sigma_{-i}$ se obtiene que:
\begin{alignat}{2}
    & u_{-i}(\sigma^*) = u_{-i}(\sigma^*_i, \sigma^*_{-i}) \geq u_{-i}(\sigma^*_i, \sigma_{-i}) \\
    \label{eq:juego-suma-cero}
    \implies\ & -u_i(\sigma^*) \geq -u_i(\sigma^*_i, \sigma_{-i}) \\
    \label{eq:multiplicar-1}
    \implies\ & u_i(\sigma^*) \leq u_i(\sigma^*_i, \sigma_{-i})
\end{alignat}

La inecuación~\ref{eq:juego-suma-cero} se obtiene al estar en un juego para dos jugadores de suma cero y la inecuación~\ref{eq:multiplicar-1} se obtiene al multiplicar por menos y cambiar la orientación de la desigualdad.
\end{proof}

\begin{reptheorem}{theo:EN-intercambiabilidad}
Sean $\sigma = (\sigma_1, \sigma_2)$ y $\sigma' = (\sigma'_1, \sigma'_2)$ equilibrios de Nash en un juego de dos jugadores con suma cero. Entonces $\sigma'' = (\sigma_1, \sigma'_2)$ y $\sigma''' = (\sigma'_1, \sigma_2)$ son también equilibrios de Nash. Además, $u_i(\sigma) = u_i(\sigma') = u_i(\sigma'') = u_i(\sigma''')$, para $i \in \{1, 2\}$.
\end{reptheorem}

\begin{proof}

Note que:
\begin{itemize}
    \item $u_i(\sigma_i, \sigma_{-i}) \leq u_i(\sigma_i, \sigma'_{-i})$ ocurre porque $\sigma$ es un equilibrio de Nash y el Teorema~\ref{theo:cota-ganancia-esperada}.
    \item $u_i(\sigma_i, \sigma'_{-i}) \leq u_i(\sigma'_i, \sigma'_{-i})$ ocurre porque $\sigma'$ es un equilibrio de Nash y entonces $\sigma'_i$ es mejor respuesta a $\sigma'_{-i}$.
    \item  $u_i(\sigma'_i, \sigma'_{-i}) \leq u_i(\sigma'_i, \sigma_{-i})$ ocurre porque $\sigma'$ es un equilibrio de Nash y el Teorema~\ref{theo:cota-ganancia-esperada}.
    \item $u_i(\sigma'_i, \sigma_{-i}) \leq u_i(\sigma_i, \sigma_{-i})$ ocurre porque $\sigma$ es un equilibrio de Nash y entonces $\sigma_i$ es mejor respuesta a $\sigma_{-i}$.
\end{itemize}

De lo anterior se obtiene que:
\begin{alignat}{1}
\label{eq:desigualdes-EN}
u_i(\sigma_i, \sigma_{-i})\ \leq\ u_i(\sigma_i, \sigma'_{-i})\ \leq\ u_i(\sigma'_i, \sigma'_{-i})\ \leq\ u_i(\sigma'_i, \sigma_{-i})\ \leq\ u_i(\sigma_i, \sigma_{-i}) \,.
\end{alignat}

% La primera desigualdad, $u_i(\sigma_i, \sigma_{-i}) \leq u_i(\sigma_i, \sigma'_{-i})$, ocurre porque $\sigma$ es un equilibrio de Nash y el Teorema~\ref{theo:cota-ganancia-esperada}. La segunda desigualdad, $u_i(\sigma_i, \sigma'_{-i}) \leq u_i(\sigma'_i, \sigma'_{-i})$, ocurre porque $\sigma'$ es un equilibrio de Nash y entonces $\sigma'_i$ es mejor respuesta a $\sigma'_{-i}$. La tercera desigualdad,  $u_i(\sigma'_i, \sigma'_{-i}) \leq u_i(\sigma'_i, \sigma_{-i})$ ocurre porque $\sigma'$ es un equilibrio de Nash y el Teorema~\ref{theo:cota-ganancia-esperada}. La última desigualdad, $u_i(\sigma'_i, \sigma_{-i}) \leq u_i(\sigma_i, \sigma_{-i})$, ocurre porque $\sigma$ es un equilibrio de Nash y entonces $\sigma_i$ es mejor respuesta a $\sigma_{-i}$.

Luego, todas las desigualdades en \ref{eq:desigualdes-EN} se cumplen como igualdad. Es decir $u_i(\sigma) = u_i(\sigma') = u_i(\sigma'') = u_i(\sigma''')$. Además $u_i(\sigma_i, \sigma'_{-i}) = u_i(\sigma'_i, \sigma'_{-i}) \geq u_i(\sigma''_i. \sigma'_{-i})$ y $u_i(\sigma'_i, \sigma_{-i}) = u_i(\sigma_i, \sigma_{-i}) \geq u_i(\sigma''_i. \sigma_{-i})$, para cualquier estrategia de $\sigma''_i$ del jugador $i$, por lo tanto $\sigma_i$ es mejor respuesta a $\sigma'_{-i}$ y $\sigma'_{i}$ es mejor respuesta a $\sigma_{i}$, para $i = 1, 2$ y por lo tanto $(\sigma'')$ y $\sigma'''$ también son equilibrios de Nash.
\end{proof}

\section*{Capítulo IV}

\begin{reptheorem}{theo:no-regret}
Sea $(s_t)_{t = 1, 2, ...}$ una secuencia de juegos de $\Gamma$.
Entonces, $R_i^t(j, k)$ converge a $0$ para cada $i$ y cada $j, k \in S_i$, con $j \neq k$, si y sólo si la secuencia de distribuciones empíricas $z_t$ converge al conjunto de equilibrio correlacionado.
\end{reptheorem}

\begin{proof}
Note que:
\begin{alignat}{1}
  D_i^t(j, k)\ 
    &=\ \frac{1}{t} \sum_{\substack{1\leq \tau \leq t \\ s_i^{\tau}=j}} u_i(k, s_{-i}^{\tau}) - u_i(s^{\tau}) \\
    &=\ \sum_{ \substack{s \in S \\ s_i = j}} \frac{1}{t} |\{1\leq\tau \leq t : s^{\tau} = s\}|\,[u_i(k, s_{-i}) - u_i(s)] \\
    &=\ \sum_{ \substack{s \in S \\ s_i = j}} z_t(s)\,[u_i(k, s_{-i}) - u_i(s)] \,.
\end{alignat}

Dado $\varepsilon > 0$, $R_i^t(j, k) \leq \varepsilon$ si y sólo si:
\begin{alignat}{1}
    \sum_{s \in S : s_i = j} z_t(s)\,[u_i(k, s_{-i}) - u_i(s)]\  =\ D_i^t(j, k)\ \leq \ \varepsilon \,,
\end{alignat}
obteniendo que $R_i^t(j, k) \leq \varepsilon$ para todo $i \in N$ y todo $j, k \in S_i$ si y sólo si $z_t$ es un $\varepsilon$-equilibrio correlacionado. Por lo tanto, todos los \textit{regrets} convergen a cero si y sólo si $z_t$ converge al conjunto de equilibrio correlacionado.
\end{proof}

\begin{reptheorem}{theo:def-proc-B}
Sea $R_t^i(j, j) = 0$. El vector $q_t^i$, definido en \ref{eq:def-inv-vector}, cumple que:
\begin{alignat}{1}
q^i_t(j)\sum_{k \in S_i} R^i_t(j,k)\ =\ \sum_{k \in S_i} q_t^i(k)R_i^t(k,j) \,.
\end{alignat}
\end{reptheorem}

\begin{proof}
\begin{alignat}{3}
  & & q^i_t(j)\ &=\ \biggl[\sum_{k \in S_i} q^i_t(k)\frac{1}{\mu}R^i_t(k,j)\biggr] + q^i_t(j)\biggl[1 - \sum_{k \in S_i} \frac{1}{\mu} R^i_t(j,k)\biggr] \\
  &\implies\
  &\mu q_t^i(j)\ &=\ \biggl[\sum_{k \in S_i}q^i_t(k)R^i_t(k,j)\biggr] + q^i_t(j)\biggl[\mu - \sum_{k \in S_i} R^i_t(j, k)\biggr] \\
  &\implies\
  &\mu q^i_t(j)\ & =\ \biggl[\sum_{k \in S_i}q^i_t(k)R^i_t(k,j)\biggr] + \mu q^i_t(j) - q^i_t(j)\sum_{k\in S_i} R^i_t(j,k) \,.
\end{alignat}

Luego,
\begin{alignat}{1}
q^i_t(j)\sum_{k \in S_i} R^i_t(j,k)\ =\ \sum_{k \in S_i} q_t^i(k)R_i^t(k,j) \,.
\end{alignat}
\end{proof}

\begin{reptheorem}{theo:conv-proc-B}
Supongamos que a cada período $t+1$, el jugador $i$ elige las estrategias acorde a un vector de distribución de probabilidad $q_t^i$ que satisface \eqref{eq:proc-B}. Entonces, $R^i_t(j, k)$ converge a cero (a. s.) para todo $j, k \in S_i$ con $j \neq k$.
\end{reptheorem}

\begin{proof}
La prueba es una aplicación directa del Teorema de Aproximación de Blackwell (Apéndice~\ref{apex:chapter:blackwell}) con $L$, $v$ y $\mathcal{C}$ definidos de la siguiente manera:
\begin{itemize}[noitemsep]
  \item $L = \{ (j, k) \in S_i \times S_{i}  : j \neq k \}$.
  \item $v(s_i, s_{-i}) \in \mathbb{R}^L$ dado por
    \begin{alignat}{1}
      [v(s_i, s_{-i})](j, k)\ =\  
        \begin{cases}
          u_i(k, s_{-i}) - u_i(j, s_{-i}) & \text{si } s_i = j \\
          0 & \text{en otro caso} \,.
        \end{cases}
    \end{alignat}
  \item $\mathcal{C} = \mathbb{R}^L_{-} = \{x \in \mathbb{R}^L : x_i \leq 0\ \forall i \in L \}$ es decir, el ortante negativo.
\end{itemize}

Se demostrará que $\mathcal{C}$ es alcanzable por $i$.
Note que:
\begin{alignat}{1}
	w_{\mathcal{C}}(\lambda)\ =\ \sup\{\lambda \cdot c : c \in \mathcal{C} \}\ =\ \sup \left \{ \sum_{i \in L} \lambda_i c_i : c_i \leq 0 \right \} \,.
\end{alignat}
Luego, si $\lambda_i \geq 0$, $\forall i \in L$, entonces $\lambda \cdot c \leq 0$ para todo $c \in \mathcal{C}$, y $w_{\mathcal{C}}(\lambda) = 0$. Por otra parte, si $\lambda_i < 0$ para algún $i\in N$, entonces $c_i \lambda_i$ no está acotado superiormente y  $w_{\mathcal{C}}(\lambda) = \infty$. Luego,
\begin{alignat}{1}
  w_{\mathcal{C}}\ =\  
	\begin{cases}
	  0 & \text{si } \lambda \in \mathbb{R}^L_+ \,, \\
	  \infty & \text{en caso contrario.}
	\end{cases}
\end{alignat}

Por otra parte, se tiene que:
\begin{alignat}{1}
	\lambda \cdot v(q_{\lambda}, s_{-i})\ 
	  &=\ \sum_{(j,k) \in L} \lambda(j,k) \cdot [v(q_{\lambda}, s_{-i})](j, k) \\
	&=\ \sum_{(j,k) \in L} \lambda(j, k)\left[\sum_{s_i \in S_i} q_{\lambda}(s_i) v(s_i, s_{-i}) \right](j, k) \\
	&=\ \sum_{(j,k) \in L} \lambda(j, k) q_{\lambda}(j) [v(j, s_{-i})](j, k) \\
	&=\ \sum_{(j,k) \in L} \lambda(j, k) q_{\lambda}(j) [u_i(k, s_{-i}) - u_i(j, s_{-i})] \\
	&=\ \sum_{(j,k) \in L} \lambda(j, k) q_{\lambda}(j)u_i(k, s_{-i}) - \sum_{(j,k) \in L} \lambda(j, k) q_{\lambda}(j)u_i(j, s_{-i}) \\
	&=\ \sum_{k \in S_i} u_i(k, s_{-i}) \sum_{j \in S_i} \lambda(j, k) q_{\lambda}(j) - \sum_{j \in S_i} q_{\lambda}(j)u_i(j, s_{-i}) \sum_{k \in S_i} \lambda(j, k) \\
	&=\ \sum_{j \in S_i} u_i(j, s_{-i}) \sum_{k \in S_i} \lambda(k, j) q_{\lambda}(k) - \sum_{j \in S_i} q_{\lambda}(j)u_i(j, s_{-i}) \sum_{k \in S_i} \lambda(j, k) \\
	&=\ \sum_{j \in S_i} u_i(j, s_{-i}) \left[ \sum_{k \in S_i} \lambda(k, j) q_{\lambda}(k) - q_{\lambda}(j) \sum_{k \in S_i} \lambda(j, k) \right] \,.
\end{alignat}

Se define:
\begin{alignat}{1}
  \alpha(j)\ =\ \sum_{k \in S_i} \lambda(k, j) q_{\lambda}(k) - q_{\lambda}(j) \sum_{k \in S_i} \lambda(j, k) \,,
\end{alignat}
entonces, $\lambda \cdot v(q_{\lambda}, s_{-i}) = \sum_{j \in S_i} u_i(j, s_{-i}) \alpha(j)$. Luego, la condición del Teorema \ref{theo:blackwell} es equivalente a:
\begin{alignat}{1}
	\sum_{j \in S_i} u_i(j, s_{-i}) \alpha(j)\ \leq\ 0 \,.
\end{alignat}

Si se elige $q_{\lambda}$ que cumpla 
\begin{alignat}{1}
  q_{\lambda}(j) \sum_{k \in S_i} \lambda(j, k)\ =\ \sum_{k \in S_i} \lambda(k, j) q_{\lambda}(k)
\end{alignat}
para todo $j \in S_i$, entonces $\alpha(j)=0$ para $j\in S_i$, y la condición del Teorema~\ref{theo:blackwell} se cumple como igualdad cuando $\mathcal{C} = \mathbb{R}^{L}_-$.

Por otra parte, sea $D_t=\frac{1}{t}\sum_{\tau=1}^t v(s_\tau)$ el promedio de los vectores de pago a tiempo $t$. Entonces,
\begin{alignat}{1}
  D_t[j, k]\ &=\ \sum_{\tau=1}^{t} v(s_{\tau})[j,k]\
	=\ \sum_{1\leq\tau \leq t, s_i^{\tau} = j} u_i(k, s_{-i}^{\tau}) - u_i(j, s_{-i}^{\tau})\
	=\ D_i^t(j, k) \,,
\end{alignat}
para $x \notin \mathbb{R}^-$, $F(x) = x^-$ y $\lambda (x) = x - x^- = x^+$, obteniendo 
\begin{alignat}{1}
	\lambda(D_t)\ =\ (R_t^i(j, k))_{(j, k) \in L} \,.
\end{alignat}

Luego, usar una estrategia que cumpla
\begin{alignat}{1}
	q_{\lambda}(j) \sum_{k \in S_i} \lambda(j, k)\ =\  \sum_{k \in S_i} q_{\lambda}(k) \lambda(k, j)
\end{alignat}
cuando $\lambda(j, k) = [D_i^t(j, k)]^+ = R_t^i(j, k)$ es equivalente que la estrategia $p_{t+1}^i \in \Delta(S_i)$ cumpla con
\begin{alignat}{1}
	p_{t+1}^i (j) \sum_{k \in S_i} R_i^t(j, k)\ =\ \sum_{k \in S_i} R_i^t(k, j) p_{t+1}^i(k) \,.
\end{alignat}

Aplicando el Teorema \ref{theo:blackwell} se tiene que al usar dicha estrategia, $D_t$ alcanza a $\mathbb{R}^-$ que es equivalente a que $R_i^t(j, k) \rightarrow 0$ para todo $j, k \in S_i$.
\end{proof}

\begin{reptheorem}{theo:conv-proc-C}
El procedimiento adaptativo definido en \eqref{eq:proc-C} es universalmente consistente para el jugador $i$.
\end{reptheorem}

\begin{proof}
La prueba es similar a la del procedimiento anterior. Se definen $L$, $v$ y $\mathcal{C}$ del Teorema \ref{theo:blackwell} de la siguiente manera:
\begin{itemize}[noitemsep]
  \item $L = S_i$.
  \item $v = v(s_i, s_{-i}) \in \mathbb{R}^L$ dada por:
    $[v(s_i, s_{-i})](k)=u_i(k, s_{-i}) - u_i(s_i, s_{-i})$.
  \item $\mathcal{C} = \mathbb{R}^L_- = \{x \in \mathbb{R}^L : x_i \leq 0\ \forall i \in L \}$ (i.e.\ el ortante negativo).
\end{itemize}

Se demostrará que $\mathcal{C}$ es alcanzable por $i$. Al igual que antes, se tiene que:
\begin{alignat}{1}
  w_{\mathcal{C}}\ =\ 
    \begin{cases}
	  0 & \text{si } \lambda \in \mathbb{R}^L_+ \,, \\
      \infty & \text{en caso contrario.}
    \end{cases}
\end{alignat}

Por otra parte,
\begin{alignat}{1}
	\lambda \cdot v(q_{\lambda}, s_{-i})\ 
	&=\ \sum_{k \in L} \lambda(k) \cdot [v(q_{\lambda}, s_{-i})](k) \\
	&=\ \sum_{k \in S_i} \lambda(k) \cdot \sum_{j \in S_i} q_{\lambda}(j)[v(j, s_{-i})] (k) \\
	&=\ \sum_{k \in S_i} \lambda(k) \cdot \sum_{j \in S_i} q_{\lambda}(j) [u_i(k, s_{-i}) - u_i(j, s_{-i})] \\
	&=\ \sum_{\substack{k \in S_i \\ j \in S_i}} \lambda(k) q_{\lambda}(j) [u_i(k, s_{-i}) - u_i(j, s_{-i})] \\
	&=\ \sum_{\substack{k \in S_i \\ j \in S_i}} \lambda(k) q_{\lambda}(j) u_i(k, s_{-i}) - \sum_{\substack{k \in S_i \\ j \in S_i}} \lambda(k) q_{\lambda}(j) u_i(j, s_{-i}) \\
	&=\ \sum_{\substack{j \in S_i \\ k \in S_i}} u_i(j, s_{-i}) \lambda(j) q_{\lambda}(k)  - \sum_{\substack{j \in S_i \\ j \in S_i}} u_i(j, s_{-i}) \lambda(k) q_{\lambda}(j) \\
	&=\ \sum_{\substack{j \in S_i \\ k \in S_i}} u_i(j, s_{-i}) [\lambda(j) q_{\lambda}(k)  - \lambda(k) q_{\lambda}(j)] \\
	&=\ \sum_{j \in S_i} u_i(j, s_{-i}) \left[ \lambda(j) \sum_{k \in S_i} q_{\lambda}(k)  - q_{\lambda}(j) \sum_{k \in S_i} \lambda(k)  \right] \\
	&=\ \sum_{j \in S_i} u_i(j, s_{-i}) \left[ \lambda(j) - q_{\lambda}(j) \sum_{k \in S_i} \lambda(k) \right] \,.
\end{alignat}

La última igualdad ocurre porque $\sum_{k\in S_i} q_{\lambda}(k)=1$. Luego, si se define:
\begin{alignat}{1}
	\alpha(j)\ =\ \lambda(j) - q_{\lambda}(j) \sum_{k \in S_i} \lambda(k) \,,
\end{alignat}
se obtiene $\lambda \cdot v(q_{\lambda}, s_{-i}) = \sum_{j \in S_i} u_i(j, s_{-i})\alpha(j)$.
Note que si $q_{\lambda(j)} = \frac{\lambda(j)}{\sum_{k \in S_i} \lambda(k)}$, entonces $\alpha(j) = 0$ para todo $j \in S_i$ y se cumple la condición del Teorema \ref{theo:blackwell} en forma de igualdad. Además, para $D_t=\frac{1}{t} \sum_{\tau = 1}^{t} v(s_{\tau})$, se tiene
\begin{alignat}{1}
	D_t[k]\ =\ \sum_{\tau = 1}^{t} v(s_{\tau})[k]\ =\ \sum_{\tau \leq t }[u_i(k, s_{-i}^{\tau}) - u_i(s_{\tau})]\ =\  D_i^t(k) \,.
\end{alignat}

Luego $F(D_t) = D_t^-$ y $\lambda(D_t) = D_t^+ = (R_i^t(k))_{k \in S_i}$, obteniendo:
\begin{alignat}{1}
	q_{\lambda(D_t)}\ =\ \frac{[\lambda(D_t)](j)}{\sum_{k \in S_i}[\lambda(D_t)](k)}\ =\ \frac{R_i^t(j)}{\sum_{k \in S_i} R_i^t(k)} \,.
\end{alignat}

Al elegir $p_{t+1}(j) = q_{\lambda(D_t)}(j) = \frac{R_i^t (j)}{\sum_{k \in S_i} R_i^t(k)}$, se obtiene que $D_t$ alcanza a $\mathbb{R^-}$, lo cual es equivalente a que $R_i^t(j) \rightarrow 0$ para todo $j \in S_i$.
\end{proof}


\begin{reptheorem}{theo:UC-EN}
Sea $\Gamma$ un juego de dos jugadores de suma cero y sea $(s^t)_{t=1,2,..., T}$ una secuencia de juegos de $\Gamma$, tales que, para todo $s_i \in S_i$, para todo $i \in {1, 2}$:
\begin{alignat}{1}
\frac{1}{T}\sum_{t = 1}^{T}u_i(s_i, s_{-i}^t) - \frac{1}{T} \sum_{t = 1}^T u_i(s^t)\ \leq\ \varepsilon
\end{alignat}
para algún $\varepsilon > 0$. Sea $\bar{\sigma}^T = (\bar{\sigma_1}^T, \bar{\sigma_2}^T)$, donde:
\begin{alignat}{1}
\bar{\sigma}_i^T(s_i)\ =\ \frac{|\{ t \leq T : s_i^t = s_i\}|}{T} \,,
\end{alignat}
es decir, $\bar{\sigma}^T$ es la distribución empírica de probabilidad, note que $|\{ t \leq T : s_i^t = s_i\}|$ es igual al número de veces que se eligió $s_i$ hasta el tiempo $T$. Entonces, $\bar{\sigma}^T$ es un $2\varepsilon$-equilibrio de Nash.
\end{reptheorem}

\begin{proof}
Se denotará con $\#(s_i)$ el número de veces que se ha elegido $s_i$ a tiempo $T$, i.e., $\#(s_i) = |\{ t \leq T : s_i^t = s_i\}|$. Por hipótesis del teorema, se tiene que:
\begin{alignat}{1}
\frac{1}{T} \sum_{t = 1}^T u_i(s_i, s_{-i}^t) - \frac{1}{T} \sum_{t = 1}^T u_i(s^t)\ \leq\ \varepsilon \,.
\end{alignat}

Reordenado la sumatoria del primer término y utilizando la definición de $\bar\sigma$, se obtiene:
\begin{alignat}{3}
& \frac{1}{T} \sum_{s_{-i} \in S_{-i}} \#(s_{-i})u_i(s_i, s_{-i}) - \frac{1}{T} \sum_{t = 1}^Tu_i(s^t)\ & \leq\ & \ \varepsilon \\
\implies\ & \sum_{s_{-i} \in S_{-i}} \bar{\sigma}_{-i}^T(s_{-i})u_i(s_i, s_{-i}) - \frac{1}{T} \sum_{t = 1}^T u_i(s^t)\ & \leq\ & \ \varepsilon \,.
\end{alignat}

Sea $\sigma_i \in \Delta(S_i)$ cualquier estrategia del jugador $i$, luego:
\begin{alignat}{4}
& & \sum_{s_i \in S_i} \sigma_i(s_i) \left[ \sum_{s_{-i} \in S_{-i}} \bar{\sigma}_{-i}^T(s_{-i})u_i(s_i, s_{-i}) - \frac{1}{T} \sum_{t = 1}^T u_i(s^t) \right]\ & \leq\  & & \ \sum_{s_i \in S_i} \sigma_i(s_i) \varepsilon \\
& \implies & \sum_{s_i \in S_i} \sum_{s_{-i} \in S_{-i}} \sigma_i(s_i)\bar{\sigma}_{-i}^T(s_{-i}) u_i(s_i, s_{-i}) - \sum_{s_i \in S_i} \sigma_i(s_i)u_i(s^t)\ & \leq\  & & \ \varepsilon \\
& \implies & u_i(\sigma_i, \bar{\sigma}_{-i}^T) - \frac{1}{T} \sum_{t = 1}^T u_i(s^t)\ & \leq\ &  & \ \varepsilon \,.
\end{alignat}

En particular, se tiene que para estrategias cualesquiera $\sigma_1 \in \Delta(S_1)$ y $\sigma_2 \in \Delta(S_2)$
\begin{alignat}{1}
\label{eq:star1}
u_1(\sigma_1, \bar{\sigma}_2^T) - \frac{1}{T} \sum_{t=1}^T u_1(s^t)\ \leq\ \varepsilon \\
u_2(\bar{\sigma}_1^T, \sigma_2) - \frac{1}{T} \sum_{t=1}^T u_2(s^t)\ \leq\ \varepsilon \,.
\end{alignat}

Además, como $\Gamma$ es un juego de suma cero, se tiene que $u_2(\bar{\sigma}_1^T, \sigma_2) = -u_1(\bar{\sigma}_1^T, \sigma_2)$ y $u_2(s^t) = -u_1(s^t)$, luego:
\begin{alignat}{1}
u_2(\bar{\sigma}_1^T, \sigma_2) - \frac{1}{T} \sum_{t=1}^T u_2(s^t)\ =\ -u_1(\bar{\sigma}_1^T, \sigma_2) - \frac{1}{T} \sum_{t=1}^T -u_1(s^t)\ \leq\ \varepsilon \,.
\end{alignat}

En particular, si $\sigma_2 = \bar{\sigma_2}^T$ entonces:
\begin{alignat}{1}
\label{eq:star2}
-u_1(\bar{\sigma}_1^T, \bar{\sigma_2}^T) + \frac{1}{T} \sum_{t=1}^T u_1(s^t)\ \leq\ \varepsilon \,.
\end{alignat}

Al sumar las desigualdades \ref{eq:star1} y \ref{eq:star2} se obtiene que:
\begin{alignat}{2}
& u_1(\sigma_1, \bar{\sigma}_2^T) - u_1(\bar{\sigma}_1^T, \bar{\sigma}_2^T)\ \leq\ 2\varepsilon \\
\implies\ & u_1(\bar{\sigma}^T) + 2\varepsilon\ \geq\ u_1(\sigma_1, \bar{\sigma}_2^T) \,.
\end{alignat}

Análogamente se tiene que $u_2(\bar{\sigma}^T) + 2\varepsilon \geq u_2(\bar{\sigma_1}^T, \sigma_2)$, con lo que se concluye que $\bar{\sigma}^T$ es un $2\varepsilon$-equilibrio de Nash.
\end{proof}

\begin{reptheorem}{theo:a-b-universalmente-consistentes}
En un procedimiento adaptativo de \textit{Regret Matching}, si el \textit{regret} condicional converge a $0$, entonces el procedimiento es universalmente consistente.
\end{reptheorem}

\begin{proof}
Se demostrará, como en el teorema anterior, que el \textit{regret} incondicional tiende a $0$. De la Ecuación~\ref{eq:diferencia-pago} se tiene que:
\begin{alignat}{1}
  D_i^t(j, k)\ =\ \frac{1}{t} \sum_{\substack{1\leq \tau \leq t \\s^\tau_i = j}} u_i(k, s_{-i}^{\tau}) - u_i(s^{\tau})\,
\end{alignat}
si se suman los $D_i^t(j, k)$ sobre $j \in S_i$, se obtiene:
\begin{alignat}{1}
  \sum_{j \in S_i} D_i^t(j, k)\ =\ \sum_{j \in S_i}{\left( \frac{1}{t} \sum_{\substack{1\leq \tau \leq t \\s^\tau_i = j}} u_i(k, s_{-i}^{\tau}) - u_i(s^{\tau}) \right)}\ =\ \frac{1}{t} \sum_{1\leq \tau \leq t} u_i(k, s_{-i}^{\tau}) - u_i(s^{\tau}) \,.
\end{alignat}
De la Ecuación~\ref{eq:diferencia-pago-ri} se obtiene el último miembro de la igualdad es igual a $D_i^t(k)$. Luego:
\begin{alignat}{1}
    \sum_{j \in S_i} D_i^t(j, k)\ =\ D_i^t(k) \,.
\end{alignat}

Por otra parte, utilizando la desigualdad triangular en $\mathbb{R}$, se tiene que
\begin{alignat}{1}
    \sum_{j \in S_i} \max \left\{ 0, D^t_i(j, k)\right\} \geq \max\left\{0, \sum_{j \in S_i} D^t_i(j, k)\right\},
\end{alignat}
al sustituir por las definiciones de $R_i^t(j, k)$ y $R_i^t(k)$ se concluye que:
\begin{alignat}{1}
    \sum_{j \in S_i} R_i^t(j, k) \geq R_i^t(k) \,.
\end{alignat}

Luego como $R_i^t(j, k) \rightarrow 0$ cuando $t \rightarrow \infty$ para todo $j, k \in S_i$, entonces $\sum_{j \in S_i} R_i^t(j, k) \rightarrow 0$ cuando $t \rightarrow \infty$ y por lo tanto $R_i^t(k) \rightarrow 0$ cuando $t \rightarrow \infty$. Luego si el \textit{regret} condicional converge a 0, entonces el \textit{regret} incondicional también converge a $0$ y por lo tanto el procedimiento es universalmente consistente.
\end{proof}

\section*{Capítulo V}

\begin{reptheorem}{theo:regret-nash}
En un juego de $2$ jugadores de suma cero si el \textit{regret} promedio general a tiempo $T$ es menor que $\varepsilon$ entonces $\sigma^{-T}$ es un $2\varepsilon$-equilibrio de Nash
\end{reptheorem}

\begin{proof}
Se probará que la probabilidad de alcanzar $z$ bajo $\bar{\sigma}_i^T$ viene dada por el promedio de alcanzar $z$ en  cada estrategia. Sean $h_1 \sqsubset h_2 \sqsubset h_3 \sqsubset ... \sqsubset h_m \sqsubset z$ todos los prefijos de $z$ correspondientes al jugador $i$, es decir $P(h_k) = i\ \forall k : 1 \leq k \leq m$ y sean $a_1, a_2, ..., a_m$ las acciones correspondientes en $z$ en cada historia respectiva. Luego:
\begin{alignat}{2}
\pi^{\bar{\sigma}_i^T}(z)\ &=\ \prod_{k = 1}^m \bar{\sigma}_i^T(I(h_k))(a_k) \\	&=\ \prod_{k = 1}^m \frac{\sum_{t = 1}^{T}  \pi^{\sigma_i^t}(I(h_k))\sigma^t_i(I(h_k))(a)}{\sum_{t = 1}^T \pi^{\sigma_i^t}(I(h_k))}
\end{alignat}

Por otra parte, note que $\pi^{\sigma_i^t}(I)\sigma_i^t(I(h_k))(a_k) = \pi^{\sigma_i^t}(I(h_{k+1}))$. Entonces:
\begin{alignat}{1}
	\pi^{\bar{\sigma}_i^T}(z)\ &=\ \frac{\sum_{t = 1}^T\pi^{\sigma_i^t}(I_m) \sigma_i^t(I_m)(a_m)}{\sum_{t = 1}^T \pi^{\sigma_i^t}(I_1)} \\
	&=\ \frac{\sum_{t = 1}^T \pi^{\sigma_i^t}(z)}{\sum_{t = 1}^T 1} \\
	&=\ \frac{1}{T} \sum_{t = 1}^T \pi^{\sigma_i^t}(z) \,.
\end{alignat}

Además, se tiene que, para cualquier jugador $i$ y cualquier estrategia de $\sigma_i$:
\begin{alignat}{2}
\frac{1}{T} \sum_{t = 1}^T u_i(\sigma'_i, \sigma_{-i}^t)\ &=\ \frac{1}{T} \sum_{t = 1}^T \left( \sum_{z \in Z} \pi^{\sigma'_i}(z) \pi^{\sigma_{-i}^t}(z) \pi^c(z) \right) \\
	&=\ \sum_{z \in Z} u_i(z) \pi^{\sigma'_i}(z) \pi^c(z) \left( \frac{1}{T} \sum_{t = 1}^T \pi^{\sigma_{-i}^t}(z) \right) \\
	&=\ \sum_{z \in Z} u_i(z) \pi^{\sigma'_i}(z) \pi^{\bar{\sigma}_i^T}(z) \pi^c(z)\ =\ u_i(\sigma'_i, \bar{\sigma}_{-i}^T) \,.
\end{alignat}

Por otra parte, como $R_2^T \leq \varepsilon$, para todo $\sigma'_2 \in B_2$ se tiene que:
\begin{alignat}{2}
	& \frac{1}{T} \sum_{t = 1}^T[u_2(\sigma_1^t, \sigma'_2) - u_2(\sigma^t)]\  \leq\  \varepsilon\\
	\implies\ & \frac{1}{T} \sum_{t = 1}^T u_2(\sigma^t) + \varepsilon\  \geq\  \frac{1}{T} \sum_{t = 1}^T u_2(\sigma_1^t, \sigma'_2) \\
	\implies\ & \frac{1}{T} \sum_{t = 1}^T u_2(\sigma^t) + \varepsilon\ \geq\ u_2(\bar{\sigma}_1^T, \sigma'_2) \,.
\end{alignat}

Como se cumple para cualquier estrategia $\sigma'_2$, se cumple para $\sigma_2^t$ para $t = 1, 2, ..., T$, obteniendo:
\begin{alignat}{2}
\frac{1}{T} \sum_{t = 1}^T u_2(\sigma^t) + \varepsilon\ & \geq\ \frac{1}{T} \sum_{t = 1}^T u_2(\bar{\sigma}_1^T, \sigma_2^t) \\
&=\ \frac{1}{T} \sum_{t = 1}^T \sum_{z \in Z} u_2(z) \pi^{\bar{\sigma}_1^T}\pi^{\sigma_2^t}(z) \pi^c(z) \\
&=\  \sum_{z \in Z} u_2(z) \pi^{\bar{\sigma}_1^T} \pi^c(z) \left( \frac{1}{T} \sum_{t = 1}^T \pi^{\sigma_2^t}(z) \right)\\
&=\  \sum_{z \in Z} u_2(z) \pi^{\bar{\sigma}_1^T} \pi^{\bar{\sigma}_2^T} \pi^c(z) \\
&=\ u_2(\bar{\sigma}^T) \,.
\end{alignat}

Como $\Gamma$ es un juego de suma cero, se tiene que $u_2(\sigma) = -u_1(\sigma)$ para cualquier estrategia $\sigma$, luego:
\begin{alignat}{2}
	& \frac{1}{T} \sum_{t = 1}^T u_2(\sigma^t) + \varepsilon\ \geq\  u_2(\bar{\sigma}^T) \\
	\implies\ & \frac{1}{T} \sum_{t = 1}^T -u_1(\sigma^t) + \varepsilon\ \geq\  -u_1(\bar{\sigma}^T) \\
	\implies\ &  u_1(\bar{\sigma}^T) + \varepsilon\ \geq\   \frac{1}{T} \sum_{t = 1}^T u_1(\sigma^t)  \\
	\implies\ &  u_1(\bar{\sigma}^T) + 2\varepsilon\ \geq\   \frac{1}{T} \sum_{t = 1}^T u_1(\sigma^t) + \varepsilon
\end{alignat}

Por otra parte, como $R_i^t \leq \varepsilon$ se tiene que, para cualquier $\sigma'_1 \in B_1$:
\begin{alignat}{1}
	\frac{1}{T} \sum_{t = 1}^T u_1(\sigma^t) + \varepsilon\ \geq\ \frac{1}{T} \sum_{t = 1} u_1(\sigma'_1, \sigma_2^t)\ =\ u_1(\sigma'_1, \bar{\sigma}_2^T) \,.
\end{alignat}

Luego, se obtiene que:
\begin{alignat}{1}
	& u_1(\bar{\sigma}^T) + 2\varepsilon\ \geq\   \frac{1}{T} \sum_{t = 1}^T u_1(\sigma^t) + \varepsilon\ \geq\ u_1(\sigma'_1, \bar{\sigma}_2^T)\\
	\implies\ & u_1(\bar{\sigma}^T) + 2\varepsilon\ \geq\ u_1(\sigma'_1, \bar{\sigma}_2^T) \,.
\end{alignat}

Análogamente, se demuestra que $u_2(\bar{\sigma}^T)\ + 2\varepsilon\ \geq\ u_2( \bar{\sigma}_1^T, \sigma'_2)$ concluyendo que $\bar{\sigma}^T$ es un $2\varepsilon$-equilibrio de Nash.
\end{proof}

\begin{reptheorem}{theo:esperanza-MCCFR}
$E_{j \sim q_j} [\tilde{u}_i(\sigma, I | j)]\ =\ u_i(\sigma, I)$
\end{reptheorem}

\begin{proof}
\begin{alignat}{2}
    E_{j \sim q_j}[\tilde{u}_i(\sigma, I | j)]\ & =\ \sum_{j} {q_j u_i(\sigma, I)} \\
    & =\ \sum_{j} { q_j \frac{\sum_{h \in I, z \in Q_j} \pi^{\sigma_{-i}}(h) \pi^{\sigma}(h, z) u_i(z)}{q(z) \pi^{\sigma_{-i}}(I)}} \\ \label{eq:def-esperanza}
    & =\ \sum_{j} \sum_{ \substack{h \in I \\ z \in Q_j}} \frac{q_j \pi^{\sigma_{-i}}(h) \pi^{\sigma}(h, z) u_i(z)}{ q(z) \pi^{\sigma_{-i}}(I)} \\ \label{eq:reorder-sum-1}
    & =\ \sum_{ \substack{h \in I \\ z \in Z} } \sum_{j | z \in Q_j} \frac{q_j \pi^{\sigma_{-i}}(h) \pi^{\sigma}(h, z) u_i(z)}{ q(z) \pi^{\sigma_{-i}}(I)} \\ \label{eq:reorder-sum-2}
    & =\ \sum_{ \substack{h \in I \\ z \in Z} } \left(\frac{\sum_{j | z \in Q_j} q_j }{q(z)}\right) \frac{\pi^{\sigma_{-i}}(h) \pi^{\sigma}(h, z) u_i(z)}{\pi^{\sigma_{-i}}(I)} \\
    & =\ \sum_{ \substack{h \in I \\ z \in Z} } \frac{\pi^{\sigma_{-i}}(h) \pi^{\sigma}(h, z) u_i(z)}{\pi^{\sigma_{-i}}(I)}  = u_i(\sigma, I) \label{eq:lemma-final-eq}
\end{alignat}
La ecuación \ref{eq:def-esperanza} se obtiene de la definición de $\tilde{u}_i(\sigma, I | j)$. \ref{eq:reorder-sum-1} y \ref{eq:reorder-sum-2} se obtienen al reordenar las sumatorias y considerando que la unión de los bloques generan a $Z$. La ecuación \ref{eq:lemma-final-eq} es la definición de utilidad contrafactual.
\end{proof}

\chapter{Teorema de Aproximación de Blackwell}
\label{apex:chapter:blackwell}

Los procedimientos que calculan equilibrios correlacionados se basan en el método de aproximación de Blackwell \cite{bib:correlated-equilibrium}.

El marco teórico en el cual se aplica el teorema está conformado por: (1)~un \textbf{decididor} $i$ que toma decisiones de un conjunto finito de acciones $S_i$, (2)~un \textbf{oponente} $-i$ que toma decisiones de un conjunto finito de acciones $S_{-i}$, (3)~un \textbf{conjunto indexado} denotado por $L$, y (4)~un \textbf{vector de pagos} $v(s_i, s_{-i}) \in \mathbb{R}^{|L|}$.
El decididor y oponente toman decisiones $s_t=(s^t_i,s^t_{-i})\in S_i\times S_{-i}$ indexadas en tiempo $t\geq 1$. El problema planteado consiste en ver si el decididor puede garantizar que el promedio de pagos $D_t$ a tiempo $t$, definido por
\begin{alignat}{1}
D_t\ =\ \frac{1}{t}\sum_{\tau=1}^t v(s_\tau)\ =\ \frac{1}{t}\sum_{\tau=1}^t v(s^\tau_i,s^\tau_{-i})
\end{alignat}
\emph{alcanza} el conjunto $\mathbb{R}^{|L|}$. Antes de enunciar el teorema es necesario presentar las definiciones de distancia de un punto a un conjunto (Definición \ref{def:distancia}), un conjunto alcanzable (Definición \ref{def:alcanzable}), y de función de soporte (Definición \ref{def:funcion-soporte}).

\begin{definition}
\label{def:distancia}
Sea $A$ un conjunto cerrado y convexo en $\mathbb{R}^n$, y $x \in \mathbb{R}^n$ un punto cualquiera. La \textbf{distancia} de $x$ a $A$ es definida por
\begin{alignat}{1}
\text{dist}(x, A)\ =\ \min\{ \|x - a\| : a \in A \}
\end{alignat}
donde $\|\cdot\|$ denota la distancia euclidiana en $\mathbb{R}^n$.
\end{definition}

\begin{definition}
\label{def:alcanzable}
Sea $\mathcal{C}$ un conjunto convexo y cerrado en $\mathbb{R}^{|L|}$. El conjunto $\mathcal{C}$ es \textbf{alcanzable} por el decididor $i$ si hay un procedimiento para $i$ que garantiza que $D_t$ alcanza a $\mathcal{C}$; es decir. $dist(D_t, \mathcal{C}) \rightarrow 0$ (a.s.) sin importar la elección del oponente $-i$.
\end{definition}

\begin{definition}
\label{def:funcion-soporte}
Sea $\mathcal{C} \in \mathbb{R}^n$ un conjunto. La \textbf{función de soporte} $w_{\mathcal{C}}$ para el conjunto $\mathcal{C}$, es definida por
\begin{alignat}{1}
	w_{\mathcal{C}}(\lambda)\ =\ \sup\{\lambda \cdot c : c \in \mathcal{C} \}
\end{alignat}
donde $\cdot$ denota el producto interno en $\mathbb{R}^n$.
\end{definition}

Dado un conjunto convexo y cerrado $\mathcal{C}$ denotaremos con $F(x)$ el punto (único) más cercano a $x$ de $C$, y con $\lambda(x)= x - F(x)$.
El Teorema de Aproximación de Blackwell establece una condición necesaria y suficiente para el problema planteado previamente.

\begin{theorem}[Aproximación de Blackwell]
\label{theo:blackwell}
Sea $\mathcal{C} \subseteq \mathbb{R}^{|L|}$ un conjunto convexo y cerrado con función de soporte $w_{\mathcal{C}}$. Entonces, $\mathcal{C}$ es alcanzable por $i$ si y sólo si para todo $\lambda \in \mathbb{R}^{|L|}$, existe una estrategia mixta $q_{\lambda} \in \Delta(S_i)$ para el decididor $i$ tal que para todo $s_{-i}\in S_{-i}$:
\begin{alignat}{1}
  \lambda \cdot v(q_{\lambda}, s_{-i})\ \leq\ w_{\mathcal{C}}(\lambda) \,.
\end{alignat}
En esta expresión, $v(q, s_{-i})$ denota $\sum_{s_i \in S_i} q(s_i)u_i(s_i, s_{-i})$. 
Además, el siguiente procedimiento garantiza que $dist(D_t, \mathcal{C}) \rightarrow 0$ (a.s.) cuando $t \rightarrow \infty$: en el tiempo $t+1$, jugar $q_{\lambda(D_t)}$ si $D_t \notin \mathcal{C}$, y jugar arbitrariamente si $D_t \in \mathcal{C}$.
\end{theorem}

%\Blai{**** Algún ejemplo? ****}

\noindent\textcolor{red}{\bf *** Hasta aquí tenemos la descripción teórica del modelo y soluciones. Ahora vienen algoritmos. Comenzar un nueva capítulo. ***}

\chapter{Estrategias Minimax y Maximin}
\label{apex:chapter:estrategias}

Una estrategia \textit{minimax} del jugador $i$, consiste en minimizar la ganancia de la mejor respuesta del jugador $-i$. Es decir, el jugador $i$ juega para \say{castigar} al jugador $-i$, sin tomar en cuenta su propia ganancia. Por otra parte en una estrategia \textit{maximin}, el jugador busca maximizar su ganancia, suponiendo que su oponente juega para perjudicarlo.

\begin{definition}[{\cite[p.~15--16]{bib:handbook-blai}}]
El conjunto de estrategias \textit{minimax} para el jugador $i$ en contra del jugador $-i$ es
\begin{alignat}{1}
\{ \sigma_i : \max_{\sigma_{-i}}u_{-i}(\sigma_i, \sigma_{-i}) = \min_{\sigma'_i}{\max_{\sigma_{-i}} u_{-i}(\sigma'_i, \sigma_{-i})} \}\,,
\end{alignat}
y el valor minimax del jugador $-i$ es $\min_{\sigma_i}{\max_{\sigma_{-i}}{u_{-i}(\sigma_i, \sigma_{-i})}}$.
El conjunto de estrategias \textit{maximin} para el jugador $i$ en contra del jugador $-i$ es
\begin{alignat}{1}
\{ \sigma_i : \min_{\sigma_{-i}}u_i(\sigma_i, \sigma_{-i}) = \max_{\sigma'_i}{\min_{\sigma_{-i}} u_i(\sigma'_i, \sigma_{-i})} \}\,,
\end{alignat}
y el valor \textit{maximin} del jugador $i$ es $\max_{\sigma_i}{\min_{\sigma_{-i}}{u_i(\sigma_i, \sigma_{-i})}}$.
\end{definition}

Como la estrategia \textit{minimax} o \textit{maximin} de un jugador no depende de la estrategia del oponente, se pueden definir perfiles estratégicos \textit{minimax} y \textit{maximin}. Un perfil estratégico mixto $\sigma = (\sigma_1, \sigma_2)$ es un perfil estratégico \textit{minimax} (\textit{maximin}) si $\sigma_1$ es un estrategia minimax (resp.\ \textit{maximin}) para el jugador $1$ y $\sigma_2$ es una estrategia \textit{minimax} (resp.\ \textit{maximin}) para el jugador $2$.

\begin{example}
\label{ex:ejemplos-min-max}
Considere el juego en forma normal de $2$ jugadores mostrado en la Tabla~\ref{table:ejemplos-min-max} donde $S_1 = S_2 = \{1, 2\}$.
\end{example}

Calculemos estrategias \textit{minimax} y \textit{maximin} para el primer jugador. Las estrategias \textit{minimax} del primer jugador vienen expresadas por:
\begin{alignat}{1}
\argmin_{(\beta_1, \beta_2) \in \Delta_2 }\ {\max_{(\theta_1, \theta_2) \in \Delta_2}
{\theta_1(4\beta_1 -\beta_2) + \theta_2(-2\beta_1 + 2\beta_2)}} \,.
\end{alignat}

Note que si $4\beta_1 - \beta_2 = x$ y $-2 \beta_1 + 2\beta_2 = y$, entonces:
\begin{alignat}{1}
\max_{(\theta_1, \theta_2) \in \Delta_2} \theta_1(4\beta_1 -\beta_2) + \theta_2(-2\beta_1 + 2\beta_2) = \max_{(\theta_1, \theta_2) \in \Delta_2} \theta_1 x + \theta_2 y = \max\{x, y\} \,.
\end{alignat}
Se puede demostrar que la estrategia \textit{minimax} ocurre cuando $x = y$, pues en este caso la ganancia del segundo jugador no depende de las elecciones de $\theta_1$ y $\theta_2$.

Esto último ocurre cuando $4\beta_1 - \beta_2 = -2\beta_1 + 2\beta_2$, lo que implica que $(\beta_1, \beta_2) = \left(\frac{1}{3}, \frac{2}{3} \right)$.  El valor \textit{minimax} del segundo jugador es entonces $\frac{2}{3}$. En este caso, el primer jugador elige su estrategia considerando, únicamente, la ganancia del oponente, sin tomar en cuenta su propia ganancia.

Por otra parte, las estrategias \textit{maximin} se corresponden con
\begin{alignat}{1}
\argmax_{(\beta_1, \beta_2) \in \Delta_2 }\ {\min_{(\theta_1, \theta_2) \in \Delta_2}
{\theta_1(2\beta_1 -\beta_2) + \theta_2(-\beta_1 + 2\beta_2)}} \,.
\end{alignat}

Análogamente, se puede probar que la estrategia \textit{maximin} es alcanzada cuando la ganancia esperada del primer jugador no depende de la elección de $\theta_1$ y $\theta_2$, es decir cuando $2\beta_1 - \beta_2 = -\beta_1 + 2\beta_2$, lo que ocurre si y sólo si $(\beta_1, \beta_2) = \left(\frac{1}{2}, \frac{1}{2}\right)$. El valor \textit{maximin} del primer jugador es $\frac{1}{2}$.

\begin{table}[t]
\begin{center}
\caption{Tabla de pagos del juego del Ejemplo~\ref{ex:ejemplos-min-max}.}
\label{table:ejemplos-min-max}
\begin{tabular}{ r | c | c |}
 \multicolumn{1}{c}{} & \multicolumn{1}{c}{1} & \multicolumn{1}{c}{2}  \\ \cline{2-3}
 1 & $2, 4$ & $-1, -2$ \\ \cline{2-3}
 2 & $-1, -1$ & $2, 2$ \\ \cline{2-3}
\end{tabular}
\end{center}
\end{table}

\begin{theorem}[\cite{bib:von-neumann}]
\label{theo:von-neumann}
Para cualquier juego finito para dos jugadores de suma cero, y para cualquier equilibrio de Nash del juego, cada jugador tiene una ganancia esperada cuyo valor es igual al valor minimax y valor maximin de dicho jugador. \end{theorem}

El Teorema~\ref{theo:von-neumann} muestra que las estrategias \textit{minimax}, \textit{maximin} y los equilibrios de Nash coinciden en los juegos de dos jugadores con suma cero.
\chapter{Forma Normal y Programación Lineal}
\label{apex:chapter:programacion-lineal}

En el Capítulo \ref{chapter:regret-matching} se afirma que para todo juego de dos jugadores de suma cero en forma normal, encontrar el equilibrio de Nash es equivalente a un problema de programación lineal. En esta sección se detalla cómo obtener el programa de programación lineal a un juego dado en forma normal.

Todo juego en forma normal puede ser descrito por una tabla $N$ dimensional, donde $N$ es el número de jugadores. En particular un juego de dos jugadores puede ser representado por una matriz, y además, si el juego es de suma $0$, basta con definir la utilidad del primer jugador. Luego, este tipo de juegos pueden ser representados con una matriz $\mathbf{A} = (a_{ij}) \in \mathbb{R}^{m \times n}$. Luego, el elemento $a_{ij}$ de la matriz $A$ es la utilidad obtenida por el primer jugador si éste utiliza la $i$-ésima estrategia y su oponente utiliza la $j$-ésima estrategia.

Si el jugador $1$ juega con una estrategia $ \mathbf{x} = (x_1, x_2, ..., x_m))$ y el jugador $2$ con una estrategia $\mathbf{y} = (y_1, y_2, ..., y_n)$, entonces la ganancia esperada viene dada por
\begin{alignat}{1}
u_1(\mathbf{x}, \mathbf{y}) = \sum_{i = 1}^m \sum_{j = 1}^n a_{ij}x_iy_j,
\end{alignat}
lo cual se puede escribir matricialmente como $\mathbf{x}^\intercal \mathbf{A} \mathbf{y}$.

Del Teorema \ref{theo:von-neumann} se sabe que el valor \textit{maximin} y \textit{minimax} es igual al valor del juego, al igual que los perfiles estratégicos \textit{maximin} y \textit{minimax} coinciden con equilibrios de Nash. Luego, un equilibrio de Nash $(x^*, y^*)$ cumple que:
\begin{alignat}{1}
x^* \in \argmax_{\mathbf{x}}{\min_{\mathbf{y}}{\mathbf{x}^\intercal \mathbf{A} \mathbf{y}}} \ \text{y} \label{apex:eq:maximin} \\
y^* \in \argmin_{\mathbf{y}}{\max_{\mathbf{x}}{\mathbf{x}^\intercal \mathbf{A} \mathbf{y}}} \,.
\label{apex:eq:minimax}
\end{alignat}

Ya que la ecuación \ref{apex:eq:maximin} es la definición de estrategia \textit{maximin} para el primer jugador y la ecuación \ref{apex:eq:minimax} es la definición \textit{minimax} para el segundo jugador. La primera observación importante proviene del Teorema~\ref{theo:mejor-respuesta}, ya que como corolario se obtiene que, para cualquier estrategia, siempre existe una mejor respuesta cuyo soporte tiene un único elemento. De esto último se obtiene que, dado $\mathbf{x} \in \mathbb{R}^m$ e $\mathbf{y} \in \mathbb{R}^n$:
\begin{alignat}{1}
\min_{\mathbf{y}}{\mathbf{x}^\intercal \mathbf{A} \mathbf{y}} = \min_{j} \sum_{i=1}^n a_{ij} x_i\\
\max_{\mathbf{x}}{\mathbf{x}^\intercal \mathbf{A} \mathbf{y}} = \max_{i} \sum_{i=1}^m a_{ij} y_j
\end{alignat}

Luego, los problemas que se quieren resolver son equivalentes a:
\begin{alignat}{1}
\max_{x} \min_{j} \sum_{i=1}^n a_{ij} x_i \label{apex:eq:maximin-problema} \\
\min_{y} \max_{i} \sum_{i=1}^m a_{ij} y_j \label{apex:eq:minimax-problema}
\end{alignat}

Sujetos a $x_i, y_j \geq 0$ y a $\sum_{i=1}^m x_i = 1$ y $\sum_{j=1}^n y_j = 1$. La observación clave consiste en ver la equivalencia de los problemas \ref{apex:eq:maximin-problema} y \ref{apex:eq:minimax-problema} con los problemas de programación lineal \ref{apex:eq:maximin-problema2} y \ref{apex:eq:minimax-problema2}, respectivamente \cite[p.~232]{bib:pl-chvatal}.
\begin{alignat}{6}
\label{apex:eq:maximin-problema2}
& \max\  & z\ &  & & & &  \\ \nonumber
& \text{sujeto a}\ & & & & & & \\  \nonumber 
& & z\ & -\ & \sum_{i=1}^m a_{ij} x_i \ \leq\  \ 0 \ & \ \ (j=1, 2, ..., n) \\\nonumber
& &    &    & \sum_{i=1}^m x_i\ \ =\           \ 1 \ & \\ \nonumber
& &    &    & x_i  \ \geq\                     \ 0 \ & \ (i=1, 2, ..., m) \\
\label{apex:eq:minimax-problema2} 
& \min\  & w\ &  & & & &  \\ \nonumber
& \text{sujeto a}\ & & & & & & \\  \nonumber 
& & w\ & -\ & \sum_{j=1}^n a_{ij} y_j \ \geq\  \ 0 \ & \ \ (i=1, 2, ..., m) \\\nonumber
& &    &    & \sum_{j=1}^n y_j\ \ =\           \ 1 \ & \\ \nonumber
& &    &    & x_i  \ \geq\                     \ 0 \ & \ \ (j=1, 2, ..., n) \,. \nonumber
\end{alignat}

Para ver la equivalencia, note que cualquier solución óptima $z^*, x_1^*, x_2^*, ..., x_m^*$ de \ref{apex:eq:maximin-problema2} satisface al menos una restricción con el signo de igualdad y por lo tanto $z^* = \min_j \sum_i a_{ij}x_i^*$. De forma similar, cualquier solución óptima $w^*, y_1^*, y_2^*, ..., y_n^*$ de \ref{apex:eq:minimax-problema2} satisface al menos una restricción con el signo de igualdad y por lo tanto $w^* = \min_i \sum_j a_{ij}y_j$. Por último, note que \ref{apex:eq:maximin-problema2} y \ref{apex:eq:minimax-problema2} son problemas duales entre sí, lo cual confirma que $z^* = w^*$.

A continuación, se presentan loa problemas de programación lineal asociados a los juegos \textit{matching pennies}, piedra, papel o tijera y ficha vs. dominó; presentados en el Capítulo~\ref{chapter:regret-matching}, con una solución primal y dual.

\begin{itemize}

\item \textbf{\textit{Matching Pennies}}
\begin{alignat}{7}
\label{apex:eq:pl-matching-pennies}
\max\ & z & & & &\\ \nonumber
\text{sujeto a}\ & & & & &\\ \nonumber
    &    &    & x_1\ & +\ & x_2\  =\    1 \\ \nonumber
    & z\ & -\ & x_1\ & +\ & x_2\  \leq\ 0 \\ \nonumber
    & z\ & +\ & x_1\ & -\ & x_2\  \leq\ 0 \\ \nonumber
    &    &    & x_1, &    & x_2\ \geq\ 0 \,. \nonumber
\end{alignat}

La única solución es:
\begin{alignat}{1}
(z^*, x_1^*, x_2^*)\ =\ (w^*, y^*_1, y^*_2)\ =\ \left(0, \frac{1}{2}, \frac{1}{2} \right) \,.
\end{alignat}

\item \textbf{Piedra, papel o tijera}
\begin{alignat}{9}
\label{apex:eq:pl-RPS}
\max\  & z\ &  & & & & & \\ \nonumber
\text{sujeto a}\ & & & & & & & \\  \nonumber 
&    &    & x_1\ & +\ & x_2\ & +\ & x_3\ & \ =\     & \ 1 \\ \nonumber
& z\ &    &      & +\ & x_2\ & -\ & x_3\ & \ \leq\  & \ 0 \\ \nonumber
& z\ & -\ & x_1\ &    &      & +\ & x_3\ & \ \leq\  & \ 0 \\ \nonumber
& z\ & +\ & x_1\ & -\ & x_2\ &    &      & \ \leq\  & \ 0 \\ \nonumber
&    &    & x_1, &    & x_2, &    & x_3\ & \ \geq\  & \ 0  \,.  \nonumber
\end{alignat}

La única solución es:
\begin{alignat}{1}
(z^*, x^*_1, x^*_2, x^*_4, x^*_5, x^*_6, x^*_6, x^*_7)\ =\ \left(\frac{1}{3}, \frac{1}{3}, \frac{1}{3}, 0, 0, 0, 0, \frac{1}{3}\right) \,, \\
(w^*, y^*_1, y^*_2, y^*_3,  y^*_4, y^*_5, y^*_6)\ =\ \left(\frac{1}{3}, \frac{1}{3}, 0, \frac{1}{3}, 0, \frac{1}{3}, 0\right) \,.
\end{alignat}

\item \textbf{Ficha vs. dominó}
\begin{alignat}{10}
\label{apex:eq:pl-domino}
\max\ &z\ & & & & & & & & & & & & & & & & \\ \nonumber
\text{sujeto a}\ & & & & & & & & & & & & & & & & \\ \nonumber 
	&   &   & x_1\ & +\ & x_2\ & +\ & x_3\ & +\ & x_4\ & +\ & x_5\ & +\ & x_6\ & +\ & x_7\ & \ =\    & \ 1 \\ \nonumber
    &z\ &+\ & x_1\ & -\ & x_2\ & -\ & x_3\ & -\ & x_4\ & +\ & x_5\ & -\ & x_6\ & -\ & x_7\ & \ \leq\ & \ 0 \\ \nonumber
    &z\ &+\ & x_1\ & -\ & x_2\ & +\ & x_3\ & -\ & x_4\ & -\ & x_5\ & +\ & x_6\ & -\ & x_7\ & \ \leq\ & \ 0 \\ \nonumber
    &z\ &-\ & x_1\ & -\ & x_2\ & +\ & x_3\ & -\ & x_4\ & -\ & x_5\ & -\ & x_6\ & +\ & x_7\ & \ \leq\ & \ 0 \\ \nonumber
    &z\ &-\ & x_1\ & +\ & x_2\ & -\ & x_3\ & -\ & x_4\ & +\ & x_5\ & -\ & x_6\ & -\ & x_7\ & \ \leq\ & \ 0 \\ \nonumber
    &z\ &-\ & x_1\ & +\ & x_2\ & -\ & x_3\ & +\ & x_4\ & -\ & x_5\ & +\ & x_6\ & -\ & x_7\ & \ \leq\ & \ 0 \\ \nonumber
    &z\ &-\ & x_1\ & -\ & x_2\ & -\ & x_3\ & +\ & x_4\ & -\ & x_5\ & -\ & x_6\ & +\ & x_7\ & \ \leq\ & \ 0 \\ \nonumber
    &   &   & x_1, &    & x_2, &    & x_3, &    & x_4, &    & x_5, &    &x_6,  &    & x_7, & \ \geq\ & \ 0 \,.
\end{alignat}

Una de las soluciones es:
\begin{alignat}{1}
(z^*, x^*_1, x^*_2, x^*_4, x^*_5, x^*_6, x^*_6, x^*_7)\ =\ \left(\frac{1}{3}, \frac{1}{3}, \frac{1}{3}, 0, 0, 0, 0, \frac{1}{3}\right) \,, \\
(w^*, y^*_1, y^*_2, y^*_3,  y^*_4, y^*_5, y^*_6)\ =\ \left(\frac{1}{3}, \frac{1}{3}, 0, \frac{1}{3}, 0, \frac{1}{3}, 0\right) \,.
\end{alignat}
\end{itemize}
\chapter{Algoritmos}
\label{apex:chapter:algoritmos}
\algnewcommand{\algorithmicand}{\textbf{ and }}
\algnewcommand{\algorithmicor}{\textbf{ or }}
\algnewcommand{\OR}{\algorithmicor}
\algnewcommand{\AND}{\algorithmicand}

%


\section*{Chance-sampled Counterfactual Regret Minimization (CFR)}

En esta sección se presenta el algoritmo \textit{chance-sampled} como es descrito en \cite{bib:introductionCFR}. Cada iteración se realiza mediante búsqueda en profundidad, seleccionando una única acción en un nodo de azar de acuerdo a su distribución de probabilidad correspondiente. En cada conjunto de información visitado en una iteración de entrenamiento, una estrategia mixta es calculada de acuerdo a la Ecuación~\ref{eq:cfr-regret-matching} (Algoritmo~\ref{algorithm:CFR}). La estrategia promedio $\sigma^T$, en cada conjunto de información $I$, aproxima a un equilibrio de Nash cuando $T \rightarrow \infty$ (Algortimo~\ref{algorithm:CFR-training}).

\vspace{12pt}
\begin{algorithm}[H]
\caption{Entrenamiento de \textit{chance-sampled} CFR}
\label{algorithm:CFR-training}
\begin{algorithmic}[1]
    \State Inicializar tablas acumulativas de \textit{regret}: $\forall I, r_I[a] \leftarrow 0$.
    \State Inicializar tablas acumulativas de estrategias: $\forall I, s_I[a] \leftarrow 0$.
    \State Inicializar perfil inicial: $\sigma^1(I, a) \leftarrow 1/|A(I)|$ \label{x}
    \State
    \Function{Solve}{}
        \For{$t = {1, 2, ..., T}$}
            \For{$i \in \{1, 2\}$}
                \State CFR($\emptyset$, $i$, $t$, $1$, $1$)
            \EndFor
        \EndFor
    \EndFunction
    \State Calcular estrategia promedio $\bar\sigma^T$ de las estrategias $\sigma^1, \sigma^2, ..., \sigma^T$.
\end{algorithmic}
\end{algorithm}

\newpage
~\vspace{1.5cm}
\begin{algorithm}[H]
\caption{\textit{Counterfactual Regret Minimization} (CFR) con \textit{chance-sampled}}
\label{algorithm:CFR}
\begin{algorithmic}[1]
    \Function{CFR}{$h$, $i$, $t$, $\pi_1$, $\pi_2$}
        \If{$h$ es terminal}
            \State \Return $u_i(h)$
        \ElsIf{$h$ es un nodo de azar}
            \State Seleccionar una acción $a \sim f_c(h)$
            \State \Return CFR($ha$, $i$, $t$, $\pi_1$, $\pi_2$)
        \EndIf
        \State Sea $I$ el conjunto de información que contiene a $h$.
        \State $v_{\sigma} \leftarrow 0$
        \State $v_{\sigma_{I \rightarrow [a]}} \leftarrow 0 $ para todo $a \in A(I)$
        \For{$a \in A(I)$}
            \If{P(h) = 1}
                \State $v_{\sigma_{I \rightarrow [a]}} \leftarrow$ CFR($ha$, $i$, $t$, $\sigma^t(I, a) \cdot \pi_1$, $\pi_2$)
            \ElsIf{P(h) = 2}
                \State $v_{\sigma_{I \rightarrow [a]}} \leftarrow$ CFR($ha$, $i$, $t$, $\pi_1$, $\sigma^t(I, a) \cdot \pi_2$)
            \EndIf
            \State $v_{\sigma} \leftarrow v_{\sigma} + \sigma^t(I, a) \cdot v_{\sigma_{I \rightarrow [a]}}$
        \EndFor
        \If{$P(h) = i$}
            \For{$a \in A(I)$}
                \State $r_I[a] \leftarrow r_I[a] + \pi_{-i} \cdot (v_{\sigma_{I \rightarrow [a]}} - v_{\sigma})$
                \State $s_{I}[a] \leftarrow s[I][a] + \pi_i \cdot \sigma^t(I, a)$
            \EndFor
            \State $\sigma^{t+1}(I) \leftarrow $ estrategia calculada con la Ecuación \ref{eq:cfr-regret-matching} y la tabla de regret $r_I$  
        \EndIf
    \EndFunction
\end{algorithmic}
\end{algorithm}

\newpage

\section*{Generilized Expectimax Best Response}

En esta sección se presenta el algoritmo \textit{Generilized Expectimax Best Response} (GEBR) (Algoritmos \ref{algorithm:GEBR}, \ref{algorithm:gebr-pass1} y \ref{algorithm:gebr-pass2}) utilizado para obtener la explotabilidad de una estrategia (Algoritmo~\ref{algorithm:explotabilidad}).

En al algoritmo GEBR (Algoritmo~\ref{algorithm:GEBR}) se tiene un jugador $i$ para el cual se calculará la mejor respuesta $\sigma^*_i$ ante una estrategia fija $\sigma_{-i}$ del jugador $-i$. Este algoritmo tiene $3$ partes, primero se recorre el árbol del juego mediante búsqueda por profundidad, para determinar las profundidades de los conjuntos de información por cada uno de los jugadores, esto es el Algoritmo \ref{algorithm:gebr-pass1}. Estas listas se ordenan de forma decreciente.

La segunda parte del algoritmo GEBR consiste en recorrer el árbol varias veces, una vez por cada profundidad diferente, de mayor a menor, como se presenta en el Algoritmo~\ref{algorithm:gebr-pass2}. En el recorrido a profundidad $d$, se calculan los valore $t_I[a]$ y $b_I[a]$ para todos los conjuntos de información $I$ a una profundidad $d$ y toda acción $a \in A(I)$.

Estos arreglos permiten calcular la utilidad contrafactual (Definición \ref{def:utilidad-contrafactual}) en los conjuntos de información $I$. En efecto, note que $t_I[a] = u_i((\sigma^*_i|_{I \rightarrow a}, \sigma_{-i}),I) \cdot \pi^{\sigma_{-i}}(I)$ y $b_I[a] = \pi^{\sigma_{-i}}(I)$. Observe que $\sigma^*_i$ se obtiene al tomar alguna acción $a \in  \argmax_{a \in A(I)} \frac{t[a]}{b[a]}$ (se utiliza una estrategia pura como mejor respuesta por lo visto en el Capítulo \ref{chapter:explotabilidad}). Luego, durante el recorrido hecho para la profundidad $d$, ya se conoce el valor $\sigma^*_i(I')$ para todos los $I'$ a una profundidad $d' > d$, por lo que es posible calcular  $u_i((\sigma^*_i|_{I \rightarrow a}, \sigma_{-i}),I)$.

La última parte del algoritmo consiste en calcular el valor esperado $u_i(\sigma^*_i, \sigma_{-i})$, esto se puede obtener al utilizar el Algoritmo~\ref{algorithm:gebr-pass2} con $d = -1$. Finalmente, la explotabilidad de la estrategia $\sigma$ se obtiene al sumar las ganancias esperadas de $u_1(\sigma_1, \sigma^*_2)$ y $u_2(\sigma^*_1, \sigma_2)$ (Algoritmo~\ref{algorithm:explotabilidad}).

La complejidad de este algoritmo es $\mathcal{O(ND)}$ donde $N$ es el número de nodos del árbol y $D$ es su profundidad. Debido a la alta complejidad asintótica, se utilizó este algoritmo únicamente para calcular la explotabilidad de la estrategia final.

\newpage
\begin{algorithm}[H]
\caption{Explotabilidad}
\label{algorithm:explotabilidad}
\begin{algorithmic}[1]
    \State Inicializar el conjunto de profundidades del jugador $i$
    \State Inicializar las tablas de los valores esperados: $\forall I, t_I[a] \leftarrow 0$
    \State Inicializar las tablas de las probabilidades de alcance: $\forall I, b_I[a] \leftarrow 0$
    \State Inicializar $\sigma$ con la estrategia para la cual se desea calcular la explotabilidad
    \State
    \Function{Explotability}{}
        \State \Return GEBR($1$) + GEBR($2$)
    \EndFunction
\end{algorithmic}
\end{algorithm}

\vspace{12pt}
\begin{algorithm}[H]
\caption{Generilized Expectimax Best Response (GEBR)}
\label{algorithm:GEBR}
\begin{algorithmic}[1]
    \Function{GEBR}{$i$}
        \State GEBR-Pass1($\emptyset$, $i$, $0$)
        \State Ordenar las profundidades en orden decreciente
        \For{$d$ en el conjunto de profundidades del jugador $i$}
            \State GEBR-Pass2($\emptyset$, $i$, $d$, $0$, $1$)
        \EndFor
        \State \Return GEBR-Pass2($\emptyset$, $i$, -1, $0$, $1$)
    \EndFunction
\end{algorithmic}
\end{algorithm}

\vspace{12pt}
\begin{algorithm}[H]
\caption{Generilized Expectimax Best Response (GEBR): primer recorrido}
\label{algorithm:gebr-pass1}
\begin{algorithmic}[1]
    \Function{GEBR-Pass1}{$h$, $i$, $d$}
        \If{$h$ es un nodo terminal}
            \State \Return
        \EndIf
        \If{$h$ no es un nodo de azar}
            \State Agregar $d$ al conjunto de profundidades del jugador $i$
        \EndIf
        \For{$a \in A(h)$}
            \State GEBR-Pass1($ha$, $i$, $d+1$)
        \EndFor
    \EndFunction
\end{algorithmic}
\end{algorithm}

\newpage
~\vspace{1.5cm}
\begin{algorithm}[H]
\caption{Generilized Expectimax Best Response (GEBR): segundos recorridos}
\label{algorithm:gebr-pass2}
\begin{algorithmic}[1]
    \Function{GEBR-Pass2}{$h$, $i$, $d$, $l$, $\pi_{-i}$}
        \If{$h$ es un nodo terminal}
            \State \Return $u_i(h)$
        \ElsIf{$h$ es un nodo de azar}
            \State \Return $\sum_{a \in A(h)} f_c(a|h) \cdot$ GEBR-Pass2($ha$, $i$, $d$, $l+1$, $\pi_{-i} \cdot f_c(a | h)$)
        \EndIf
        \State Sea $I$ el conjunto de información que contiene a $h$
        \State $v \leftarrow 0$
        \If{$P(I) = i$ \AND $l > d$}
            \State $a \leftarrow \argmax_{a \in A(I)} \frac{t[a]}{b[a]}$
            \State \Return GEBR-Pass2($ha$, $i$, $d$, $l+1$, $\pi_{-i}$)
        \EndIf
        \For{$a \in A(I)$}
            \State $\pi'_{-i} \leftarrow \pi_{-i}$
            \If{$P(I) = -i$}
                \State $\pi'_{-i} \leftarrow \pi_{-i} \cdot \sigma(I, a)$
            \EndIf
            \State $v' \leftarrow $ GEBR-Pass2($ha$, $i$, $d$, $l+1$, $\pi'_{-i}$)
            \If{$P(I) = -i$}
                \State $v \leftarrow v + \sigma(I, a) \cdot v'$
            \ElsIf{$P(I) = i$ \AND $l=d$}
                \State $t_I[a] \leftarrow t_I[a] + v' \cdot \pi_{-i}$
                \State $b_I[a] \leftarrow b_I[a] + \pi_{-i}$
            \EndIf
        \EndFor
        \State \Return $v$
    \EndFunction
\end{algorithmic}
\end{algorithm}



\chapter{Regret Matching}
\label{apex:chapter:experimentos-rm}

En este apéndice se presentan las tablas detalladas para los juegos en forma normal descritos en el capítulo \ref{chapter:regret-matching}. Para cada juego se muestra una tabla con la estrategia obtenida en la última corrida de cada uno de los procedimientos y, en caso de conocerse, el equilibrio de Nash (EN). Para cada estrategia se muestra la utilidad de cada jugador si utilizan una mejor respuesta frente a la estrategia calculada para el oponente $v_1$ y $v_2$, así como la explotabilidad $\varepsilon_{\sigma}$ (ver la Sección \ref{chapter:explotabilidad} para definiciones formales).

Además, se presenta una tabla que indica el tiempo de cada ejecución ($T$), el número de iteraciones para alcanzar la cota deseada ($I$) y el tiempo promedio de cada iteración en cada una de las ejecuciones ($T/I$), así como el promedio del tiempo y del número de iteraciones para cada procedimiento.


\section*{Matching Pennies}

En este juego, si un jugador elige cada acción con una probabilidad de $0.5$, entonces su ganancia esperada es igual a $0$, sin importar la estrategia de su oponente, obteniendo el equilibrio de Nash cuando ambos jugadores utilizan esta estrategia. Las estrategias obtenidas no corresponden al equilibrio de Nash, sin embargo, garantizan una utilidad cercana a $0$ en todos los casos, obteniendo una explotabilidad no mayor a $0.008$, como se muestra en la Tabla \ref{tab:estrategias-matching-pennies}. Por lo que todas las estrategias obtenidas son un $\varepsilon$- equilibrio de Nash, con $\varepsilon < 0.008$.

\begin{table}[h]
    \centering
    \caption{Estrategias obtenidas en el juego \textit{matching pennies}.}
    \label{tab:estrategias-matching-pennies}
    \begin{tabular}{c c c c c}
    \toprule
        & EN & A & B & C \\ \midrule
        $\sigma_1$   & (0,500 0,500) & (0,500 0,500) & (0,500 0,500) & (0,500 0,500) \\
        $\sigma_2$   & (0,500 0,500) & (0,497 0,503) & (0,503 0,497) & (0,504 0,496) \\ 
        $(v_1  v_2)$ & (0,000 0,000) & (0,006 0,000) & (0,006 0,000) & (0,008 0,000) \\
        $\varepsilon_{\sigma}$ & 0 & 0,006 & 0,006 & 0,008 \\ \bottomrule
    \end{tabular}
\end{table}

La Tabla \ref{tab:resultados-matching-pennies} muestra los resultados obtenidos relacionados al tiempo y número de iteraciones de los procedimientos. El procedimiento A, regret condicional, tuvo una duración promedio de 10,276 segundos, con un número promedio de iteraciones de 3.892.550,4; obteniendo un promedio de $2,64 {\times} 10^{-6}$ segundos por iteración. Con el procedimiento B, que utiliza un vector invariante de probabilidad, se obtuvo un tiempo, número de iteraciones y tiempo por iteración promedios de 3,777 segundos, 25.616,6 iteraciones y $3,03 {\times} 10^{-5}$ segundos por iteración, respectivamente. Por último, el procedimiento C, regret incondicional, se obtuvo un tiempo promedio de 0,042, el número de iteraciones promedio fue de 16.260,5; obteniendo un promedio de $2,58 {\times} 10^{-6}$ segundos por iteración. 

\begin{table}[h]
    \centering
    \caption{Resultados experimentales del juego \textit{matching pennies}.}
    \label{tab:resultados-matching-pennies}
    \scriptsize
    \begin{tabular}{r r r r r r r r r}
    \toprule
    \multicolumn{3}{c}{A} & \multicolumn{3}{c}{B} & \multicolumn{3}{c}{C} \\ \cmidrule(r){1-3} \cmidrule(lr){4-6} \cmidrule(l){7-9}
    $T$ & $I$ & $T/I$ & $T$ & $I$ & $T/I$ & $T$ & $I$ & $T/I$ \\  \cmidrule(r){1-3} \cmidrule(lr){4-6} \cmidrule(l){7-9}
	 7,663 & 3.068.341   & $2,50 {\times} 10^{-6}$ & 0,985 & 32.510   & $3,03 {\times} 10^{-5}$ & 0,002 &    955   & $2,53 {\times} 10^{-6}$ \\
	 9,650 & 3.857.071   & $2,50 {\times} 10^{-6}$ & 1,748 & 56.946   & $3,07 {\times} 10^{-5}$ & 0,064 & 24.968   & $2,55 {\times} 10^{-6}$ \\
	23,313 & 8.950.013   & $2,60 {\times} 10^{-6}$ & 0,552 & 18.401   & $3,00 {\times} 10^{-5}$ & 0,061 & 23.854   & $2,57 {\times} 10^{-6}$ \\
	11,757 & 4.240.611   & $2,77 {\times} 10^{-6}$ & 0,309 & 10.197   & $3,03 {\times} 10^{-5}$ & 0,025 &  9.724   & $2,57 {\times} 10^{-6}$ \\
	 2,377 &   877.335   & $2,71 {\times} 10^{-6}$ & 0,747 & 24.892   & $3,00 {\times} 10^{-5}$ & 0,011 &  4.188   & $2,59 {\times} 10^{-6}$ \\
	 5,062 & 1.818.992   & $2,78 {\times} 10^{-6}$ & 0,848 & 28.142   & $3,01 {\times} 10^{-5}$ & 0,025 &  9.666   & $2,60 {\times} 10^{-6}$ \\
	 4,281 & 1.557.496   & $2,75 {\times} 10^{-6}$ & 0,132 &  4.405   & $3,01 {\times} 10^{-5}$ & 0,045 & 16.951   & $2,64 {\times} 10^{-6}$ \\
	22,110 & 8.230.100   & $2,69 {\times} 10^{-6}$ & 1,307 & 43.116   & $3,03 {\times} 10^{-5}$ & 0,021 &  8.155   & $2,64 {\times} 10^{-6}$ \\
	 3,691 & 1.432.846   & $2,58 {\times} 10^{-6}$ & 0,639 & 21.311   & $3,00 {\times} 10^{-5}$ & 0,093 & 35.270   & $2,64 {\times} 10^{-6}$ \\
	12,853 & 4.892.699   & $2,63 {\times} 10^{-6}$ & 0,500 & 16.246   & $3,08 {\times} 10^{-5}$ & 0,076 & 28.874   & $2,64 {\times} 10^{-6}$ \\ \cmidrule(r){1-3} \cmidrule(lr){4-6} \cmidrule(l){7-9}
	10,276 & 3.892.550,4 & $2,64 {\times} 10^{-6}$ & 0,777 & 25.616,6 & $3,03 {\times} 10^{-5}$ & 0,042 & 16.260,5 & $2,58 {\times} 10^{-6}$ \\\bottomrule
    \end{tabular}
\end{table}

\section*{Piedra, Papel o Tijera}

En este juego, al igual que en el anterior, ambos jugadores pueden garantizar una utilidad esperada de $0$ sin importar la estrategia utilizada por su oponente, que se obtiene al elegir cada acción con igual probabilidad. Las estrategias obtenidas son presentadas en la tabla \ref{tab:estrategias-RPS}. No todas corresponden al equilibrio de Nash exacto, sin embargo, cada una de ellas es un $\varepsilon$-equilibrio de Nash con $\varepsilon < 0,01$.

\begin{table}[h]
    \centering
    \caption{Estrategias obtenidas del juego piedra, papel o tijera.}
    \label{tab:estrategias-RPS}
    \begin{tabular}{c c c r r}
        \toprule
        & & Estrategias & $v_1 / v_2$ & $\varepsilon_{\sigma}$ \\
        \midrule
        \multirow{2}{*}{EN}
        & $\sigma_1$ & (0,333 0,333 0,333) & 0,000 & \multirow{2}{*}{0,000}\\
        & $\sigma_2$ & (0,333 0,333 0,333) & 0,000 & \\
        \hline
        \multirow{2}{*}{A}
        & $\sigma_1$ & (0,332 0,335 0,332) & 0,003 & \multirow{2}{*}{0,006}\\
        & $\sigma_2$ & (0,331 0,334 0,335) & 0,003 & \\
        \hline
        \multirow{2}{*}{B}
        & $\sigma_1$ & (0,330 0,334 0,336) & 0,006 & \multirow{2}{*}{0,010}\\
        & $\sigma_2$ & (0,329 0,335 0,337) & 0,004 & \\
        \hline
        \multirow{2}{*}{C}
        & $\sigma_1$ & (0,333 0,337 0,330) & 0,005 & \multirow{2}{*}{0,009} \\
        & $\sigma_2$ & (0,336 0,330 0,335) & 0,004 & \\
        \bottomrule
    \end{tabular}
\end{table}

La Tabla \ref{tab:resultados-RPS} muestra los resultados obtenidos relacionados al tiempo y número de iteraciones de los procedimientos. El procedimiento A, \textit{regret} condicional, tuvo una duración promedio de 25,715 segundos, con un número promedio de iteraciones de 4.519.054,1, obteniendo un promedio de $2,7 {\times} 10^{-6}$ segundos por iteración. Con el procedimiento B, que utiliza un vector invariante de probabilidad, se obtuvo un tiempo, número de iteraciones y tiempo por iteración promedios de 0,345 segundos, 6.601,3 iteraciones y $5,23 {\times} 10^{-5}$ segundos por iteración, respectivamente. Por último, el procedimiento C, \textit{regret} incondicional, se obtuvo un tiempo promedio de 0,049, el número de iteraciones promedio fue de 19.321,1, obteniendo un promedio de $2,54 {\times} 10^{-6}$ segundos por iteración.

\begin{table}[h]
    \centering
    \caption{Resultados experimentales del juego piedra, papel o tijera.}
    \label{tab:resultados-RPS}
    \scriptsize
    \begin{tabular}{r r r r r r r r r}
    \toprule
    \multicolumn{3}{c}{A} & \multicolumn{3}{c}{B} & \multicolumn{3}{c}{C} \\ \cmidrule(r){1-3} \cmidrule(lr){4-6} \cmidrule(l){7-9}
    $T$ & $I$ & $T/I$ & $T$ & $I$ & $T/I$ & $T$ & $I$ & $T/I$ \\ \cmidrule(r){1-3} \cmidrule(lr){4-6} \cmidrule(l){7-9}
    25,715 &  9.107.389   & $2,82 {\times} 10^{-6}$ & 0,724 & 13.750   & $5,26 {\times} 10^{-5}$ & 0,034 & 12.967   & $2,64 {\times} 10^{-6}$ \\
    29,494 & 10.951.479   & $2,69 {\times} 10^{-6}$ & 0,692 & 13.257   & $5,22 {\times} 10^{-5}$ & 0,041 & 16.096   & $2,57 {\times} 10^{-6}$ \\
     7,015 &  2.641.656   & $2,66 {\times} 10^{-6}$ & 0,000 & 6        & $4,36 {\times} 10^{-5}$ & 0,063 & 24.423   & $2,56 {\times} 10^{-6}$ \\
     4,610 &  1.748.365   & $2,64 {\times} 10^{-6}$ & 0,849 & 16.255   & $5,22 {\times} 10^{-5}$ & 0,048 & 18.613   & $2,56 {\times} 10^{-6}$ \\
     8,051 &  3.033.028   & $2,65 {\times} 10^{-6}$ & 0,000 & 3        & $4,28 {\times} 10^{-5}$ & 0,082 & 32.222   & $2,55 {\times} 10^{-6}$ \\
     9,870 &  3.717.278   & $2,66 {\times} 10^{-6}$ & 0,000 & 3        & $4,28 {\times} 10^{-5}$ & 0,084 & 33.042   & $2,54 {\times} 10^{-6}$ \\
     2,749 &  1.037.895   & $2,65 {\times} 10^{-6}$ & 0,000 & 3        & $4,06 {\times} 10^{-5}$ & 0,049 & 19.316   & $2,55 {\times} 10^{-6}$ \\
    11,971 &  4.517.546   & $2,65 {\times} 10^{-6}$ & 0,556 & 10.644   & $5,23 {\times} 10^{-5}$ & 0,024 &  9.601   & $2,54 {\times} 10^{-6}$ \\
    14,974 &  5.606.070   & $2,67 {\times} 10^{-6}$ & 0,000 & 3        & $3,74 {\times} 10^{-5}$ & 0,014 &  5.621   & $2,55 {\times} 10^{-6}$ \\
     7,532 &  2.829.835   & $2,66 {\times} 10^{-6}$ & 0,631 & 12.089   & $5,22 {\times} 10^{-5}$ & 0,054 & 21.310   & $2,55 {\times} 10^{-6}$ \\ \cmidrule(r){1-3} \cmidrule(lr){4-6} \cmidrule(l){7-9}
    12,198 &  4.519.054,1 & $2,70 {\times} 10^{-6}$ & 0,345 &  6.601,3 & $5,23 {\times} 10^{-5}$ & 0,049 & 19.321,1 & $2,54 {\times} 10^{-6}$ \\ \bottomrule
    \end{tabular}
\end{table}

\section*{Ficha vs. Dominó}

El primer jugador puede garantizar una ganancia esperada de, al menos $1/3$, por lo que el segundo jugador puede garantizar no perder más de $1/3$. A diferencia de los juegos anteriores, la matriz de pagos de este juegos no es simétrica y el primer jugador tiene ventaja sobre el segundo. Además, este juego no tiene un equilibrio de Nash único. En la Tabla \ref{tab:estrategias-RPS} se observa que las estrategias obtenidas para el primer jugador le permiten obtener una ganancia esperada al menos de 0,330, 0,326 y 0,329, respectivamente para los procedimientos A, B y C. Todos estos valores son menores que $1/3$, pero con una diferencia menor que 0,01. Por otra parte el segundo jugador puede garantizar un valor esperado no menor que -0,338 con cualquiera de los procedimientos.

\begin{table}[h]
    \centering
    \caption{Estrategias obtenidas del juego ficha vs dominó.}
    \label{tab:estrategias-domino}
    \begin{tabular}{c c c r r}
        \toprule
        & & Estrategias & $v_1 / v_2$ & $\varepsilon_{\sigma}$ \\
        \midrule
        \multirow{2}{*}{EN}
        & $\sigma_1$ & (0,333 0,333 0,000 0,000 0,000 0,000 0,333) & 0,333 & \multirow{2}{*}{0,000}\\
        & $\sigma_2$ & (0,333 0,000 0,333 0,000 0,333 0,000) &  -0,333 & \\
        \midrule
        \multirow{2}{*}{A}
        & $\sigma_1$ & (0,136 0,137 0,116 0,118 0,198 0,081 0,214) & 0,338 &\multirow{2}{*}{0,010} \\
        & $\sigma_2$ & (0,165 0,171 0,163 0,166 0,166 0,169) & -0,328 &\\
        \midrule
        \multirow{2}{*}{B}
        & $\sigma_1$ & (0,121 0,118 0,135 0,137 0,214 0,078 0,198) & 0,335) & \multirow{2}{*}{0,007} \\
        & $\sigma_2$ & (0,157 0,178 0,156 0,177 0,157 0,175) & -0,331 & \\
        \midrule
        \multirow{2}{*}{C}
        & $\sigma_1$ & (0,128 0,128 0,129 0,134 0,208 0,073 0,202) & 0,334 & \multirow{2}{*}{0.004} \\
        & $\sigma_2$ & (0,169 0,165 0,168 0,164 0,169 0,165) & -0,330 & \\
        \bottomrule
    \end{tabular}
\end{table}

La Tabla \ref{tab:resultados-domino} muestra los resultados obtenidos relacionados al tiempo y número de iteraciones de los procedimientos de este juego. El procedimiento A, regret condicional, tuvo una duración promedio de 319,179 segundos, con un número promedio de iteraciones de 108.319.272,4, obteniendo un promedio de $2,95 {\times} 10^{-6}$ segundos por iteración. Con el procedimiento B, que utiliza un vector invariante de probabilidad, se obtuvo un tiempo, número de iteraciones y tiempo por iteración promedios de 11,275 segundos, 75.250,2 iteraciones y $1,5 {\times} 10^{-4}$ segundos por iteración, respectivamente. Por último, el procedimiento C, regret incondicional, se obtuvo un tiempo promedio de 0,237, el número de iteraciones promedio fue de 84.318,5, obteniendo un promedio de $2,81 {\times} 10^{-6}$ segundos por iteración. 

\begin{table}[h]
    \centering
    \caption{Resultados experimentales del juego ficha vs. dominó.}
    \label{tab:resultados-domino}
    \scriptsize
    \begin{tabular}{r r r r r r r r r}
    \toprule
    \multicolumn{3}{c}{A} & \multicolumn{3}{c}{B} & \multicolumn{3}{c}{C} \\ \cmidrule(r){1-3} \cmidrule(lr){4-6} \cmidrule(l){7-9}
    $T$ & $I$ & $T/I$ & $T$ & $I$ & $T/I$ & $T$ & $I$ & $T/I$ \\ \cmidrule(r){1-3} \cmidrule(lr){4-6} \cmidrule(l){7-9}
	669,839 & 215.859.538   & $3,10 {\times} 10^{-6}$ &  4,458 &  29.721   & $1,50 {\times} 10^{-4}$ & 0,188 &  66.700   & $2,81 {\times} 10^{-6}$ \\
	309,685 & 117.568.373   & $2,63 {\times} 10^{-6}$ &  9,019 &  60.333   & $1,49 {\times} 10^{-4}$ & 0,260 &  92.401   & $2,82 {\times} 10^{-6}$ \\
	399,170 & 152.612.646   & $2,62 {\times} 10^{-6}$ &  3,646 &  24.338   & $1,50 {\times} 10^{-4}$ & 0,212 &  75.674   & $2,81 {\times} 10^{-6}$ \\
	131,570 &  38.097.125   & $3,45 {\times} 10^{-6}$ & 12,996 &  86.898   & $1,50 {\times} 10^{-4}$ & 0,145 &  51.776   & $2,80 {\times} 10^{-6}$ \\
	263,482 &  96.741.015   & $2,72 {\times} 10^{-6}$ &  4,516 &  30.170   & $1,50 {\times} 10^{-4}$ & 0,134 &  47.862   & $2,80 {\times} 10^{-6}$ \\
	203,854 &  77.156.602   & $2,64 {\times} 10^{-6}$ & 15,420 & 103.021   & $1,50 {\times} 10^{-4}$ & 0,385 & 136.950   & $2,81 {\times} 10^{-6}$ \\
	201,267 &  76.467.409   & $2,63 {\times} 10^{-6}$ & 17,399 & 115.935   & $1,50 {\times} 10^{-4}$ & 0,351 & 124.882   & $2,81 {\times} 10^{-6}$ \\
	316,007 &  97.849.871   & $3,23 {\times} 10^{-6}$ & 17,266 & 115.056   & $1,50 {\times} 10^{-4}$ & 0,203 &  72.315   & $2,81 {\times} 10^{-6}$ \\
	383,736 & 110.341.861   & $3,48 {\times} 10^{-6}$ & 12,805 &  85.532   & $1,50 {\times} 10^{-4}$ & 0,271 &  96.438   & $2,81 {\times} 10^{-6}$ \\
	313,177 & 100.498.284   & $3,12 {\times} 10^{-6}$ & 15,227 & 101.498   & $1,50 {\times} 10^{-4}$ & 0,220 &  78.187   & $2,81 {\times} 10^{-6}$ \\ \cmidrule(r){1-3} \cmidrule(lr){4-6} \cmidrule(l){7-9}
	319,179 & 108.319.272,4 & $2.95 {\times} 10^{-6}$ & 11,275 &  75.250,2 & $1,50 {\times} 10^{-04}$ & 0,237 &  84.318,5 & $2,81 {\times} 10^{-6}$ \\ \bottomrule
    \end{tabular}
\end{table}


\section*{Coronel Blotto}

En este juego no se posee un equilibrio de Nash como referencia. Sin embargo, como la matriz de pagos es simétrica, el valor del juego debe ser $0$, así que las estrategias obtenidas, se mostradas en la Tabla \ref{tab:estrategias-coronel-blotto}, deben garantizar un valor esperado cercano a $0$. En esta tabla, también se observa que cada una de las estrategias tienen una explotabilidad menor o igual que $0.011$.

\begin{table}[h]
    \centering
    \caption{Estrategias obtenidas del juego coronel Blotto.}
    \label{tab:estrategias-coronel-blotto}
    \scriptsize
    \begin{tabular}{c}
    \toprule
        Estrategias \\
        \midrule
        Procedimiento A \\ \midrule
         (0 0 0,126 0,113 0 0 0 0,080 0 0,100 0 0,131 0 0,001 0,111 0,118 0,094 0,124 0 0 0) \\
         (0 0 0,101 0,109 0 0 0 0,116 0 0,139 0 0,132 0 0,002 0,076 0,076 0,141 0,106 0 0 0) \\
         $(v_1\ v_2) = (0,002\ 0,008)$ \\
         $\varepsilon_{\sigma} = 0,01$ \\
        \midrule
        Procedimiento B \\ \midrule
         (0 0,002 0,093 0,110 0,001 0 0,002 0,111 0,001 0,128 0,001 0,126 0 0,001 0,076 0,112 0,088 0,145 0,001 0,001 0) \\
         (0 0,001 0,102 0,107 0,001 0 0,0 0,154 0,001 0,099 0 0,055 0,001 0 0,156 0,113 0,140 0,069 0,002 0,001 0) \\
         $(v_1\ v_2) = (0,004\ 0,007)$ \\
         $\varepsilon_{\sigma} = 0,011$ \\
        \midrule
        Procedimiento C \\ \midrule
         (0 0 0,119 0,106 0 0 0 0,110 0 0,107 0 0,108 0 0 0,122 0,122 0,117 0,1 0 0 0) \\
         (0 0 0,148 0,096 0 0 0 0,099 0 0,095 0 0,093 0 0 0,155 0,126 0,117 0,070 0 0 0) \\
         $(v_1\ v_2) = (0,004\ 0,005)$ \\
         $\varepsilon_{\sigma} = 0,009$ \\
        \bottomrule
    \end{tabular}
\end{table}

Los resultados obtenidos relacionados al tiempo y número de iteraciones de cada procedimiento son mostrados en la Tabla \ref{tab:resultados-coronel-blotto}.

\begin{table}[t]
    \centering
    \caption{Resultados experimentales del juego coronel Blotto.}
    \label{tab:resultados-coronel-blotto}
    \scriptsize
    \begin{tabular}{r r r r r r r r r}
    \toprule
    \multicolumn{3}{c}{A} & \multicolumn{3}{c}{B} & \multicolumn{3}{c}{C} \\ \cmidrule(r){1-3} \cmidrule(lr){4-6} \cmidrule(l){7-9}
    $T$ & $I$ & $T/I$ & $T$ & $I$ & $T/I$ & $T$ & $I$ & $T/I$ \\  \cmidrule(r){1-3} \cmidrule(lr){4-6} \cmidrule(l){7-9}
	  940,377 & 197.127.165   & $4,77 {\times} 10^{-6}$ &  90,239 &  75.420   & $1,20 {\times} 10^{-3}$ & 0,047 &  13.559   & $3,50 {\times} 10^{-6}$ \\
	  532,020 & 109.697.363   & $4,85 {\times} 10^{-6}$ &  74,886 &  62.704   & $1,19 {\times} 10^{-3}$ & 0,192 &  56.383   & $3,41 {\times} 10^{-6}$ \\
	  396,583 &  82.924.728   & $4,78 {\times} 10^{-6}$ &  56,735 &  47.416   & $1,20 {\times} 10^{-3}$ & 0,046 &  13.664   & $3,39 {\times} 10^{-6}$ \\
	  362,203 &  80.521.418   & $4,50 {\times} 10^{-6}$ &  41,290 &  34.596   & $1,19 {\times} 10^{-3}$ & 0,162 &  47.742   & $3,40 {\times} 10^{-6}$ \\
	  967,890 & 207.963.652   & $4,65 {\times} 10^{-6}$ &  69,359 &  58.123   & $1,19 {\times} 10^{-3}$ & 0,090 &  26.547   & $3,40 {\times} 10^{-6}$ \\
	1.016,540 & 245.737.655   & $4,14 {\times} 10^{-6}$ &  64,457 &  53.560   & $1,20 {\times} 10^{-3}$ & 0,118 &  34.715   & $3,41 {\times} 10^{-6}$ \\
	  553,971 & 112.170.109   & $4,94 {\times} 10^{-6}$ &  80,789 &  67.624   & $1,19 {\times} 10^{-3}$ & 0,261 &  76.657   & $3,40 {\times} 10^{-6}$ \\
	  966,339 & 204.832.370   & $4,72 {\times} 10^{-6}$ & 138,294 & 115.846   & $1,19 {\times} 10^{-3}$ & 0,358 & 105.149   & $3,40 {\times} 10^{-6}$ \\
	1.787,020 & 384.044.065   & $4,65 {\times} 10^{-6}$ &  84,924 &  70.978   & $1,20 {\times} 10^{-3}$ & 0,121 &  35.434   & $3,42 {\times} 10^{-6}$ \\
	1.232,380 & 277.204.528   & $4,45 {\times} 10^{-6}$ &  92,610 &  77.517   & $1,19 {\times} 10^{-3}$ & 0,260 &  76.285   & $3,41 {\times} 10^{-6}$ \\ \cmidrule(r){1-3} \cmidrule(lr){4-6} \cmidrule(l){7-9}
	  875,533 & 190.222.305,3 & $4,60 {\times} 10^{-6}$ &  79,358 &  66.378,4 & $1,20 {\times} 10^{-3}$ & 0,166 &  48.613,5 & $3,41 {\times} 10^{-6}$ \\ \bottomrule
    \end{tabular}
\end{table}

 El procedimiento A, regret condicional, tuvo una duración promedio de 875,533 segundos, con un número promedio de iteraciones de 190.222.305,3, obteniendo un promedio de $4,60 {\times} 10^{-6}$ segundos por iteración. Con el procedimiento B, que utiliza un vector invariante de probabilidad, se obtuvo un tiempo, número de iteraciones y tiempo por iteración promedios de 79,358 segundos, 66.378,4 iteraciones y $1,2 {\times} 10^{-3}$ segundos por iteración, respectivamente. Por último, el procedimiento C, regret incondicional, se obtuvo un tiempo promedio de 0,166, el número de iteraciones promedio fue de 48.613,5, obteniendo un promedio de $3,41 {\times} 10^{-6}$ segundos por iteración.
 

\chapter{Conterfactual Regret Minimization}
\label{apex:chapter:experimentos-cfr}

\newcommand{\graphicsCFRTwo}[4]{
\begin{figure}[H]
    \centering
    \subfigure[#2(#3)]{\includegraphics[width=0.49\textwidth]{graficas/cfr/#1/#2(#3).png}}
    \subfigure[#2(#4)]{\includegraphics[width=0.49\textwidth]{graficas/cfr/#1/#2(#4).png}}
    \vspace{25pt}
    \caption{Gráficas del regret con respecto al número de iteraciones de los juegos #2(#3) y #2(#4)}
\end{figure}
}

\newcommand{\graphicsCFRThree}[5]{
\begin{figure}[H]
    \centering
    
    \vfill
    \subfigure[#2(#3)]{\includegraphics[width=0.49\textwidth]{graficas/cfr/#1/#2(#3).png}}
    \subfigure[#2(#4)]{\includegraphics[width=0.49\textwidth]{graficas/cfr/#1/#2(#4).png}}
    \vspace{25pt}
    \subfigure[#2(#5)]{\includegraphics[width=0.49\textwidth]{graficas/cfr/#1/#2(#5).png}}
    \caption{Gráficas del regret con respecto al número de iteraciones de los juegos #2(#3), #2(#4) y #2(#5)}
    \vfill
\end{figure}
}

\newcommand{\graphicsCFR}[6]{
\begin{figure}[H]
    \centering
    
    \subfigure[#2(#3)]{\includegraphics[width=0.49\textwidth]{graficas/cfr/#1/#2(#3).png}}
    \subfigure[#2(#4)]{\includegraphics[width=0.49\textwidth]{graficas/cfr/#1/#2(#4).png}}
    \vspace{25pt}
    \subfigure[#2(#5)]{\includegraphics[width=0.49\textwidth]{graficas/cfr/#1/#2(#5).png}}
    \subfigure[#2(#6)]{\includegraphics[width=0.49\textwidth]{graficas/cfr/#1/#2(#6).png}}
    \caption{Gráficas del regret con respecto al número de iteraciones de los juegos #2(#3), #2(#4), #2(#5) y #2(#6)}
\end{figure}
}


\section*{Gráficas del juego One Card Poker}
\vspace{1cm}
\graphicsCFRTwo{ocp}{OCP}{3}{12}
\newpage
~\vspace{2cm}
\graphicsCFR{ocp}{OCP}{50}{200}{1000}{5000}

\newpage
\section*{Gráficas del juego Dudo}
\vspace{1cm}
\graphicsCFR{dudo}{Dudo}{3,1,1}{3,1,2}{3,2,1}{3,2,2}
\newpage
~\vspace{2cm}
\graphicsCFR{dudo}{Dudo}{4,1,1}{4,1,2}{4,2,1}{4,2,2}
\newpage
~\vspace{2cm}
\graphicsCFR{dudo}{Dudo}{5,1,1}{5,1,2}{5,2,1}{5,2,2}
\newpage
~\vspace{2cm}
\graphicsCFRThree{dudo}{Dudo}{6,1,1}{6,1,2}{6,2,1}
\vfill

\newpage
\section*{Gráficas del juego Dominó}
\vspace{1cm}
\graphicsCFR{domino}{Domino}{2,2}{3,2}{3,3}{3,4}



\end{document}
