\section{Detalles de Implementación y Ejecución}

Los algoritmos fueron implementados en el lenguaje de programación C++, utilizando la librería estándar y una librería adicional llamada \textit{Eigen}, para factorizar matrices y resolver sistemas de ecuaciones.

Se implementó una clase para encontrar un equilibrio de Nash mediante el algoritmo de \textit{Regret Matching}. En cada iteración la actualización de las estrategias depende de cada procedimiento según las fórmulas propuestas en la sección anterior.

En el juego Coronel Blotto la matriz de pagos no tiene un tamaño fijo y además no es proporcionada de forma explícita, por lo que es necesario generarla dependiendo de los parámetros. Para esto se creó un programa que, dado el número de campos de batalla ($N$) y el número de soldados ($S$), genera todas las posibles distribuciones de cada uno de los jugadores mediante un algoritmo de \textit{backtracking} y calcula el pago para cada juego posible, obteniendo como salida del programa la matriz deseada. De esta forma se generó la matriz de pagos para este juego cuando $N = 3$ y $S = 5$.

Las ejecuciones de estos algoritmos se realizaron en una máquina personal, con las siguientes características:
\begin{itemize}[noitemsep]
    \item Procesador: Intel\textsuperscript{\textregistered} Core\textsuperscript{\texttrademark} i5-8250U CPU  \makeatletter{@} 1.60GHz
    \item 8CPUs
    \item 8Gb de memoria RAM
    \item Sistema Operativo: Ubuntu 18.04.3 LTS
\end{itemize}

