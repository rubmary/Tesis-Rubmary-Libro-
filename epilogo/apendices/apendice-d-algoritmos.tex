\chapter{Algoritmos}
\label{apex:chapter:algoritmos}
\algnewcommand{\algorithmicand}{\textbf{ and }}
\algnewcommand{\algorithmicor}{\textbf{ or }}
\algnewcommand{\OR}{\algorithmicor}
\algnewcommand{\AND}{\algorithmicand}

En esta sección se presentan los algoritmos \textit{Counterfactual Regret Minimization} (CFR) con muestreo de los nodos de azar (Algoritmo~\ref{algorithm:CFR}) y \textit{Generilized Expectimax Best Response} (GEBR) (Algoritmos \ref{algorithm:GEBR}, \ref{algorithm:gebr-pass1} y \ref{algorithm:gebr-pass2}) este último utilizado para obtener la explotabilidad de una estrategia (Algoritmo~\ref{algorithm:explotabilidad}). Es importante notar que $h$ representa una historia (o un estado en el juego) y $ha$ representa la hitoria $h$ con la acción $a$ anexada, es decir, el estado del juego en el cual después de la historia $h$ fue elegida la acción $a$.

En al algoritmo GEBR (Algoritmo~\ref{algorithm:GEBR}) se tiene un jugador $i$ para el cual se calculará la mejor respuesta $\sigma^*_i$ ante una estrategia fija $\sigma_{-i}$ del jugador $-i$. Este algoritmo tiene $3$ partes, primero se recorre el árbol del juego mediante DFS para determinar las profundidades de los conjuntos de información por cada uno de los jugadores, esto es el Algoritmo \ref{algorithm:gebr-pass1}. Estas listas se ordenan de forma decreciente.

La segunda parte del algoritmo GEBR consiste en recorrer el árbol varias veces, una vez por cada profundidad diferente, de mayor a menor, como se presenta en el Algoritmo~\ref{algorithm:gebr-pass2}. En el recorrido a profundidad $d$, se calculan los valore $t_I[a]$ y $b_I[a]$ para todos los conjuntos de información $I$ a una profundidad $d$ y toda acción $a \in A(I)$. Estos arreglos nos permiten calcular la utilidad contrafactual (Definición \ref{def:utilidad-contrafactual}) en los conjuntos de información $I$. En efecto, note que $t_I[a] = u_i((\sigma^*_i|_{I \rightarrow a}, \sigma_{-i}),I) \cdot \pi^{\sigma_{-i}}(I)$ y $b_I[a] = \pi^{\sigma_{-i}}(I)$. Es importante destacar que $\sigma^*_i$ se obtiene al tomar alguna acción $a \in  \argmax_{a \in A(I)} \frac{t[a]}{b[a]}$ (se utiliza una estrategia pura como mejor respuesta por lo visto en el Capítulo \ref{chapter:explotabilidad}). Luego, durante el recorrido hecho para la profundidad $d$, ya se conoce el valor $\sigma^*_i(I')$ para todos los $I'$ a una profundidad $d' > d$, por lo que es posible calcular  $u_i((\sigma^*_i|_{I \rightarrow a}, \sigma_{-i}),I)$.

La última parte del algoritmo consiste en calcular el valor esperado $u_i(\sigma^*_i, \sigma_{-i})$, esto se puede obtener al utilizar el Algoritmo~\ref{algorithm:gebr-pass2} con $d = -1$. Finalmente, la explotabilidad de la estrategia $\sigma$ se obtiene al sumar las ganancias esperadas de $u_1(\sigma_1, \sigma^*_2)$ y $u_2(\sigma^*_1, \sigma_2)$ (Algoritmo~\ref{algorithm:explotabilidad}).

\begin{algorithm}
\caption{\textit{Counterfactual Regret Minimization} (CFR) con \textit{chance-sampled}}
\label{algorithm:CFR}
\begin{algorithmic}[1]
    \State Inicializar tablas acumulativas de \textit{regret}: $\forall I, r_I[a] \leftarrow 0$.
    \State Inicializar tablas acumulativas de estrategias: $\forall I, s_I[a] \leftarrow 0$.
    \State Inicializar perfil inicial: $\sigma^1(I, a) \leftarrow 1/|A(I)|$ \label{x}
    \State
    \Function{CFR}{$h$, $i$, $t$, $\pi_1$, $\pi_2$}
        \If{$h$ es terminal}
            \State \Return $u_i(h)$
        \ElsIf{$h$ es un nodo de azar}
            \State Seleccionar una acción $a \sim f_c(h)$
            \State \Return CFR($ha$, $i$, $t$, $\pi_1$, $\pi_2$)
        \EndIf
        \State Sea $I$ el conjunto de información que contiene a $h$.
        \State $v_{\sigma} \leftarrow 0$
        \State $v_{\sigma_{I \rightarrow [a]}} \leftarrow 0 $ para todo $a \in A(I)$
        \For{$a \in A(I)$}
            \If{P(h) = 1}
                \State $v_{\sigma_{I \rightarrow [a]}} \leftarrow$ CFR($ha$, $i$, $t$, $\sigma^t(I, a) \cdot \pi_1$, $\pi_2$)
            \ElsIf{P(h) = 2}
                \State $v_{\sigma_{I \rightarrow [a]}} \leftarrow$ CFR($ha$, $i$, $t$, $\pi_1$, $\sigma^t(I, a) \cdot \pi_2$)
            \EndIf
            \State $v_{\sigma} \leftarrow v_{\sigma} + \sigma^t(I, a) \cdot v_{\sigma_{I \rightarrow [a]}}$
        \EndFor
        \If{$P(h) = i$}
            \For{$a \in A(I)$}
                \State $r_I[a] \leftarrow r_I[a] + \pi_{-i} \cdot (v_{\sigma_{I \rightarrow [a]}} - v_{\sigma})$
                \State $s_{I}[a] \leftarrow s[I][a] + \pi_i \cdot \sigma^t(I, a)$
            \EndFor
            \State $\sigma^{t+1}(I) \leftarrow $ estrategia calculada con la Ecuación \ref{eq:cfr-regret-matching} y la tabla de regret $r_I$  
        \EndIf
    \EndFunction
    \State

    \Function{Solve}{}
        \For{$t = {1, 2, ..., T}$}
            \For{$i \in \{1, 2\}$}
                \State CFR($\emptyset$, $i$, $t$, $1$, $1$)
            \EndFor
        \EndFor
    \EndFunction
\end{algorithmic}
\end{algorithm}

\begin{algorithm}
\caption{Explotabilidad}
\label{algorithm:explotabilidad}
\begin{algorithmic}[1]
    \State Inicializar el conjunto de profundidades del jugador $i$
    \State Inicializar las tablas de los valores esperados: $\forall I, t_I[a] \leftarrow 0$
    \State Inicializar las tablas de las probabilidades de alcance: $\forall I, b_I[a] \leftarrow 0$
    \State Inicializar $\sigma$ con la estrategia para la cual se desea calcular la explotabilidad
    \State
    \Function{Explotability}{}
        \State \Return GEBR($1$) + GEBR($2$)
    \EndFunction
\end{algorithmic}
\end{algorithm}

\begin{algorithm}
\caption{Generilized Expectimax Best Response (GEBR)}
\label{algorithm:GEBR}
\begin{algorithmic}[1]
    \Function{GEBR}{$i$}
        \State GEBR-Pass1($\emptyset$, $i$, $0$)
        \State Ordenar las profundidades en orden decreciente
        \For{$d$ en el conjunto de profundidades del jugador $i$}
            \State GEBR-Pass2($\emptyset$, $i$, $d$, $0$, $1$)
        \EndFor
        \State \Return GEBR-Pass2($\emptyset$, $i$, -1, $0$, $1$)
    \EndFunction
\end{algorithmic}
\end{algorithm}

\begin{algorithm}
\caption{Generilized Expectimax Best Response (GEBR): primer recorrido}
\label{algorithm:gebr-pass1}
\begin{algorithmic}[1]
    \Function{GEBR-Pass1}{$h$, $i$, $d$}
        \If{$h$ es un nodo terminal}
            \State \Return
        \EndIf
        \If{$h$ no es un nodo de azar}
            \State Agregar $d$ al conjunto de profundidades del jugador $i$
        \EndIf
        \For{$a \in A(h)$}
            \State GEBR-Pass1($ha$, $i$, $d+1$)
        \EndFor
    \EndFunction
\end{algorithmic}
\end{algorithm}


\begin{algorithm}
\caption{Generilized Expectimax Best Response (GEBR): segundos recorridos}
\label{algorithm:gebr-pass2}
\begin{algorithmic}[1]
    \Function{GEBR-Pass2}{$h$, $i$, $d$, $l$, $\pi_{-i}$}
        \If{$h$ es un nodo terminal}
            \State \Return $u_i(h)$
        \ElsIf{$h$ es un nodo de azar}
            \State \Return $\sum_{a \in A(h)} f_c(a|h) \cdot$ GEBR-Pass2($ha$, $i$, $d$, $l+1$, $\pi_{-i} \cdot f_c(a | h)$)
        \EndIf
        \State Sea $I$ el conjunto de información que contiene a $h$
        \State $v \leftarrow 0$
        \If{$P(I) = i$ \AND $l > d$}
            \State $a \leftarrow \argmax_{a \in A(I)} \frac{t[a]}{b[a]}$
            \State \Return GEBR-Pass2($ha$, $i$, $d$, $l+1$, $\pi_{-i}$)
        \EndIf
        \For{$a \in A(I)$}
            \State $\pi'_{-i} \leftarrow \pi_{-i}$
            \If{$P(I) = -i$}
                \State $\pi'_{-i} \leftarrow \pi_{-i} \cdot \sigma(I, a)$
            \EndIf
            \State $v' \leftarrow $ GEBR-Pass2($ha$, $i$, $d$, $l+1$, $\pi'_{-i}$)
            \If{$P(I) = -i$}
                \State $v \leftarrow v + \sigma(I, a) \cdot v'$
            \ElsIf{$P(I) = i$ \AND $l=d$}
                \State $t_I[a] \leftarrow t_I[a] + v' \cdot \pi_{-i}$
                \State $b_I[a] \leftarrow b_I[a] + \pi_{-i}$
            \EndIf
        \EndFor
        \State \Return $v$
    \EndFunction
\end{algorithmic}
\end{algorithm}





