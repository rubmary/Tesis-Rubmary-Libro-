\chapter{Estrategias Minimax y Maximin}
\label{apex:chapter:estrategias}

Una estrategia \textit{minimax} del jugador $i$, consiste en minimizar la ganancia de la mejor respuesta del jugador $-i$. Es decir, el jugador $i$ juega para \say{castigar} al jugador $-i$, sin tomar en cuenta su propia ganancia. Por otra parte en una estrategia \textit{maximin}, el jugador busca maximizar su ganancia, suponiendo que su oponente juega para perjudicarlo.

\begin{definition}[{\cite[p.~15--16]{bib:handbook-blai}}]
El conjunto de estrategias \textit{minimax} para el jugador $i$ en contra del jugador $-i$ es
\begin{alignat}{1}
\{ \sigma_i : \max_{\sigma_{-i}}u_{-i}(\sigma_i, \sigma_{-i}) = \argmin_{\sigma'_i}{\max_{\sigma_{-i}} u_{-i}(\sigma'_i, \sigma_{-i})} \}\,,
\end{alignat}
y el valor minimax del jugador $-i$ es $\min_{\sigma_i}{\max_{\sigma_{-i}}{u_{-i}(\sigma_i, \sigma_{-i})}}$.
El conjunto de estrategias \textit{maximin} para el jugador $i$ en contra del jugador $-i$ es
\begin{alignat}{1}
\{ \sigma_i : \min_{\sigma_{-i}}u_i(\sigma_i, \sigma_{-i}) = \argmax_{\sigma'_i}{\min_{\sigma_{-i}} u_i(\sigma'_i, \sigma_{-i})} \}\,,
\end{alignat}
y el valor \textit{maximin} del jugador $i$ es $\max_{\sigma_i}{\min_{\sigma_{-i}}{u_i(\sigma_i, \sigma_{-i})}}$.
\end{definition}

Como la estrategia \textit{minimax} o \textit{maximin} de un jugador no depende de la estrategia del oponente, se pueden definir perfiles estratégicos \textit{minimax} y \textit{maximin}. Un perfil estratégico mixto $\sigma = (\sigma_1, \sigma_2)$ es un perfil estratégico \textit{minimax} (\textit{maximin}) si $\sigma_1$ es un estrategia minimax (resp.\ \textit{maximin}) para el jugador $1$ y $\sigma_2$ es una estrategia \textit{minimax} (resp.\ \textit{maximin}) para el jugador $2$.

\begin{example}
\label{ex:ejemplos-min-max}
Considere el juego en forma normal de $2$ jugadores mostrado en la Tabla~\ref{table:ejemplos-min-max} donde $S_1 = S_2 = \{1, 2\}$.
\end{example}

Calculemos estrategias \textit{minimax} y \textit{maximin} para el primer jugador. Las estrategias \textit{minimax} del primer jugador vienen expresadas por
\begin{alignat}{1}
\argmin_{(\beta_1, \beta_2) \in \Delta_2 }\ {\max_{(\theta_1, \theta_2) \in \Delta_2}
{\theta_1(4\beta_1 -\beta_2) + \theta_2(-2\beta_1 + 2\beta_2)}} \,.
\end{alignat}

La solución es obtenida cuando la ganancia del segundo jugador no depende de las elecciones de $\theta_1$ y $\theta_2$, es decir, cuando $4\beta_1 - \beta_2 = -2\beta_1 + 2\beta_2$, lo que implica que $(\beta_1, \beta_2) = \left(\frac{1}{3}, \frac{2}{3} \right)$. 
\textcolor{red}{\bf **** No se entiende bien como se concluye los valores de beta ****}
El valor \textit{minimax} del segundo jugador es $\frac{2}{3}$. En este caso, el primer jugador elige su estrategia considerando, únicamente, la ganancia del oponente, sin tomar en cuenta su propia ganancia.
Por otra parte, las estrategias \textit{maximin} se corresponden con
\begin{alignat}{1}
\argmax_{(\beta_1, \beta_2) \in \Delta_2 }\ {\min_{(\theta_1, \theta_2) \in \Delta_2}
{\theta_1(2\beta_1 -\beta_2) + \theta_2(-\beta_1 + 2\beta_2)}} \,.
\end{alignat}

La estrategia \textit{maximin} es alcanzada cuando la ganancia esperada del primer jugador no depende de la elección de $\theta_1$ y $\theta_2$, es decir cuando $2\beta_1 - \beta_2 = -\beta_1 + 2\beta_2$, lo que ocurre si y sólo si $(\beta_1, \beta_2) = \left(\frac{1}{2}, \frac{1}{2}\right)$. El valor \textit{maximin} del primer jugador es $\frac{1}{2}$.

\begin{table}[t]
\begin{center}
\caption{Tabla de pagos del juego del Ejemplo~\ref{ex:ejemplos-min-max}.}
\label{table:ejemplos-min-max}
\begin{tabular}{ r | c | c |}
 %& \multicolumn{1}{c}{} & \multicolumn{2}{c}{Jugador $2$} \\
 \multicolumn{1}{c}{} & \multicolumn{1}{c}{1} & \multicolumn{1}{c}{2}  \\ \cline{2-3}
 %\multirow{2}{*}{Jugador $1$}
 1 & $2, 4$ & $-1, -2$ \\ \cline{2-3}
 2 & $-1, -1$ & $2, 2$ \\ \cline{2-3}
\end{tabular}
\end{center}
\end{table}

\begin{theorem}[\cite{bib:von-neumann}]
\label{theo:von-neumann}
Para cualquier juego finito para dos jugadores de suma cero, y para cualquier equilibrio de Nash del juego, cada jugador tiene una ganancia esperada cuyo valor es igual al valor minimax y valor maximin de dicho jugador. \end{theorem}

El Teorema~\ref{theo:von-neumann} muestra que las estrategias \textit{minimax}, \textit{maximin} y los equilibrios de Nash coinciden en los juegos de dos jugadores con suma cero.