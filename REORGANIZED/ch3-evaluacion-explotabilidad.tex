\chapter{Evaluación de Estrategias y Explotabilidad}
\label{section:explotabilidad}

En un juego de suma cero el \textbf{valor del juego} es igual a la ganancia esperada del primer jugador cuando los jugadores utilizan un Equilibrio de Nash $\sigma^* = (\sigma^*_1, \sigma^*_2)$. Es decir, el valor del juego es igual a $u = u_1(\sigma^*)$. Supongamos ahora que el jugador $1$  usa una estrategia $\sigma_1$, que es una ligera modificación de $\sigma^*$, entonces el jugador $2$ puede usar una estrategia que sea mejor respuesta a $\sigma_1$, digamos $\sigma^{\prime}_2$. Luego,
\begin{alignat}{1}
u_2(\sigma_1, \sigma^{\prime}_2)\ \geq\ u_2(\sigma_1, \sigma^*_2)\ \geq\ u_2(\sigma^*_1, \sigma^*_2) \,.
\end{alignat}

La primera desigualdad se obtiene porque $\sigma^{\prime}_2$ es mejor respuesta del jugador 2 a $\sigma_1$ y la segunda desigualdad ocurre porque $\sigma^*_1$ es mejor respuesta del jugador 1 a $\sigma^*_2$. Luego $u_2(\sigma_1, \sigma^{\prime}_2) = u_2(\sigma^*_1, \sigma^*_2) + \varepsilon_1$ para algún $\varepsilon_1 \geq 0$. Por lo tanto, la estrategia del jugador $1$ se volvió \textit{explotable} por una cantidad $\varepsilon_1$. De forma análoga se puede obtener que, si el jugador $2$ utiliza una estrategia $\sigma_2$ ligeramente alejada del equilibrio de Nash, esta estrategia será explotable por una cantidad no negativa $\varepsilon_2$.

La \textbf{explotabilidad} $\varepsilon_\sigma$ de una estrategia $\sigma = (\sigma_1, \sigma_2)$ es definida por la expresión $\varepsilon_{\sigma} = \varepsilon_1 + \varepsilon_2$. La explotabilidad es usada frecuentemente para medir la distancia de una estretegia al equilibrio de Nash \cite[p. 7]{bib:thesis-marc-lanctot}. Si definimos $v_i = u_i(\sigma_i, \sigma^{\prime}_{-i})$, entonces por lo anterior $v_i = u_i(\sigma^*) + \varepsilon_i$. Si $u=u_1(\sigma^*)$ es el valor del juego, note que $v_1 = u_1(\sigma^*) + \varepsilon_1 = u + \varepsilon_1$ y $v_2 = u_2(\sigma^*) + \varepsilon_2 = -u + \varepsilon_2$. Entonces,  $\varepsilon_{\sigma} = u + \varepsilon_1 - u + \varepsilon_2 =  v_1 + v_2$.

Queremos encontrar una forma sencilla de calcular la cantidad $v_i$. Para lograr esto utilizamos el hecho que para cualquier estrategia de cualquier jugador siempre existe una mejor respuesta cuyo soporte tiene un único elemento (Corolario del Teorema \ref{theo:mejor-respuesta}). Este resultado permite obtener la siguiente expresión para $v_i$ que permite calcular la explotabilidad $\varepsilon_\sigma$ de una estrategia dada $\sigma=(\sigma_1,\sigma_2)$:
\begin{alignat}{1}
v_i\ =\ \max_{s_{-i} \in S_{-i}} u_i(\sigma_i, s_{-i}) \,.
\end{alignat}

\noindent\textcolor{red}{\bf **** EJEMPLOS: RPS? ****}

Estas fórmulas se usan para calcular la explotabilidad de las estrategias obtenidas al ejecutar cada uno de los procedimientos implementados en cada uno de los juegos en forma normal que se utilizan en los experimentos. \textcolor{red}{\bf **** Qué pasa para los juegos en forma extensa? ****}



