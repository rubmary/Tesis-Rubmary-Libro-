\chapter{Juegos en Formal Normal o Estratégica}
\label{chapter:forma-normal}

En un juego en forma normal cada uno de los jugadores eligen una única ``acción'' (que puede representar una estrategia completa para un juego complejo) de forma simultánea, y cada jugador obtiene un pago de acuerdo a las acciones realizadas por todos los jugadores. Frecuentemente, estos juegos también se llaman \textit{one-shot game} (juegos de un sólo disparo) ya que cada uno de los jugadores realiza una única acción \cite{bib:introductionCFR}. El ejemplo clásico es el juego piedra, papel o tijera (RPS por sus siglas en inglés). En este juego cada uno de los dos jugadores elige una de tres opciones mediante un gesto con sus manos: piedra (con un puño cerrado), papel (con la mano extendida) o tijera (con los dedos índice y medio levantados en forma de ``V"). La piedra gana contra la tijera, la tijera gana contra el papel y el papel gana contra la piedra. Si ambos jugadores eligen la misma opción, entonces es un empate.

\begin{definition}[\cite{bib:correlated-equilibrium}]
\label{def:forma-normal}
Un juego de N personas en \textbf{forma normal} (o estratégica) es una tupla $\Gamma = (N, (S_i)_{i \in N}, (u_i)_{i \in N})$, donde:
	\begin{itemize}[]
		\item $N = \{1, 2, \dots, N\}$ es el conjunto de jugadores.
		\item  $S_i$ es el conjunto de \textbf{estrategias puras} (o acciones) del jugador $i$.
		\item $u_i : \Pi _{i \in N} S_i \rightarrow \mathbb{R}$ es la \textbf{función de pago} del jugador $i$.
	\end{itemize}
\end{definition}

\section{Perfiles Estratégicos, Estrategias Mixtas, y Perfiles Estratégicos Mixtos}
A continuación se presentan conceptos básicos que denotan las diversas formas en que los jugadores pueden comportarse para un juego en forma normal. Las estrategias puras son la base a partir de las cuales se construyen las estrategias mixtas. Las estrategias puras se agregan en perfiles estratégicos que denotan el comportamiento de todos los jugadores de forma simultánea, y los perfiles estratégicos mixtos agregan las estrategias mixtas \cite{bib:tutorial-existence-nash}.

\begin{definition} Un \textbf{perfil estratégico} (o perfil de acción) es una $N$-tupla formada por una estrategia pura para cada jugador. $S = \Pi_{i \in N}S_i$ es el conjunto de perfiles estratégicos y $s = (s_i)_{i \in N}$ representa un elemento genérico de $S$.  
\end{definition}

Se denotará con $s_{-i}$ la combinación de las estrategias de todos los jugadores excepto la del jugador $i$, i.e., $s_{-i} = (s_{i'})_{i' \neq i}$. Frecuentemente se descompone una estrategia $s$ en un par $(s_i,s_{-i})$ donde la primera componente es una estrategia pura para el jugador $i$ y la segunda componente es un vector de estrategias puras para los otros jugadores. Además, los juegos en forma normal pueden representarse como una tabla $N$-dimensional donde cada dimensión está asociada a un jugador y sus filas/columnas corresponden a las acciones de su jugador correspondiente. Cada una de las entradas de la tabla corresponde a un único perfil estratégico (pues representan la intersección de una única acción para cada jugador) y éstas contienen un vector de pagos para cada jugador \cite{bib:introductionCFR}.

Piedra, papel o tijera es un juego para dos jugadores y las acciones (o estrategias puras) son las mismas para cada jugador: $S_1 = S_2 = \{\mathcal{R}, \mathcal{P}, \mathcal{S} \}$ donde $\mathcal{R}$ es piedra, $\mathcal{P}$ es papel, y $\mathcal{S}$ es tijera. La Tabla \ref{table:pago-RPS} es la tabla de pagos correspondiente a este juego, en el cual $N = 2$, por lo que la tabla es de $2$ dimensiones. Las filas representan las acciones del jugador $1$ y las columnas las del jugador $2$, además cada celda corresponde a un perfil estratégico. Por ejemplo, si el primer jugador elige tijera y el segundo jugador elige papel, la estrategia pura es $s = (\mathcal{S}, \mathcal{P})$ y la celda correspondiente es la que se encuentra en la posición $(3, 2)$ (tercera fila y segunda columna). Esta celda contiene el vector $(1, -1)$ ya que en este caso el primer jugador gana obteniendo una utilidad de $1$, mientras que el segundo jugador pierde obteniendo una utilidad de $-1$.

\begin{table}[h]
\begin{center}
\caption{Tabla de pagos de la forma normal del juego piedra, papel o tijera.}
\label{table:pago-RPS}
\begin{tabular}{l |c | c | c |}
  \multicolumn{1}{c}{} & \multicolumn{1}{c}{$\mathcal{R}$ (piedra)} & \multicolumn{1}{c}{$\mathcal{P}$ (papel)} & \multicolumn{1}{c}{$\mathcal{S}$ (tijera)} \\ \cline{2-4}
$\mathcal{R}$ (piedra) & $0,0$ & $-1,1$ & $1,-1$ \\ \cline{2-4}
$\mathcal{P}$ (papel) & $1,-1$ & $0,0$ & $-1,1$ \\ \cline{2-4}
$\mathcal{S}$ (tijera) & $-1,1$ & $1,-1$ & $0,0$ \\ \cline{2-4}
\end{tabular}
\end{center}
\end{table}

En vez de realizar siempre la misma acción, un jugador puede elegir su jugada de acuerdo a una distribución de probabilidad, la cual se denomina una \textbf{estrategia mixta}. Dado un conjunto finito $A$, se denota con $\Delta(A)$ al conjunto de distribuciones de probabilidad sobre $A$, es decir $\Delta(A) = \{ (x_a)_{a \in A} : \sum_{a \in A} x_a = 1, x_a \geq 0\}$. También se denotará a $\Delta^n$, como el simplex $n$-dimensional, i.e., $\Delta^n = \{ x \in \mathbb{R}^{n+1} : \sum_{0 \leq i \leq n}x_i = 1, x_i \geq 0 \}$.

\begin{definition} Una \textbf{estrategia mixta} del jugador $i$, denotada con $\sigma_i$, es una distribución de probabilidad sobre el conjunto $S_i$; es decir $\sigma_i \in \Delta(S_i)$. Denotamos con $\sigma_i(s_i)$ la probabilidad de que el jugador $i$ elija la acción $s_i \in S_i$. 
\end{definition}

\begin{definition}
El \textbf{soporte} (support) de una estrategia mixta $\sigma_i \in \Delta(S_i)$ del jugador $i$ es el conjunto de estrategias puras con una probabilidad positiva de ser elegidas:
\begin{alignat}{1}
\text{support}(\sigma_i)\ =\ \{s_i : \sigma_i(s_i) > 0 \} \,.
\end{alignat}
\end{definition}

\begin{definition}
Un \textbf{perfil estratégico mixto} $\sigma$ consiste en una estrategia mixta para cada jugador; es decir, $\sigma \in \Pi_{i \in N} \Delta(S_i)$ es una tupla de forma $\sigma=(\sigma_i)_{i \in N}$.
\end{definition}

Para $\sigma = (\sigma_i)_{i \in N}$ y $s = (s_i)_{i \in N}$, $\sigma(s)$ denota la probabilidad de que el perfil estratégico mixto elija la estrategia mixta $s$; i.e., $\sigma(s)=\prod_{i\in N} \sigma_i(s_i)$. Para un jugador $i$, un perfil $\sigma$ se descompone en un par $(\sigma_i,\sigma_{-i})$ donde $\sigma_i$ es una estrategia mixta del jugador $i$ y $\sigma_{-i}$ es un perfil para el resto de los jugadores. Similarmente, $\sigma_{-i}(s_{-i})=\prod_{j\in N,j\neq i}\sigma_j(s_j)$ denota la probabilidad de que los jugadores en $\{j\}_{j\neq i}$ elijan las estrategias $\{s_j\}_{j\neq i}$ especificadas en el perfil $s_{-i}$. Finalmente, si $x$ es una estrategia pura para el jugador $i$, también utilizamos $x$ para denotar la estrategia mixta $\sigma_i$ para el jugador $i$ que elige $x$ con probabilidad 1 y elige las otras estrategias puras del jugador $i$ con probabilidad 0; i.e., $\sigma_i(x)=1$ y $\sigma_i(x')=0$ para toda $x'\in S_i$ con $x\neq x$.

En el juego piedra, papel o tijera una posible estrategia mixta para el jugador $i$ es elegir piedra o papel con probabilidad $\frac{1}{2}$ y nunca eligir tijera. Si denotamos a dicha estrategia con $\sigma_i$, entonces $\sigma_i(\mathcal{R}) = \sigma_i(\mathcal{P}) = \frac{1}{2}$ y $\sigma_i(\mathcal{S}) = 0$. Esta estrategia también puede ser representada por $\sigma_i = (\frac{1}{2}, \frac{1}{2}, 0)$, donde la primera componente representa la probabilidad del jugador de elegir piedra, la segunda la probabilidad de elegir papel y la última la de elegir tijera. Otra posible estrategia mixta $\sigma'_i$ consiste en elegir piedra con probabilidad $\frac{1}{3}$, papel con probabilidad $\frac{1}{2}$ y tijera con probabilidad $\frac{1}{6}$, i.e. $\sigma'_i = \left(\frac{1}{3}, \frac{1}{2}, \frac{1}{6} \right)$. Note que el soporte de $\sigma_i$ es igual a $\text{support}(\sigma_i) = \{\mathcal{R}, \mathcal{P} \}$ y el soporte de $\sigma'_{i}$ es $\text{support}(\sigma'_i) = S_i$, pues en esta última estrategia todas las acciones tienen una probabilidad positiva de ser elegidas. Luego, si el jugador $1$ decide utilizar la estrategia $\sigma_i$ y el jugador $2$ la estrategia $\sigma'_i$, entonces $\sigma_1 = \left( \frac{1}{2}, \frac{1}{2}, 0 \right)$, $\sigma_2 = \left(\frac{1}{3}, \frac{1}{2}, \frac{1}{6}\right)$ y $\sigma = (\sigma_1, \sigma_2)$ es un perfil estratégico mixto. Sea $s = (\mathcal{P}, \mathcal{S})$ el perfil estratégico (puro), donde el primer jugador elige papel y el segundo jugador elige tijera, luego $\sigma(s) = \sigma(\mathcal{P}) \cdot \sigma(\mathcal{S}) = \frac{1}{2} \cdot \frac{1}{6} = \frac{1}{12}$. Por otra parte, si $s' = (\mathcal{S}, \mathcal{P})$ entonces $\sigma(s') = 0 \cdot \frac{1}{2} = 0$.

\section{Ganancia Esperada y Mejor Respuesta}
La ganancia esperada del jugador $i$ asociada al perfil estratégico mixto $\sigma$ denota el valor promedio que el jugador $i$ obtendría después de jugar el juego infinitas veces cuando todos los jugadores utilizan las estrategias mixtas especificadas en $\sigma$. 

\begin{definition}
\label{def:ganancia-esperada}
La \textbf{ganancia esperada} del jugador $i$ dado un perfil estratégico mixto $\sigma$ es
\begin{alignat}{1}
	u_i(\sigma)\ =\ \sum_{s \in S} u_i(s) \sigma(s)\ =\ \sum_{s \in S} u_i(s) \prod _{j \in N} \sigma_j(s_j)\ =\ \sum_{s \in S} u_i(s) \sigma_i(s_i) \sigma_{-i}(s_{-i})\,.
\end{alignat}
\end{definition}

La ganancia esperada del jugador $i$ la podemos descomponer como se muestra a continuación. (La demostración de este Teorema y otros contenidos en la tesis se encuentran en el Apéndice~\ref{apex:chapter:pruebas}).

\begin{theorem}
\label{theo:ganancia-esperada}
La ganancia esperada $u_i(\sigma)$ del jugador $i$ dado el perfil estratégico $\sigma$ satisface:
\begin{alignat}{1}
u_i(\sigma)\ =\ \sum_{s_i\in S_i} \sigma_i(s_i) \sum_{s_{-i}\in S_{-i}} \sigma_{-i}(s_{-i}) u_i(s_i,s_{-i}) \,.
\end{alignat}
\end{theorem}

Dado un perfil estratégico mixto $\sigma$, tiene sentido preguntarse si el jugador $i$ está jugando de la mejor forma dadas las estrategias seleccionadas por los otros jugadores. A partir de esta pregunta, definimos el concepto de mejor respuesta para el jugador $i$ dado un perfil $\sigma_{-i}$ para los otros jugadores.

\begin{definition}
\label{def:mejor-respuesta}
Sea $i\in N$ un jugador, $\sigma_i$ una estrategia mixta para el jugador $i$, y $\sigma_{-i}$ un perfil estratégico mixto para el resto de los jugadores. Decimos que $\sigma_i$ es una \textbf{mejor respuesta} con respecto a $\sigma_{-i}$ si y s\'olo si
$u_i(\sigma_i,\sigma_{-i}) \geq u_i(\sigma'_i,\sigma_{-i})$ para toda estrategia mixta $\sigma'_i$ para el jugador $i$.
\end{definition}

Una mejor respuesta no es necesariamente única. En efecto, salvo el caso extremo en el que hay una única mejor respuesta, que como veremos debe ser una estrategia pura, el número de mejores respuestas es infinito. Cuando el soporte de una estrategia mixta que es mejor respuesta incluye dos o más estrategias puras (acciones), el agente debe ser indiferente a cualquiera de éstas y cualquier mezcla de estas  acciones también será mejor respuesta \cite{bib:tutorial-existence-nash}.

\begin{theorem}
\label{theo:mejor-respuesta}
Sea $\sigma^*_i$ una estrategia mixta para el jugador $i$ que es mejor respuesta a $\sigma_{-i}$. Cualquier estrategia mixta $\sigma_i$ para el jugador $i$ cuyo soporte sea un subconjunto del soporte de $\sigma^*_i$ es también una mejor respuesta a $\sigma_{-i}$.
\end{theorem}

En el juego piedra, papel o tijera, la ganancia esperada del jugador $i$ cuando utiliza la estrategia $\sigma = (\sigma_1, \sigma_2)$, viene dada por:
\begin{alignat}{1}
	u_i(\sigma)\ &=\ \sigma_i(\mathcal{R}) \left( \sigma_{-i}(\mathcal{R}) u_i(\mathcal{R}, \mathcal{R}) + \sigma_{-i}(\mathcal{P}) u_i(\mathcal{R}, \mathcal{P})+ \sigma_{-i}(\mathcal{S}) u_i(\mathcal{R}, \mathcal{S}) \right) \\ \nonumber
    &+\ \sigma_i(\mathcal{P}) \left(\sigma_{-i}(\mathcal{R}) u_1(\mathcal{P}, \mathcal{R}) + \sigma_{-i}(\mathcal{P}) u_i(\mathcal{P}, \mathcal{P})+ \sigma_{-i}(\mathcal{S}) u_i(\mathcal{P}, \mathcal{S})\right) \\ \nonumber
	&+\ \sigma_i(\mathcal{S}) \left(  \sigma_{-i}(\mathcal{R})  u_1(\mathcal{S}, \mathcal{R}) + \sigma_{-i}(\mathcal{P}) u_i(\mathcal{S}, \mathcal{P})+ \sigma_{-i}(\mathcal{S}) u_i(\mathcal{S}, \mathcal{S})\right)\,.
\end{alignat}

Al sustituir las utilidades de las estrategias puras y eliminar los términos nulos, se obtiene:
\begin{alignat}{1}
	u_i(\sigma)\
	&=\ \sigma_i(\mathcal{R}) \left(\sigma_{-i}(\mathcal{S}) - \sigma_{-i}(\mathcal{P}) \right) \\ \nonumber
    &+\ \sigma_i(\mathcal{P}) \left(\sigma_{-i}(\mathcal{R}) - \sigma_{-i}(\mathcal{S}) \right) \\ \nonumber
	&+\ \sigma_i(\mathcal{S}) \left(\sigma_{-i}(\mathcal{P}) - \sigma_{-i}(\mathcal{R}) \right)\,.
\end{alignat}

En particular, para la estrategia presentada previamente $\sigma = (\sigma_1, \sigma_2)$, con $\sigma_1 = \left( \frac{1}{2}, \frac{1}{2}, 0 \right)$ y $\sigma_2 = \left(\frac{1}{3}, \frac{1}{2}, \frac{1}{6} \right)$, las ganancias esperadas del primer y segundo jugador son iguales a:
\begin{alignat}{1}
    u_1(\sigma)\ &=\ \frac{1}{2} \left(\frac{1}{6} - \frac{1}{2}\right) + \frac{1}{2} \left( \frac{1}{3} -\frac{1}{6} \right) + 0 \left( \frac{1}{2} - \frac{1}{6} \right)\ =\ -\frac{1}{12} \\
    u_2(\sigma)\ &=\ \frac{1}{3}\left(0 -\frac{1}{2} \right) + \frac{1}{2} \left( \frac{1}{2} - 0 \right) + \frac{1}{6} \left( \frac{1}{2} - \frac{1}{2} \right)\ =\ \frac{1}{12}\,.
\end{alignat}

Calculemos ahora las mejores respuesta a $\sigma_1$ y $\sigma_2$. Supongamos que el jugador $1$ utiliza $\sigma_1$ y sea $\sigma^*_2 = (x, y, z)$ la mejor respuesta a $\sigma_1$. Entonces la ganancia esperada del jugador $2$ viene dada por:
\begin{alignat}{2}
u_2(\sigma_1, \sigma^*_2)\ &=\ x\left(0 -\frac{1}{2}\right) + y\left(\frac{1}{2} - 0\right) + z\left(\frac{1}{2} - \frac{1}{2}\right) \\
u_2(\sigma_1, \sigma^*_2)\ &=\ -\frac{1}{2}x + \frac{1}{2}y
\end{alignat}

Luego, la mejor respuesta a $\sigma_1$ se obtiene al maximizar $f(x, y) = -\frac{1}{2}x + \frac{1}{2}y$, con $x+y+z =1$ y $x, y, z \geq 0$. Es claro que la función se maximiza en dicho dominio cuando $x = z = 0$ y $y = 1$. Luego $\sigma^*_2 = (0, 1, 0)$; i.e., la estrategia en el que el jugador $2$ siempre elige papel. En este caso existe una única mejor respuesta a $\sigma_1 = \left(\frac{1}{2}, \frac{1}{2}, 0\right)$, cuyo soporte tiene un único elemento: $\text{support}(\sigma^*_2) = \{\mathcal{P}\}$.

De forma similar se obtiene que, si $\sigma^*_1 = (x, y, z)$ es mejor respuesta a $\sigma_2$, entonces $u_1(\sigma^*_1, \sigma_2) = -\frac{1}{3}x + \frac{1}{6}y + \frac{1}{6}z$. Note que en este caso se maximiza la función cuando $y = \alpha$ y $z = 1 -\alpha$, para cualquier $\alpha \in [0, 1]$. Luego, el jugador es indiferente ante las estrategias $\mathcal{P}$ y $\mathcal{S}$, por lo que existen infinitas estrategias que son mejor respuesta a $\sigma_1$. Finalmente se obtiene que $\sigma^*_1$ es mejor respuesta si y sólo si $\sigma^*_1 = (0, \alpha, 1 - \alpha)$ para cualquier  $\alpha \in [0, 1]$, i.e. $\sigma^*_1$ es mejor respuesta si y sólo si $\text{support}(\sigma^*_1) \subseteq \{\mathcal{P}, \mathcal{S} \}$.

\section{Equilibrio de Nash}
Cuando cada jugador juega con una mejor respuesta frente a las estrategias del resto de los jugadores se dice que tenemos un equilibrio de Nash. En un equilibrio de Nash ningún jugador puede mejorar su ganancia esperada cambiando su estrategia de forma aislada. Por otra parte, si el juego es finito, siempre existe al menos un equilibrio de Nash. Un juego es finito si el número de jugadores es finito, y si el conjunto de estrategias puras para cada jugador es también finito. El concepto de equilibrio de Nash es uno de los conceptos de solución más importantes en el área de teoría de juegos no cooperativos, y es el principal concepto de solución utilizado en el presente trabajo.

\begin{definition}
\label{def:equilibrio-nash} Un perfil estratégico mixto $\sigma$ es un \textbf{equilibrio de Nash} si y s\'olo si para todo jugador $i$, la estrategia $\sigma_i$ es mejor respuesta del jugador $i$ para $\sigma_{-i}$.
\end{definition}

\begin{theorem}[\cite{bib:tutorial-existence-nash}]
\label{theo:existencia-nash}
Todo juego finito tiene al menos un equilibrio de Nash.
\end{theorem}

Es importante destacar que el Teorema \ref{theo:existencia-nash} es cierto al considerar perfiles estratégicos mixtos, pero no al considerar únicamente perfiles estratégicos puros. No todos los juegos tienen algún perfil estratégico puro que sea equilibrio de Nash, como por ejemplo, el juego RPS. Para ver esto, supongamos que el primer jugador juega con alguna estrategia pura, digamos $\mathcal{R}$, luego, la única mejor respuesta a esta estrategia es jugar con la estrategia pura $\mathcal{P}$. Pero esto no es un equilibrio de Nash, pues ahora la mejor respuesta a la estrategia del segundo jugador es que el primer jugador juegue con la estrategia pura $\mathcal{S}$. Análogamente, cuando el primer o segundo jugador juegan con cualquier estrategia pura no se puede obtener un equilibrio de Nash.

En RPS existe un único equilibrio de Nash, que ocurre cuando $\sigma^*_1 = \sigma^*_2 = \left(\frac{1}{3}, \frac{1}{3}, \frac{1}{3} \right)$. En efecto, note que si el jugador $i$ utiliza $\sigma^*_i$, para cualquier estrategia $\sigma_{-i}$, la ganancia del jugador $-i$ viene dada por:
\begin{alignat}{1}
	u_{-i}(\sigma^*_i, \sigma_{-i})\ =\ \sigma_{-i}(\mathcal{R}) \left( \frac{1}{3} -  \frac{1}{3} \right) + \sigma_{-i}(\mathcal{P}) \left( \frac{1}{3} -  \frac{1}{3} \right) + \sigma_{-i}(\mathcal{S}) \left( \frac{1}{3} -  \frac{1}{3} \right)\ =\ 0 \,.
\end{alignat}

Luego, el jugador $-i$ es indiferente ante cualquier estrategia mixta, pues su ganancia esperada siempre es igual a $0$, y por lo tanto cualquier estrategia $\sigma_{-i}$ es mejor respuesta a $\sigma^*_i$. En particular $\sigma^*_1$ es mejor respuesta a $\sigma^*_2$ y $\sigma^*_2$ es mejor respuesta a $\sigma^*_1$ y el perfil estratégico $\sigma^* = (\sigma^*_1, \sigma^*_2)$ es un equilibrio de Nash.

\section{Equilibrio Correlacionado}
\label{section:equilibrio-correlacionado}

Aunque el equilibrio de Nash es uno de los principales conceptos de solución, es importante destacar que éste no garantiza el mejor resultado si los jugadores toman sus decisiones en conjunto. Si a los jugadores se les permite correlacionar sus acciones (es decir, trabajar en grupo), pueden existir estrategias con mayores ganancias para ellos. 
Este tipo de situaciones son las que considera la noción de equilibrio correlacionado que generaliza al equilibrio de Nash. Todo equilibrio de Nash es un equilibrio correlacionado, pero este último permite otras soluciones importantes \cite{bib:correlated-equilibrium}. La relación entre los conceptos de equilibro de Nash y correlacionado se muestra en los Teoremas \ref{theo:nash-correlacionado} y \ref{theo:correlacionado-nash}.

\begin{definition}
\label{def:equilibrio-correlacionado}
Una distribución $\psi\in\Delta(S)$ es un \textbf{equilibrio correlacionado} si y sólo si para cualquier jugador $i$, y para cualesquiera estrategias puras $x, y \in S_i$,
\begin{alignat}{1}
\label{eq:equilibrio-correlacionado}
\sum_{s_{-i}\in S_{-i}} \psi(x,s_{-i}) [ u_i(x,s_{-i}) - u_i(y,s_{-i})]\ \geq\ 0 \,.
\end{alignat}
\end{definition}

Si en la desigualdad \eqref{eq:equilibrio-correlacionado} se cambia el $0$ por un $\epsilon > 0$ se obtiene la definición de $\epsilon$-equilibrio correlacionado.


\begin{theorem}
\label{theo:nash-correlacionado}
Si $\sigma$ es un equilibrio de Nash, entonces $\sigma$ es un equilibrio correlacionado.
\end{theorem}

\begin{theorem}
\label{theo:correlacionado-nash}
Sea $\psi\in\Delta(S)$ un equilibrio correlacionado. Si $\psi$ se factoriza como $\psi=\prod_{i\in N} \sigma_i$ donde $\{\sigma_i\}_{i\in N}$ es un conjunto de estrategias mixtas para cada jugador (i.e., $\psi(s)=\prod_{i \in N} \sigma_i(s_i)$ para todo $s\in S$), entonces $\psi$ es un equilibrio de Nash.
\end{theorem}

A diferencia del conjunto de equilibrios de Nash, el cual es un conjunto matemáticamente complejo (un conjunto de puntos fijos), el conjunto de equilibrios correlacionados en un conjunto bastante simple. En particular, el conjunto de equilibrios correlacionado es un politopo (generalización de un polígono en $\mathbb{R}^N$) convexo (ver Teorema~\ref{theo:correlacionado-linealidad} abajo). Por lo tanto puede esperarse que existan procedimientos simples para calcular equilibrios correlacionados  \cite{bib:correlated-equilibrium}. En el Capítulo \ref{chapter:regret-matching} se presentan algunos procedimientos que permiten calcular equilibrios correlacionados, y estos procedimientos permiten calcular un equilibrio de Nash cuando los juegos cumplen con ciertas condiciones.

% \noindent\textcolor{red}{\bf *** Y esto nos va a llevar a calcular equilibrios de Nash? ***}

\begin{theorem}
\label{theo:correlacionado-linealidad}
Sean $\sigma$ y $\sigma'$ dos equilibrios correlacionados, y $\alpha$ un número real en $(0,1)$. Entonces, la distribución $\alpha\sigma + (1-\alpha)\sigma'$ es un equilibrio correlacionado.
\end{theorem}

\begin{example}[{\cite[p.~67]{bib:teoria-juegos-es}}]
\label{ex:batalla-sexos}
Juego \say{batalla de los sexos}. Considere una pareja María y José; ellos tendrán una cita por lo que deben elegir un evento para la cita. A María le gusta el ballet y a José el béisbol. Ellos prefieren ir juntos al mismo evento que ir a eventos diferentes. Sin embargo, cada uno se sentiría más feliz si deciden ir al evento de su preferencia. La Tabla~\ref{table:pagos-batallas-sexo} es la tabla de pagos correspondiente.
\end{example}

\begin{table}[h]
\begin{center}
\caption{Tabla de pagos del juego ``batalla de los sexos''.}
\label{table:pagos-batallas-sexo}
\begin{tabular}{ c c | c | c |}
 & \multicolumn{1}{c}{} & \multicolumn{2}{c}{José} \\
 & \multicolumn{1}{c}{} & \multicolumn{1}{c}{ballet} & \multicolumn{1}{c}{béisbol}  \\ \cline{3-4}
 \multirow{2}{*}{María}
 & ballet  & $2, 1$ & $0, 0$ \\ \cline{3-4}
 & béisbol & $0, 0$ & $1, 2$ \\ \cline{3-4}
\end{tabular}
\end{center}
\end{table}

En este caso sí existen equilibrios de Nash con estrategias puras. El perfil estratégico (ballet, ballet) es un equilibrio de Nash. En efecto, si José sabe que María siempre elige ballet, lo mejor que él puede hacer es ir al ballet con su compañera, asimismo, si María sabe que José siempre elegirá ballet, lo mejor que puede hacer ella es elegir también ballet. De forma análoga se obtiene que la estrategia (béisbol, béisbol) también es un equilibrio de Nash.

Un tercer equilibrio de Nash ocurre cuando cada jugador utiliza una estrategia tal que su oponente sea indiferente ante la elección de su propia estrategia (en relación a su utilidad). Es decir, José utiliza una estrategia tal que María obtenga siempre la misma ganancia esperada indiferentemente de la estrategia que ella utilice. Análogamente María utiliza una estrategia para la cual José siempre obtiene la misma ganancia sin importar lo que él elija.

Suponga que María (que será considerada el primer jugador) utiliza una estrategia $\sigma_1 = (x, 1-x)$ (donde la primera componente corresponde a la probabilidad de elegir ballet). Si José utiliza una estrategia $\sigma_2 = (\beta_1, \beta_2)$ entonces su ganancia esperada es igual a $u_2(\sigma_1, \sigma_2) = x\beta_1 + 2(1-x)\beta_2$. Luego, José es indiferente a los valores $\beta_1$ y $\beta_2$ cuando $x = 2(1-x)$, lo que ocurre cuando $x = \frac{2}{3}$. En efecto, si $\sigma_1 = \left(\frac{2}{3}, \frac{1}{3} \right)$, entonces la ganancia esperada de José siempre es igual a:
\begin{alignat}{1}
u_2(\sigma_1, \sigma_2)\ =\ \frac{2}{3}\beta_1 + 2\left(1-\frac{2}{3} \right)\beta_2\ =\ \frac{2}{3}(\beta_1 + \beta_2)\ =\ \frac{2}{3} \,.
\end{alignat}

Por otra parte, si José utiliza una estrategia $\sigma_2 = (y, 1-y)$ y María utiliza una estrategia $\sigma_1 = (\theta_1, \theta_2)$ la ganancia esperada para María es igual a $u_2(\sigma_1, \sigma_2) = 2y\theta_1 + (1-y)\theta_2$ y ella será indiferente a la elección de $\theta_1$ y $\theta_2$ cuando $2y = 1-y$, i.e., cuando $y = \frac{1}{3}$. Note que cuando $\sigma_2 = \left(\frac{1}{3}, \frac{2}{3} \right)$, entonces:
\begin{alignat}{1}
u_2(\sigma_1, \sigma_2)\ =\ \frac{2}{3}\beta_1 + 2\left(1-\frac{2}{3} \right)\beta_2\ =\ \frac{2}{3}(\beta_1 + \beta_2)\ =\ \frac{2}{3} \,.
\end{alignat}

Luego se tiene que $\sigma = \left(\left(\frac{2}{3}, \frac{1}{3}\right), \left(\frac{1}{3}, \frac{2}{3}\right)\right)$ es un equilibrio de Nash. Sin embargo, ninguna de las $3$ soluciones parece realmente satisfactoria. Las $2$ primeras son claramente más beneficiosas para uno de los jugadores y la última, aunque podría parecer más justa, proporciona un ganancia esperada menor que las estrategias anteriores para ambos jugadores. ?`Será posible que los jugadores cooperen entre sí para obtener mejores resultados?

Los jugadores podrían realizar lo siguiente: lanzar una moneda, si el resultado es cara van al ballet y si es sello van al béisbol. En este caso ya no se limitan a estrategias mixtas y están eligiendo una distribución de probabilidad sobre todos los perfiles estratégicos. La distribución propuesta es $\psi \in \Delta(S)$, tal que $\psi(\text{ballet}, \text{ballet}) = \psi(\text{beisbol}, \text{beisbol}) = \frac{1}{2}$ y $\psi(\text{ballet}, \text{beisbol}) = \psi(\text{beisbol}, \text{ballet}) =0$. $\psi$ es un equilibrio correlacionado y ambos jugadores obtendrían una ganancia esperada de $\frac{3}{2}$.