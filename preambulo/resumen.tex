\begin{resumen}
    La teoría de juegos se encarga de estudiar la toma de decisiones estratégicas de individuos racionales en estructuras denominadas juegos. Los juegos no deterministas con información incompleta son aquellos que tienen azar y hay información oculta para los jugadores, como por ejemplo el juego de póker. En este trabajo de grado se estudian dos modelos diferentes para este tipo de juegos: forma normal y forma extensiva; algunos conceptos de solución, como equilibrio de Nash y equilibrio correlacionado; y algoritmos para resolver los juegos planteados. La forma normal se utiliza para representar juegos en los cuales los jugadores eligen una única acción en forma simultánea y la forma extensiva se utiliza para representar juegos secuenciales donde los jugadores toman decisiones por turnos. Se limita la investigación a juegos de dos jugadores con suma cero, para los cuales algoritmos como \textit{Regret Matching} y \textit{Conterfactual Regret Minimization} (CFR) permiten calcular una aproximación a un equilibrio de Nash, el principal concepto de solución utilizado. Estos algoritmos fueron implementados para diferentes juegos captados por los modelos mencionados. Los juegos en forma normal presentados fueron: piedra, papel o tijera, \textit{matching pennies}, ficha vs. dominó y coronel Blotto, para los cuales se calculó una aproximación a un equilibrio de Nash mediante el algoritmo de \textit{Regret Matching}. Los juegos en forma extensiva estudiados fueron: \textit{One Card Póker} (OCP), dudo (un juego de dados), y dominó para dos personas. Estos juegos fueron parametrizados según el número de cartas, dados o piezas, entre otros elementos; obteniendo múltiples instancias para cada uno de ellos. Para cada instancia se calculó una aproximación a un equilibrio de Nash y se midió el error mediante la explotabilidad. Un juego fue considerado resuelto si la explotabilidad de la estrategia obtenida fue menor que un límite establecido. En el juego OCP fue posible resolver todas las instancias planteadas (utilizando hasta 5.000 cartas). En el juego dudo se resolvieron instancias con dados de hasta 4 caras y 2 dados por jugador. Por último, la instancia más grande resuelta en el juego de dominó incluye fichas que tienen hasta 3 puntos por lado y una distribución inicial de 3 fichas por jugador.
    
    \vfill
    \textbf{Palabras claves:} juegos, forma normal, forma extensiva, no determinismo, información incompleta, estrategias.
\end{resumen}