\chapter*{Conclusiones y Recomendaciones}

En esta investigación se estudiaron los modelos para representar juegos no deterministas con información incompleta, siendo la forma normal el modelo utilizado para representar juegos de una única acción y la forma extensiva el modelo utilizado para representar juegos secuenciales. Además, se estudiaron diferentes conceptos de solución y se utilizó el equilibrio de Nash, como el concepto de solución principal en el caso de juegos de dos jugadores con suma cero, que fueron los considerados en esta investigación.

Para encontrar equilibrios de Nash en juegos en forma normal se utilizaron procedimientos de \textit{Regret Matching} y se evaluaron las estrategias mediante la explotabilidad. Los procedimientos fueron probados en 4 juegos diferentes: piedra, papel o tijera, \textit{matching pennies}, ficha vs.\ dominó y una instancia del juego coronel Blotto. En todos los juegos se encontraron aproximaciones al equilibrio de Nash con una explotabilidad no mayor que 0,010 (lo que representa el  1\% de la mínima ganancia positiva en todos los casos) por lo que se consideran resueltos.

Para encontrar equilibrios de Nash en juegos en forma extensiva se utilizó el algoritmo \textit{chance-sampled} CFR. Los juegos estudiados presentan \textit{perfect recall}, condición que garantiza la convergencia en el algoritmo en juegos de dos jugadores de suma cero. Este algoritmo fue probado en tres clases de juegos: \textit{One Card Poker} (OCP), dudo (un juego de dados) y una versión del juego de dominó para dos personas.

El juego OCP fue parametrizado por el número de cartas iniciales, representándose con OCP($N$). En este juego todas las instancias probadas (usando entre 3 y 5.000 cartas) se consideraron resueltas.

El juego dudo, representado con Dudo$(K, D_1, D_2)$ fue parametrizado por la cantidad de caras de los dados ($K$) y el número de dados de cada jugador ($D_1, D_2$). Con $10$ horas de entrenamiento fue posible resolver todas las instancias con dados de hasta 5 caras y 2 dados por jugador, i.e., $K \leq 4$ y $D_1, D_2 \leq 2$, con excepción de las instancias  Dudo$(4, 2, 2)$ y Dudo$(5, 2, 2)$, en esta última instancia la explotabilidad fue mayor que $15\%$. Para $K = 6$ sólo fue posible resolver la instancia Dudo$(6, 1, 1)$ en dicho tiempo. Por otra parte, con $200$ horas de entrenamiento, la única instancia que no fue posible resolver fue la instancia Dudo$(5,2,2)$, pero la explotabilidad bajó a menos de $2\%$.

El juego de dominó fue parametrizado por el mayor doble presente en el mazo $M$ y la cantidad de fichas repartidas a cada jugador al inicio del juego $N$, representándose con Domino$(M, N)$. Se consideraron la instancia Domino$(2, 2)$ y las instancias Domino$(3, N)$ con $N \leq 4$. Con $10$ horas de entrenamiento se resolvieron todas las instancias propuestas con la excepción de la instancia Domino$(3, 4)$ donde se obtuvo una explotabilidad de $1,4871\%$, sin embargo, fue posible resolver esta última instancia con $200$ horas de entrenamiento.

El primer aporte realizado con este trabajo de grado consistió en la recopilación y formalización de diversas propiedades y teoremas, los cuales se demostraron rigurosamente para su verificación y futuras referencias (ver Apéndice~\ref{apex:chapter:pruebas}).

Por otra parte, se realizó una implementación propia del algoritmo de CFR con muestreo en los nodos de azar, que puede ser utilizada con cualquier juego que se desee estudiar, cuya definición se recibe como parámetro en el algoritmo. Además, se proporciona una interfaz con las funciones que deben ser implementadas para definir un juego y poder utilizar el algoritmo. De esta forma, es posible utilizar la implementación realizada para estudiar nuevos juegos; para lograrlo se debe crear una clase derivada de la clase \textit{Game} y definir las funciones virtuales que permiten recorrer el árbol de forma implícita. También se realizó la implementación del algoritmo \textit{Generalized Expectimax Best Response} (GEBR) que permite calcular la explotabilidad de una estrategia para un juego en particular.

Otro aporte que se puede destacar es el estudio por primera vez de una versión del juego de dominó. Se utilizó una versión de $2$ jugadores descrita en el Capítulo~\ref{chapter:cfr}. Una dificultad adicional que surge en este juego es que las acciones de los jugadores son parcialmente observables, es decir, un jugador no sabe cuales son las acciones disponibles de su oponente, ya que el conjunto de acciones posibles no es un conjunto fijo como ocurre, por ejemplo, en el juego de póker. Fue posible encontrar aproximaciones a estrategias óptimas en varias instancias del juego de dominó, una de las cuales fue probada en una aplicación web, donde puede ser probada contra personas reales o contra el equilibrio de Nash.

Futuras investigaciones pueden estar enfocadas en el juego de dominó en particular intentando resolver instancias más grandes. Para esto, se recomienda utilizar abstracciones (juegos más pequeños), que pueden ser obtenidos mediante la unión de varios conjuntos de información en uno solo. La idea es unir conjuntos de información similares o idénticos, tales que la información que se pierda no sea importante en cuanto a las estrategias \cite[pp.~71-72]{bib:thesis-marc-lanctot}. Cabe destacar que estas abstracciones pueden o no tener \textit{perfect recall}, y CFR no garantiza calcular una aproximación a un equilibrio de Nash en un juego con \textit{imperfect recall}. Sin embargo, aún sin las garantías teóricas, esta técnica ha sido utilizada para calcular estrategias fuertes en versiones del juegos de póker que pueden incluso superar sus contrapartes con \textit{perfect recall} \cite{bib:imperfect-recall}.

Posibles abstracciones que pueden ser utilizadas en el juego de dominó consisten en considerar únicamente la secuencia final después de cada jugada y no el orden específico en que las fichas fueron colocadas; o simplemente considerar dicha secuencia como un conjunto, sabiendo cuales son los números en los extremos y la información sobre las fichas jugadas por cada jugador. Incluso se puede considerar una abstracción donde no se distingan qué fichas fueron jugadas por cada jugador. La motivación para proponer estas abstracciones consiste en que las jugadas posibles dependen únicamente de las manos actuales, las fichas restantes y las fichas en los extremos. Sin embargo, se pierde información sobre las secuencia de las jugadas del oponente que puede ser relevante en ciertas circunstancias.

Finalmente, otro posible enfoque para este juego consiste en intentar resolver el juego de dominó para 4 jugadores con equipos de 2 personas, según las reglas clásicas venezolanas, considerando cada equipo como un sólo jugador con \textit{imperfect recall} y observar si es posible calcular una aproximación a un equilibrio de Nash con el algoritmo CFR (o alguna variación del mismo).
