\chapter{Forma Normal y Programación Lineal}
\label{apex:chapter:programacion-lineal}

En el Capítulo \ref{chapter:regret-matching} se afirma que para todo juego de dos jugadores de suma cero en forma normal, encontrar el equilibrio de Nash es equivalente a un problema de programación lineal. En esta sección se detalla cómo obtener el programa de programación lineal a un juego dado en forma normal \cite[pp.~228-233]{bib:pl-chvatal}.

Todo juego en forma normal puede ser descrito por una tabla $N$ dimensional, donde $N$ es el número de jugadores. En particular un juego de dos jugadores puede ser representado por una matriz, y además, si el juego es de suma $0$, basta con definir la utilidad del primer jugador. Luego, este tipo de juegos pueden ser representados con una matriz $\mathbf{A} = (a_{ij}) \in \mathbb{R}^{m \times n}$. Luego, el elemento $a_{ij}$ de la matriz $A$ es la utilidad obtenida por el primer jugador si éste utiliza la $i$-ésima estrategia y su oponente utiliza la $j$-ésima estrategia.

Si el jugador $1$ juega con una estrategia $ \mathbf{x} = (x_1, x_2, ..., x_m))$ y el jugador $2$ con una estrategia $\mathbf{y} = (y_1, y_2, ..., y_n)$, entonces la ganancia esperada viene dada por
\begin{alignat}{1}
u_1(\mathbf{x}, \mathbf{y}) = \sum_{i = 1}^m \sum_{j = 1}^n a_{ij}x_iy_j,
\end{alignat}
lo cual se puede escribir matricialmente como $\mathbf{x}^\intercal \mathbf{A} \mathbf{y}$.

Del Teorema \ref{theo:von-neumann} se sabe que el valor \textit{maximin} y \textit{minimax} es igual al valor del juego, al igual que los perfiles estratégicos \textit{maximin} y \textit{minimax} coinciden con equilibrios de Nash. Luego, un equilibrio de Nash $(x^*, y^*)$ cumple que:
\begin{alignat}{1}
x^* \in \argmax_{\mathbf{x}}{\min_{\mathbf{y}}{\mathbf{x}^\intercal \mathbf{A} \mathbf{y}}} \ \text{y} \label{apex:eq:maximin} \\
y^* \in \argmin_{\mathbf{y}}{\max_{\mathbf{x}}{\mathbf{x}^\intercal \mathbf{A} \mathbf{y}}} \,.
\label{apex:eq:minimax}
\end{alignat}

Ya que la ecuación \ref{apex:eq:maximin} es la definición de estrategia \textit{maximin} para el primer jugador y la ecuación \ref{apex:eq:minimax} es la definición \textit{minimax} para el segundo jugador. La primera observación importante proviene del Teorema~\ref{theo:mejor-respuesta}, ya que como corolario se obtiene que, para cualquier estrategia, siempre existe una mejor respuesta cuyo soporte tiene un único elemento. De esto último se obtiene que, dado $\mathbf{x} \in \mathbb{R}^m$ e $\mathbf{y} \in \mathbb{R}^n$:
\begin{alignat}{1}
\min_{\mathbf{y}}{\mathbf{x}^\intercal \mathbf{A} \mathbf{y}} = \min_{j} \sum_{i=1}^n a_{ij} x_i\\
\max_{\mathbf{x}}{\mathbf{x}^\intercal \mathbf{A} \mathbf{y}} = \max_{i} \sum_{i=1}^m a_{ij} y_j
\end{alignat}

Luego, los problemas que se quieren resolver son equivalentes a:
\begin{alignat}{1}
\max_{x} \min_{j} \sum_{i=1}^n a_{ij} x_i \label{apex:eq:maximin-problema} \\
\min_{y} \max_{i} \sum_{i=1}^m a_{ij} y_j \label{apex:eq:minimax-problema}
\end{alignat}

Sujetos a $x_i, y_j \geq 0$ y a $\sum_{i=1}^m x_i = 1$ y $\sum_{j=1}^n y_j = 1$. La observación clave consiste en ver la equivalencia de los problemas \ref{apex:eq:maximin-problema} y \ref{apex:eq:minimax-problema} con los problemas de programación lineal \ref{apex:eq:maximin-problema2} y \ref{apex:eq:minimax-problema2}, respectivamente \cite[p.~232]{bib:pl-chvatal}.
\begin{alignat}{6}
\label{apex:eq:maximin-problema2}
& \max\  & z\ &  & & & &  \\ \nonumber
& \text{sujeto a}\ & & & & & & \\  \nonumber 
& & z\ & -\ & \sum_{i=1}^m a_{ij} x_i \ \leq\  \ 0 \ & \ \ (j=1, 2, ..., n) \\\nonumber
& &    &    & \sum_{i=1}^m x_i\ \ =\           \ 1 \ & \\ \nonumber
& &    &    & x_i  \ \geq\                     \ 0 \ & \ (i=1, 2, ..., m) \\
\label{apex:eq:minimax-problema2} 
& \min\  & w\ &  & & & &  \\ \nonumber
& \text{sujeto a}\ & & & & & & \\  \nonumber 
& & w\ & -\ & \sum_{j=1}^n a_{ij} y_j \ \geq\  \ 0 \ & \ \ (i=1, 2, ..., m) \\\nonumber
& &    &    & \sum_{j=1}^n y_j\ \ =\           \ 1 \ & \\ \nonumber
& &    &    & x_i  \ \geq\                     \ 0 \ & \ \ (j=1, 2, ..., n) \,. \nonumber
\end{alignat}

Para ver la equivalencia, note que cualquier solución óptima $z^*, x_1^*, x_2^*, ..., x_m^*$ de \ref{apex:eq:maximin-problema2} satisface al menos una restricción con el signo de igualdad y por lo tanto $z^* = \min_j \sum_i a_{ij}x_i^*$. De forma similar, cualquier solución óptima $w^*, y_1^*, y_2^*, ..., y_n^*$ de \ref{apex:eq:minimax-problema2} satisface al menos una restricción con el signo de igualdad y por lo tanto $w^* = \min_i \sum_j a_{ij}y_j$. Por último, note que \ref{apex:eq:maximin-problema2} y \ref{apex:eq:minimax-problema2} son problemas duales entre sí, lo cual confirma que $z^* = w^*$.

Con respecto a los juego en forma extensiva, si bien se podrían representar en forma normal para resolverlos con programación lineal, esto no es práctico, pues la forma normal es exponencialmente más grande que la forma extensiva. En \cite{bib:algoritmo-pl} se propone un modelo para calcular un equilibrio de Nash mediante programación lineal con una complejidad polinómica respecto al número de nodos del árbol, esto fue el estado del arte hasta que se desarrolló el algoritmo \textit{Counterfactual Regret Minimization} (CFR).

A continuación, se presentan loa problemas de programación lineal asociados a los juegos \textit{matching pennies}, piedra, papel o tijera y ficha vs. dominó; presentados en el Capítulo~\ref{chapter:regret-matching}, con una solución primal y dual.

\begin{itemize}

\item \textbf{\textit{Matching Pennies}}
\begin{alignat}{7}
\label{apex:eq:pl-matching-pennies}
\max\ & z & & & &\\ \nonumber
\text{sujeto a}\ & & & & &\\ \nonumber
    &    &    & x_1\ & +\ & x_2\  =\    1 \\ \nonumber
    & z\ & -\ & x_1\ & +\ & x_2\  \leq\ 0 \\ \nonumber
    & z\ & +\ & x_1\ & -\ & x_2\  \leq\ 0 \\ \nonumber
    &    &    & x_1, &    & x_2\ \geq\ 0 \,. \nonumber
\end{alignat}

La única solución es:
\begin{alignat}{1}
(z^*, x_1^*, x_2^*)\ =\ (w^*, y^*_1, y^*_2)\ =\ \left(0, \frac{1}{2}, \frac{1}{2} \right) \,.
\end{alignat}

\item \textbf{Piedra, papel o tijera}
\begin{alignat}{9}
\label{apex:eq:pl-RPS}
\max\  & z\ &  & & & & & \\ \nonumber
\text{sujeto a}\ & & & & & & & \\  \nonumber 
&    &    & x_1\ & +\ & x_2\ & +\ & x_3\ & \ =\     & \ 1 \\ \nonumber
& z\ &    &      & +\ & x_2\ & -\ & x_3\ & \ \leq\  & \ 0 \\ \nonumber
& z\ & -\ & x_1\ &    &      & +\ & x_3\ & \ \leq\  & \ 0 \\ \nonumber
& z\ & +\ & x_1\ & -\ & x_2\ &    &      & \ \leq\  & \ 0 \\ \nonumber
&    &    & x_1, &    & x_2, &    & x_3\ & \ \geq\  & \ 0  \,.  \nonumber
\end{alignat}

La única solución es:
\begin{alignat}{1}
(z^*, x^*_1, x^*_2, x^*_4, x^*_5, x^*_6, x^*_6, x^*_7)\ =\ \left(\frac{1}{3}, \frac{1}{3}, \frac{1}{3}, 0, 0, 0, 0, \frac{1}{3}\right) \,, \\
(w^*, y^*_1, y^*_2, y^*_3,  y^*_4, y^*_5, y^*_6)\ =\ \left(\frac{1}{3}, \frac{1}{3}, 0, \frac{1}{3}, 0, \frac{1}{3}, 0\right) \,.
\end{alignat}

\item \textbf{Ficha vs. dominó}
\begin{alignat}{10}
\label{apex:eq:pl-domino}
\max\ &z\ & & & & & & & & & & & & & & & & \\ \nonumber
\text{sujeto a}\ & & & & & & & & & & & & & & & & \\ \nonumber 
	&   &   & x_1\ & +\ & x_2\ & +\ & x_3\ & +\ & x_4\ & +\ & x_5\ & +\ & x_6\ & +\ & x_7\ & \ =\    & \ 1 \\ \nonumber
    &z\ &+\ & x_1\ & -\ & x_2\ & -\ & x_3\ & -\ & x_4\ & +\ & x_5\ & -\ & x_6\ & -\ & x_7\ & \ \leq\ & \ 0 \\ \nonumber
    &z\ &+\ & x_1\ & -\ & x_2\ & +\ & x_3\ & -\ & x_4\ & -\ & x_5\ & +\ & x_6\ & -\ & x_7\ & \ \leq\ & \ 0 \\ \nonumber
    &z\ &-\ & x_1\ & -\ & x_2\ & +\ & x_3\ & -\ & x_4\ & -\ & x_5\ & -\ & x_6\ & +\ & x_7\ & \ \leq\ & \ 0 \\ \nonumber
    &z\ &-\ & x_1\ & +\ & x_2\ & -\ & x_3\ & -\ & x_4\ & +\ & x_5\ & -\ & x_6\ & -\ & x_7\ & \ \leq\ & \ 0 \\ \nonumber
    &z\ &-\ & x_1\ & +\ & x_2\ & -\ & x_3\ & +\ & x_4\ & -\ & x_5\ & +\ & x_6\ & -\ & x_7\ & \ \leq\ & \ 0 \\ \nonumber
    &z\ &-\ & x_1\ & -\ & x_2\ & -\ & x_3\ & +\ & x_4\ & -\ & x_5\ & -\ & x_6\ & +\ & x_7\ & \ \leq\ & \ 0 \\ \nonumber
    &   &   & x_1, &    & x_2, &    & x_3, &    & x_4, &    & x_5, &    &x_6,  &    & x_7, & \ \geq\ & \ 0 \,.
\end{alignat}

Una de las soluciones es:
\begin{alignat}{1}
(z^*, x^*_1, x^*_2, x^*_4, x^*_5, x^*_6, x^*_6, x^*_7)\ =\ \left(\frac{1}{3}, \frac{1}{3}, \frac{1}{3}, 0, 0, 0, 0, \frac{1}{3}\right) \,, \\
(w^*, y^*_1, y^*_2, y^*_3,  y^*_4, y^*_5, y^*_6)\ =\ \left(\frac{1}{3}, \frac{1}{3}, 0, \frac{1}{3}, 0, \frac{1}{3}, 0\right) \,.
\end{alignat}
\end{itemize}