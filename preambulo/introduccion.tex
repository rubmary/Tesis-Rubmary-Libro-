\chapter*{Introducción}

La teoría de juegos puede ser definida como el estudio de modelos matemáticos de conflicto y cooperación entre agentes que deben tomar decisiones de forma racional e inteligente \cite[p.~1]{bib:game-theory-book}; estos modelos se denominarán \textbf{juegos}. Esta disciplina tiene aplicaciones en diversas áreas, incluyendo ciencias sociales, economía, matemática y ciencias de la computación.

Uno de los principales pioneros de esta disciplina fue John von Neumann, con su publicación \textit{Zur Theorie der Gesellschaftsspiele} (Sobre la Teoría de Juegos) en el año 1928 \cite{bib:von-neumann}. Asimismo, John Forbes Nash Jr. con su publicación \textit{Non-cooperative Games} (Juegos no Cooperativos) en el año 1951 \cite{bib:nash}, introduce importantes conceptos, entre los cuales se encuentra el concepto de solución que hoy en día se conoce como equilibrio de Nash.

Aunque hay diferentes tipos de juegos, este trabajo se enfoca en juegos no deterministas con información incompleta. Con no determinismo se hace referencia a que los juegos incluyen incertidumbre probabilística, esta incertidumbre puede ocurrir, por ejemplo, al lanzar una moneda, repartir cartas de forma aleatoria o lanzar dados. Por otra parte, un juego con información incompleta permite modelar situaciones donde los jugadores tienen información parcial sobre algunas de las acciones que ya han sido tomadas \cite[p.~199]{bib:course-game-theory}.

El juego de póker (con sus diferentes versiones) es uno de los juegos más estudiados en esta categoría. Note que es un juego no determinista ya que se reparten cartas de forma aleatoria al inicio del mismo. Por otra parte, cada jugador desconoce las cartas que poseen los demás jugadores, por lo que poseen información parcial de la distribución inicial de las cartas. En contraste, juegos como el ajedrez, las damas o \textit{go}, son todos juegos deterministas (no hay elementos de azar) y además con información completa, pues todos los jugadores saben lo que ha ocurrido durante el juego y no hay información oculta entre ellos.

Uno de los retos para esta categoría de modelos consiste en determinar qué significa que un juego sea resuelto o que un jugador juegue de forma óptima. Para esto es necesario introducir el concepto de estrategias, las cuales indican las acciones o planes de acción que tomarán los jugadores en un momento determinado \cite[p.~24]{bib:teoria-juegos-es}. Luego, resolver un juego puede tener diferentes significados acorde al concepto de solución que se utilice, siendo el equilibrio de Nash uno los más importantes y el utilizado en el presente trabajo. Es importante destacar que en un equilibrio de Nash las acciones de los jugadores no son necesariamente deterministas, es decir, un jugador puede tomar decisiones diferentes ante el mismo escenario.

Por otra parte, Hart y Mas-Colell (2000) introducen el concepto de \textit{regret matching} \cite{bib:correlated-equilibrium}, en el cual los jugadores alcanzan un equilibrio teniendo en cuenta el \say{arrepentimiento} de sus jugadas previas, el cual se mide con una métrica denominada \textit{regret}, y haciendo las futuras jugadas proporcionales al \textit{regret} positivo. Este concepto es la base para el algoritmo \textit{Counterfactual Regret Minimization} (CFR), propuesto por Zinkevich, Johanson, Bowling y Piccione (2007) que permite encontrar una aproximación del equilibrio de Nash en cierto tipo de juegos con información incompleta, que sean de dos jugadores con suma cero \cite{bib:cfr}.

Dentro de este contexto, el objetivo de este proyecto de grado es comprender los conceptos en el área de juegos de dos personas que involucran información incompleta y no determinismo, así como implementar los algoritmos de \textit{Regret Matching} y CFR, realizando experimentos sobre distintos juegos que son capturados por el modelo. Con el fin de alcanzar el objetivo general se proponen los siguiente objetivos específicos:
\begin{itemize}
    \item Comprender los diferentes modelos de juegos y los elementos que los componen. Incluyendo juegos en forma normal y juegos en forma extensiva.
    \item Comprender los diferentes conceptos de solución para el tipo de juegos, como equilibrio correlacionado y equilibrio de Nash.
    \item Comprender los resultados teóricos más relevantes, y sus demostraciones, en relación a los modelos de juegos estudiados y los algoritmos implementados.
    \item Implementar los algoritmos \textit{Regret Matching} y \textit{Conterfactual Regret Minimization} que permiten encontrar equilibrios de Nash para el tipo de juego planteado.
    \item Implementar una clase general que permita representar los juegos que se quieren estudiar (independientemente de las reglas específicas de cada juego), así como diferentes juegos concretos que sean captados por el modelo.
    \item Realizar experimentos sobre los juegos propuesto utilizando los algoritmos implementados.
    \item Evaluar las estrategias obtenidas en cada uno de los juegos implementados.
\end{itemize}

Este libro se estructura en 5 capítulos. El Capítulo \ref{chapter:forma-normal} contiene el marco teórico de los juegos en forma normal o estratégica. Se presenta una definición formal de este tipo de juegos y los elementos que los componen. También se presentan dos conceptos de solución importantes: equilibrio de Nash y equilibrio correlacionado. El Capítulo \ref{chapter:juegos-forma-extensiva} contiene el marco teórico de los juegos en forma extensiva, se presentan los elementos en este tipo de juegos y se comparan con los juegos en forma normal. Además, se introduce una clasificación dentro de este tipo de juegos: juegos con \textit{perfect recall} o con \textit{imperfect recall}. Ambos capítulos contienen diversos ejemplos que ilustran los conceptos introducidos.

El Capítulo \ref{chapter:explotabilidad} presenta las propiedades que tienen los juegos de dos jugadores de suma cero, y explica por qué el equilibrio de Nash es importante en este tipo de juegos. Además, introduce dos nuevos conceptos de solución: estrategias \textit{minimax} y \textit{maximin}. Por último, se explica el concepto de explotabilidad que es la métrica que se utiliza para evaluar las estrategias obtenidas de forma experimental en los juegos.

El Capítulo \ref{chapter:regret-matching} presenta tres procedimientos que utilizan \textit{Regret Matching}, los cuales conducen a un equilibrio de Nash cuando los juegos son de dos jugadores de suma cero. Además, se presentan 4 juegos en forma normal y los resultados experimentales que se obtienen al aplicar los procedimientos sobre ellos. El Capítulo \ref{chapter:cfr} presenta el algoritmo CFR y una familia de este tipo de algoritmo, denominada \textit{Monte Carlo CFR} (MCCFR). Este capítulo también incluye 3 clases de juegos en forma extensiva y los resultados obtenidos al aplicar una versión de MCCFR sobre ellos. Finalmente, se presentan las conclusiones y las recomendaciones de este proyecto para las investigaciones futuras en el área.

La implementación de los procedimientos de \textit{Regret Matching}, junto con los resultados experimentales reportados en esta tesis se encuentran de forma pública \url{https://github.com/rubmary/regret-matching}. Similarmente, la implementación de los juegos y el algoritmo CFR, junto a las estrategias obtenidas y resultados experimentales se encuentran en \url{https://github.com/rubmary/cfr}.

Adicionalmente se desarrolló una aplicación web que permite observar la estrategia obtenida y la cual está disponible en \url{https://github.com/rubmary/domino-app}. Es importante destacar y agradecer la contribución de Samuel Nacache para el desarrollo de la interfaz gráfica de la aplicación.