\section{Detalles de implementación}
Los algoritmos y la representación de los juegos fueron implementados en el lenguaje de programación C++. Para la representación de los juegos se utilizó una clase abstracta llamada \textit{Game}, que recibe como template los tipos para el estado, las acciones, las propiedades, los conjunto de información y el Hash del juego específico.

Esta clase contiene las funciones virtuales necesarias para recorrer el árbol del juego de forma \textbf{implícita}, tales como: \textit{actions}, que retornan las acciones del juegos en el estado actual, \textit{update\_state}, que actualiza el estado del juego dada una acción a realizar, \textit{terminal\_state} que indica si un estado es terminal o no, \textit{utiliy} que retorna la utilidad en un estado terminal, entre otras. Los algoritmos CFR y GEBR utilizan esta clase abstracta en su implementación.

Para cada tipo de juego, se creó una clase derivada de la clase \textit{Game}, donde se implementaron las funciones según las reglas de cada juego. De esta forma se puede utilizar la misma implementación de los algoritmos para todos los juegos.

Cabe destacar que todos los juegos fueron representados mediantes árboles con la raíz como único nodo de azar. Algunos juegos tienen esta representación de forma natural, por ejemplo, el Kuhn Poker, ya que las cartas se reparten al inicio y luego se juega acorde a esa distribución, sin volver a introducir ninguna jugada aleatoria. Otros juegos pueden tener nodos de azar distintos a la raíz, sin embargo siempre es posible transformarlos a un árbol que represente el mismo juego donde todos los nodos de azar son condensados en la raíz y cada hijo de la raíz representa una elección por cada uno de los nodos de azar del árbol original. En esta representación se asume que todas las decisiones aleatorias se toman al inicio del juego.

La clase \textit{Game} y todos los algoritmos se implementados suponiendo la raíz como único nodo de azar del juego.