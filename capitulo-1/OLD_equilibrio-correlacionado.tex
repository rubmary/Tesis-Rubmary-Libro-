\section{Equilibrio Correlacionado}

Aunque el equilibrio de Nash es uno de los principales conceptos de solución, es importante destacar que éste no garantiza el mejor resultado si los jugadores toman sus decisiones en conjunto. Si a los jugadores se les permite correlacionar sus acciones (es decir, trabajar en grupo), pueden existir estrategias con mayores ganancias para ellos. 
Este tipo de situaciones son las que considera la noción de equilibrio correlacionado que generaliza al equilibrio de Nash \cite{bib:correlated-equilibrium}. Todo equilibrio de Nash es un equilibrio correlacionado, pero este último permite otras soluciones importantes \cite{bib:correlated-equilibrium}. La relación entre los conceptos de equilibro de Nash y correlacionado se muestra en los Teoremas \ref{theo:nash-correlacionado} y \ref{theo:correlacionado-nash}.

\begin{definition}
\label{def:equilibrio-correlacionado}
Una distribución $\psi\in\Delta(S)$ es un \textbf{equilibrio correlacionado} si y sólo si para cualquier jugador $i$, y para cualesquiera estrategias puras $x, y \in S_i$,
\begin{alignat}{1}
\label{eq:equilibrio-correlacionado}
\sum_{s_{-i}\in S_{-i}} \psi(x,s_{-i}) [ u_i(x,s_{-i}) - u_i(y,s_{-i})]\ \geq\ 0 \,.
\end{alignat}
\end{definition}

Si en la desigualdad \eqref{eq:equilibrio-correlacionado} se cambia el $0$ por un $\epsilon > 0$ se obtiene la definición de $\epsilon$-equilibrio correlacionado.


\begin{theorem}
\label{theo:nash-correlacionado}
Si $\sigma$ es un equilibrio de Nash, entonces $\sigma$ es un equilibrio correlacionado.
\end{theorem}
\begin{proof}
Sea $\sigma$ un equilibrio de Nash, sean $x,y\in S_i$ estrategias puras distintas cualesquiera para el jugador $i$, y sea $\sigma'_i$ una estrategia mixta cualquiera para el jugador $i$. Por el Lema~\ref{lemma:2},
\begin{alignat}{1}
  \sigma_i(x) \sum_{s_{-i}} u_i(x,s_{-i}) \sigma_{-i}(s_{-i})\ \geq\  \sigma_i(x)\sum_{s_{-i}} u_i(y,s_{-i}) \sigma_{-i}(s_{-i}) \,.
\end{alignat}
Es decir,
\begin{alignat}{1}
  0\ \leq\ \sigma_i(x) \sum_{s_{-i}} \sigma_{-i}(s_{-i}) [u_i(x,s_{-i}) - u_i(y,s_{-i})]\ =\ \sum_{s_{-i}} \sigma(x,s_{-i}) [u_i(x,s_{-i}) - u_i(y,s_{-i})] \,.
\end{alignat}
Luego, $\sigma$ es un equilibrio correlacionado.
\end{proof}

\begin{theorem}
\label{theo:correlacionado-nash}
Sea $\psi\in\Delta(S)$ un equilibrio correlacionado. Si $\psi$ se factoriza como $\psi=\prod_{i\in N} \sigma_i$ donde $\{\sigma_i\}_{i\in N}$ es un conjunto de estrategias mixtas para cada jugador (i.e., $\psi(s)=\prod_{i \in N} \sigma_i(s_i)$ para todo $s\in S$), entonces $\psi$ es un equilibrio de Nash.
\end{theorem}
\begin{proof}
Sea $\psi= \prod_{i \in N} \sigma_i$ un equilibrio correlacionado en forma factorizada. Debemos mostrar que para cualquier jugador $i$ y estrategia mixta $\sigma'_i$ para el jugador $i$, se cumple $u_i(\sigma) \geq u_i(\sigma'_i, \sigma_{-i})$.

Sean $x$ y $y$ estrategias puras para el jugador $i$.
Como $\sigma$ es un equilibrio correlacionado,
\begin{alignat}{1}
\label{eq:1:theo:correlacionado-nash}
0\ \leq\ \sigma_i(x) \sum_{s_{-i}} \sigma_{-i}(s_{-i})[u_i(x, s_{-i}) - u_i(y, s_{-i})] \,.
\end{alignat}
Al sumar sobre $x\in S_i$ obtenemos, 
\begin{alignat}{2}
\label{eq:2:theo:correlacionado-nash}
0\ \leq\ \sum_{x\in S_i} \sum_{s_{-i}} \sigma(x,s_{-i}) [u_i(x, s_{-i}) - u_i(y, s_{-i})]\ =\ \sum_s \sigma(s) [u_i(s) - u_i(y, s_{-i})] \,.
\end{alignat}
Si $x^* \in S_i$ es tal que $\sigma_i(x^*)>0$, obtenemos de \eqref{eq:1:theo:correlacionado-nash} al multiplicar por $\sigma'_i(y)$ y sumar sobre $y\in S_i$:
\begin{alignat}{1}
\label{eq:3:theo:correlacionado-nash}
\sum_{y \in S_i} \sigma'_i(y) \sum_{s_{-i}} \sigma_{-i} (s_{-i}) [u_i(x^*, s_{-i}) - u_i(y, s_{-i})]\ =\ \sum_{s} \sigma'(s) [u_i(x^*, s_{-i}) - u_i(s)]\ \geq\ 0
\end{alignat}
donde $\sigma'$ denota la estrategia $\sigma'=(\sigma'_i,\sigma_{-i})$. 
Al sumar \eqref{eq:2:theo:correlacionado-nash} y
\eqref{eq:3:theo:correlacionado-nash}, obtenemos que para cualquier $y$ y $x^*$ tal que $\sigma_i(x^*)>0$:
\begin{alignat}{1}
\label{eq:4:theo:correlacionado-nash}
\sum_{s \in S} u_i(s) [\sigma(s) - \sigma'(s)] - \sum_{s \in S} \sigma(s)u_i(y, s_{-i}) + \sum_{s \in S} \sigma'(s) u_i(x^*, s_{-i})\ \geq\ 0\ \,.
\end{alignat}
Por otra parte, note que:
\begin{alignat}{1}
  \sum_{s \in S} \sigma(s) u_i(x^*,s_{-i}) - \sum_{s \in S} &\sigma'(s) u_i(x^*,s_{-i}) \\
    &\qquad=\ \sum_{s_{-i}} u_i(x^*,s_{-i}) \sigma_{-i}(s_{-i}) \sum_{z\in S_i} [\sigma_i(z) - \sigma'_i(z)] \\
    &\qquad=\ \sum_{s_{-i}} u_i(x^*,s_{-i}) \sigma_{-i}(s_{-i}) \biggl[ \sum_{z\in S_i} \sigma_i(z) - \sum_{z\in S_i} \sigma'_i(z) \biggr] \\
    &\qquad=\ 0 \,.
\end{alignat}
Luego, al tomar $y=x^*$ en \eqref{eq:4:theo:correlacionado-nash},
\begin{alignat}{1}
 \sum_{s \in S} u_i(s) [\sigma(s) - \sigma'(s)]\ =\ \sum_{s \in S} u_i(s)\sigma(s) - \sum_{s \in S} u_i(s)\sigma'(s)\ =\ u_i(\sigma) - u_i(\sigma'_i, \sigma_{-i})\ \geq\ 0 \,.
\end{alignat}
Como $\sigma'_i$ es una estrategia cualquiera para el jugador $i$, $\sigma$ es un equilibrio de Nash.
\end{proof}

A diferencia del conjunto de equilibrios de Nash, el cual es un conjunto matemáticamente complejo (un conjunto de puntos fijos), el conjunto de equilibrios correlacionados en un conjunto bastante simple. En particular, el conjunto de equilibrios correlacionado es un politopo (generalización de un polígono en $\mathbb{R}^N$) convexo. Por lo tanto puede esperarse que existan procedimientos simples para calcular equilibrios correlacionados \cite{bib:correlated-equilibrium}.

\noindent\textcolor{red}{\bf *** Y esto nos va a llevar a calcular equilibrios de Nash? ***}

\begin{theorem}
Sean $\sigma$ y $\sigma'$ dos equilibrios correlacionados, y $\alpha$ un número real en $(0,1)$. Entonces, la distribución $\alpha\sigma + (1-\alpha)\sigma'$ es un equilibrio correlacionado.
\end{theorem}
\begin{proof}
Como $\sigma$ y $\sigma'$ son equilibrios correlacionados y $\alpha, 1 - \alpha \in (0, 1)$ se cumple que para cualesquiera $x$ e $y$:
\begin{alignat}{1}
 \alpha \sum_{s_{-i} \in S_{-i}} \sigma(x, s_{-i})[u_i(x, s_{-i}) - u_i(y, s_{-i})] \geq 0 & \, \text{ y } \\
 (1 - \alpha) \sum_{s_{-i} \in S_{-i}} \sigma'(x, s_{-i})[u_i(x, s_{-i}) - u_i(y, s_{-i})] \geq 0
\end{alignat}
Sumando las ecuaciones anteriores y factorizando se obtiene:
\begin{alignat}{1}
 \sum _{s_{-i} \in S_{-i}} [ \alpha \sigma(x, s_{-i}) +  (1 - \alpha) \sigma'(x, s_{-i})][u_i(x, s_{-i}) - u_i(y, s_{-i})] \geq 0
\end{alignat}
Por lo que $\alpha \sigma +  (1 - \alpha) \sigma'$ es un equilibrio correlacionado.
\end{proof}

