\chapter{Juegos en Formal Normal o Estratégica}
\label{section:forma-normal}

En un juego en formal normal los jugadores eligen una única acción (o estrategia) de forma simultánea, obteniendo un pago de acuerdo a las acciones realizada por cada uno de ellos. Estos juegos también se llaman frecuentemente \textit{``one-shot game''}   (juegos de un sólo disparo), ya que cada uno de los jugadores realiza una única acción \cite{bib:introductionCFR} (que puede representar la elección de una estrategia a usar en un juego de múltiples pasos. El ejemplo clásico es el juego de \textit{piedra, papel o tijera}. En este juego cada jugador elige una de las tres opciones mediante un gesto con sus manos: piedra (con un puño cerrado), papel (con la mano extendida) o tijera (con los dedos índice y medio levantados en forma de ``V"). La piedra gana contra la tijera, la tijera gana contra el papel y el papel gana contra la piedra. Si los jugadores eligen la misma opción, entonces es un empate.

\begin{definition}[\cite{bib:correlated-equilibrium}]
\label{def:forma-normal}
Un juego de N personas en \textbf{forma normal} (o estratégica) es una tupla $\Gamma = (N, (S_i)_{i \in N}, (u_i)_{i \in N})$, donde:
	\begin{itemize}[]
		\item $N = \{1, 2, \dots, N\}$ es el conjunto de jugadores.
		\item  $S_i$ es el conjunto de \textbf{estrategias puras} (o acciones) del jugador $i$.
		\item $u_i : \Pi _{i \in N} S_i \rightarrow \mathbb{R}$ es la \textbf{función de pago} del jugador $i$.
	\end{itemize}
\end{definition}

\section{Perfiles Estratégicos, Estrategias Mixtas, y Perfiles Estratégicos Mixtos}
A continuación presentamos conceptos básicos que denotan las diversas formas en que los jugadores pueden comportarse para un juego en forma normal. Las estrategias puras son la base a partir de las cuales se construyen las estrategias mixtas. Las estrategias puras se agregan en perfiles estratégicos que denotan el comportamiento de todos los jugadores de forma simultanea, y los perfiles estratégicos mixtos agregan las estrategias mixtas \cite{bib:tutorial-existence-nash}.

\begin{definition} Un \textbf{perfil estratégico} (o perfil de acción) es una $N$-tupla formada por una estrategia para cada jugador. $S = \Pi_{i \in N}S_i$ es el conjunto de perfiles estratégicos y $s = (s_i)_{i \in N}$ representa un elemento genérico de $S$.  
\end{definition}

Se denotará con $s_{-i}$ la combinación de las estrategias de todos los jugadores excepto la del jugador $i$, i.e., $s_{-i} = (s_{i'})_{i' \neq i}$. Frecuentemente descomponemos una estrategia $s$ es un par $(s_i,s_{-i})$ donde la primera componente es una estrategia pura para el jugador $i$ y la segunda componente es un vector de estrategias puras para los otros jugadores.

\textit{Piedra, papel o tijeras} es un juego para dos jugadores y las acciones (o estrategias puras) son las mismas para cada jugador: $S_1 = S_2 = \{R, P, S \}$ donde $R$ es piedra, $P$ es papel, y $S$ es tijeras. Los juegos en forma normal pueden representarse como una tabla $n$-dimensional, donde cada dimensión está asociada a un jugador y sus filas/columnas corresponden a las acciones de su jugador correspondiente. Cada una de las entradas de la tabla corresponde a un único perfil estratégico (pues representan la intersección de una única acción de cada jugador) y éstas contienen un vector de pagos para cada jugador \cite{bib:introductionCFR}. La Tabla \ref{table:pago-RPS} es la tabla de pagos correspondiente al juego piedra, papel o tijeras.

\begin{table}[t]
\begin{center}
\caption{Tabla de pagos de la forma normal del juego \textit{Piedra, Papel o Tijeras}. \textcolor{red}{\bf*** EXPLICAR UN POCO LA TABLA. POR LO MENOS UNA ENTRADA ***}}
\label{table:pago-RPS}
\begin{tabular}{l  c  c  c}
  & R (piedra) & P (papel) & S (tijeras) \\ \midrule
R (piedra) & $0,0$ & $-1,1$ & $1,-1$ \\ 
P (papel) & $1,-1$ & $0,0$ & $-1,1$ \\ 
S (tijeras) & $-1,1$ & $1,-1$ & $0,0$ \\ 
\end{tabular}
\end{center}
\end{table}

En vez de realizar siempre la misma acción, un jugador puede elegir su jugada de acuerdo a una distribución de probabilidad la cual se denomina una \textbf{estrategia mixta}. Dado un conjunto finito $A$, se denota con $\Delta(A)$ al conjunto de distribuciones de probabilidad sobre $A$, es decir $\Delta(A) = \{ (x_a)_{a \in A} : \sum_{a \in A} x_a = 1, x_a \geq 0\}$. 
%Se presentan a continuación, definiciones formales para estrategia mixta, soporte de una estrategia mixta, perfil estratégico mixto y ganancia esperada.

\begin{definition} Una \textbf{estrategia mixta} del jugador $i$, denotada con $\sigma_i$, es una distribución de probabilidad sobre el conjunto $S_i$; es decir $\sigma_i \in \Delta(S_i)$. Denotamos con $\sigma_i(s_i)$ la probabilidad que el jugador $i$ elija la acción $s_i \in S_i$. 
\end{definition}

\begin{definition}
El \textbf{soporte} (support) de una estrategia mixta $\sigma_i \in \Delta(S_i)$ del jugador $i$ es el conjunto de estrategias puras con una probabilidad positiva de ser elegidas:
\begin{alignat}{1}
\text{support}(\sigma_i)\ =\ \{s_i : \sigma_i(s_i) > 0 \} \,.
\end{alignat}
%\Blai{*** conjunto por extensi\'on denotado con el s\'{\i}mbolo `:' y `$|$': elegir una misma forma de hacerlo a lo largo de la tesis ***}
\end{definition}

\begin{definition}
Un \textbf{perfil estratégico mixto} $\sigma$ consiste en una estrategia mixta para cada jugador; es decir, $\sigma \in \Pi_{i \in N} \Delta(S_i)$ es una tupla de forma $\sigma=(\sigma_i)_{i \in N}$.
\end{definition}

Para $\sigma = (\sigma_i)_{i \in N}$ y $s = (s_i)_{i \in N}$, $\sigma(s)$ denota la probabilidad que el perfil estratégico mixto elija la estrategia mixta $s$; i.e., $\sigma(s)=\prod_{i\in N} \sigma_i(s_i)$. Para un perfil $\sigma$ y jugador $i$, descomponemos $\sigma$ en $(\sigma_i,\sigma_{-i})$ como la combinación de la estrategia para el jugador $i$ y el perfil $\sigma_{-i}$ para el resto de los jugadores. Similarmente, $\sigma_{-i}(s_{-i})=\prod_{j\in N,j\neq i}\sigma_j(s_j)$ denota la probabilidad de que los jugadores diferentes al jugador $i$ elijan las estrategias mixtas en el perfil $\sigma_{-i}$. Finalmente, si $x$ es una estrategia pura para el jugador $i$, también utilizamos $x$ para denotar la estrategia mixta $\sigma_i$ para el jugador $i$ tal que $\sigma_i(x)=1$.

\noindent\textcolor{red}{\bf*** Faltan ejemplos que ilustren estas ideas. Usar RPS. ***}

\section{Ganancia Esperada y Mejor Respuesta}
La ganancia esperada del jugador $i$ asociada al perfil estratégico mixto $\sigma$ denota el el valor promedio que el jugador $i$ obtendría después de jugar el juego infinitas veces cuando todos los jugadores utilizan las estratégias mixtas especificadas en $
\sigma$. 

\begin{definition}
\label{def:ganancia-esperada}
La \textbf{ganancia esperada} del jugador $i$ dado un perfil estratégico mixto $\sigma$ es
\begin{alignat}{1}
	u_i(\sigma)\ =\ \sum_{s \in S} u_i(s) \sigma(s)\ =\ \sum_{s \in S} u_i(s) \prod _{j \in N} \sigma_j(s_j)\ =\ \sum_{s \in S} u_i(s) \sigma_i(s_i) \sigma_{-i}(s_{-i})\,.
\end{alignat}
\end{definition}

La ganancia esperada del jugador $i$ la podemos descomponer como se muestra a continuación:

\begin{theorem}
\label{lemma:1}
La ganancia esperada $u_i(\sigma)$ del jugador $i$ dado el perfil estratégico $\sigma$ satisface:
\begin{alignat}{1}
u_i(\sigma)\ =\ \sum_{s_i\in S_i} \sigma_i(s_i) \sum_{s_{-i}\in S_{-i}} \sigma_{-i}(s_{-i}) u_i(s_i,s_{-i}) \,.
\end{alignat}
\end{theorem}

Dado una perfil estratégico mixto $\sigma$, tiene sentido preguntarse si el jugador $i$ está jugando de la mejor forma dadas las estrategias seleccionadas por los otros jugadores. A partir de esta pregunta, definimos el concepto de mejor respuesta para el jugador $i$ dado un perfil $\sigma_{-i}$ para los otros jugadores.

%También es importante definir el concepto de mejor respuesta; término que se utiliza para caracterizar una estrategia que maximiza la ganancia esperada de un jugador fijo, conociendo las estrategias del resto de los jugadores.

\begin{definition}
\label{def:mejor-respuesta}
Sea $i\in N$ un jugador, $\sigma_i$ una estrategia mixta para el jugador $i$, y $\sigma_{-i}$ un perfil estratégico mixto para el resto de los jugadores. Decimos que $\sigma_i$ es una \textbf{mejor respuesta} con respecto a $\sigma_{-i}$ si y s\'olo si
$u_i(\sigma_i,\sigma_{-i}) \geq u_i(\sigma'_i,\sigma_{-i})$ para toda estrategia mixta $\sigma'_i$ para el jugador $i$.
\end{definition}

Una mejor respuesta no es necesariamente única. En efecto, salvo el caso extremo en el que hay una única mejor respuesta, que como veremos debe ser una estrategia pura, el número de mejores respuestas es infinito. Cuando el soporte de una estrategia mixta que es mejor respuesta incluye dos o más estrategias puras (acciones), el agente debe ser indiferente a cualquiera de éstas y cualquier mezcla de estas  acciones también será mejor respuesta \cite{bib:tutorial-existence-nash}.

\begin{theorem}
\label{theo:mejor-respuesta}
Sea $\sigma^*_i$ una estrategia mixta para el jugador $i$ que es mejor respuesta a $\sigma_{-i}$. Cualquier estrategia mixta $\sigma_i$ para el jugador $i$ cuyo soporte sea un subconjunto del soporte de $\sigma^*_i$ es también una mejor respuesta a $\sigma_{-i}$.
\end{theorem}

\noindent\textcolor{red}{\bf*** Faltan ejemplos que ilustren estas ideas. Usar RPS y ejemplos concretos donde calcular ganancia esperada y mejor respuesta. ***}

\section{Equilibrio de Nash}
Cuando cada jugador juega con una mejor respuesta frente a las estrategias del resto de los jugadores se dice que tenemos un Equilibrio de Nash. En un Equilibrio de Nash ningún jugador puede mejorar su ganancia esperada cambiando su estrategia de forma aislada. Por otra parte, si el juego es finito, siempre existe al menos un equilibrio de Nash. Un juego es finito si el número de jugadores es finito, y si el conjunto de estrategias puras para cada jugador es también finito. El concepto de equilibrio de Nash es uno de los conceptos de solución más importantes en el área de teoría de juegos no cooperativos, y es el principal concepto de solución utilizado en el presente trabajo.

\begin{definition}
\label{def:equilibrio-nash} Un perfil estratégico mixto $\sigma$ es un \textbf{equilibrio de Nash} si y s\'olo si para todo jugador $i$, la estrategia $\sigma_i$ es mejor respuesta del jugador $i$ para $\sigma_{-i}$.
\end{definition}

\begin{theorem}[\cite{bib:tutorial-existence-nash}]
\label{theo:existencia-nash}
Todo juego finito tiene al menos un equilibrio de Nash.
\end{theorem}

\noindent\textcolor{red}{\bf*** Faltan ejemplos que ilustren estas ideas. Usar RPS. Presentar un equlibro de Nash para RPS. Discutirlo. Argumentar que no puede haber un perfil estratégico que sea un equilibrio de Nash (?). ETC ***}


\section{Equilibrio Correlacionado}

Aunque el equilibrio de Nash es uno de los principales conceptos de solución, es importante destacar que éste no garantiza el mejor resultado si los jugadores toman sus decisiones en conjunto. Si a los jugadores se les permite correlacionar sus acciones (es decir, trabajar en grupo), pueden existir estrategias con mayores ganancias para ellos. 
Este tipo de situaciones son las que considera la noción de equilibrio correlacionado que generaliza al equilibrio de Nash \cite{bib:correlated-equilibrium}. Todo equilibrio de Nash es un equilibrio correlacionado, pero este último permite otras soluciones importantes \cite{bib:correlated-equilibrium}. La relación entre los conceptos de equilibro de Nash y correlacionado se muestra en los Teoremas \ref{theo:nash-correlacionado} y \ref{theo:correlacionado-nash}.

\begin{definition}
\label{def:equilibrio-correlacionado}
Una distribución $\psi\in\Delta(S)$ es un \textbf{equilibrio correlacionado} si y sólo si para cualquier jugador $i$, y para cualesquiera estrategias puras $x, y \in S_i$,
\begin{alignat}{1}
\label{eq:equilibrio-correlacionado}
\sum_{s_{-i}\in S_{-i}} \psi(x,s_{-i}) [ u_i(x,s_{-i}) - u_i(y,s_{-i})]\ \geq\ 0 \,.
\end{alignat}
\end{definition}

Si en la desigualdad \eqref{eq:equilibrio-correlacionado} se cambia el $0$ por un $\epsilon > 0$ se obtiene la definición de $\epsilon$-equilibrio correlacionado.


\begin{theorem}
\label{theo:nash-correlacionado}
Si $\sigma$ es un equilibrio de Nash, entonces $\sigma$ es un equilibrio correlacionado.
\end{theorem}

\begin{theorem}
\label{theo:correlacionado-nash}
Sea $\psi\in\Delta(S)$ un equilibrio correlacionado. Si $\psi$ se factoriza como $\psi=\prod_{i\in N} \sigma_i$ donde $\{\sigma_i\}_{i\in N}$ es un conjunto de estrategias mixtas para cada jugador (i.e., $\psi(s)=\prod_{i \in N} \sigma_i(s_i)$ para todo $s\in S$), entonces $\psi$ es un equilibrio de Nash.
\end{theorem}

A diferencia del conjunto de equilibrios de Nash, el cual es un conjunto matemáticamente complejo (un conjunto de puntos fijos), el conjunto de equilibrios correlacionados en un conjunto bastante simple. En particular, el conjunto de equilibrios correlacionado es un politopo (generalización de un polígono en $\mathbb{R}^N$) convexo. Por lo tanto puede esperarse que existan procedimientos simples para calcular equilibrios correlacionados \cite{bib:correlated-equilibrium}.

\noindent\textcolor{red}{\bf *** Y esto nos va a llevar a calcular equilibrios de Nash? ***}

\begin{theorem}
Sean $\sigma$ y $\sigma'$ dos equilibrios correlacionados, y $\alpha$ un número real en $(0,1)$. Entonces, la distribución $\alpha\sigma + (1-\alpha)\sigma'$ es un equilibrio correlacionado.
\end{theorem}
