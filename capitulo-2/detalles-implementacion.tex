\section{Detalles de Implementación}

En esta sección se describen los detalles relacionados a la implementación de los algoritmos utilizados para la parte experimental de la investigación. Los algoritmos fueron implementados en el lenguaje de programación C++, utilizando la librería estándar y una librería adicional llamada \textit{Eigen}, para factorizar matrices y resolver sistemas de ecuaciones.

\subsection{Regret Matching}
Se implementó una clase para encontrar un equilibrio de Nash mediante el algoritmo de \textit{Regret Matching}. En cada iteración la actualización de las estrategias depende de cada procedimiento según las fórmulas propuestas en la sección anterior.

\subsection{Juegos en forma extensiva}

Para representar el árbol en forma extensiva se creó la clase \textit{Nodo}, la cual tiene como atributos el jugador correspondiente, el conjunto de información, la distribución de probabilidad en caso que sea un nodo de azar y la ganancia del jugador $1$ en caso que sea un nodo terminal o apuntadores a sus hijos en caso que no lo sea.

El constructor de esta clase utiliza \textit{Depth First Search} para la lectura del árbol desde un archivo. El archivo de texto debe seguir el siguiente formato, una línea con tres valores enteros: el número del jugador ($p$), el conjunto de información ($I$) y el número de hijos del nodo ($N$). El nodo es de azar si el número del jugador es igual a cero, en este caso la línea debe tener $N$ números adicionales que sean una distribución de probabilidad sobre los hijos. Por otra parte, si el nodo es terminal ($N = 0$) entonces la línea debe tener un valor adicional que representa la utilidad para el jugador $1$. Una vez leído la información del nodo actual el constructor es llamado con cada uno de los hijos para construir cada uno de los subárboles.

\subsection{Forma normal a partir de la forma extensiva}
\label{subsec:FN-FE}

Para aplicar los algoritmos de \textit{Regret Matching} en juegos en formas extensiva es necesario hallar la forma normal correspondiente. Para esto se utiliza \textit{dfs} para determinar la cantidad de conjuntos de información de cada jugador y cuantas acciones diferentes tiene en cada uno, el árbol debe ser una instancia de la clase descrita anteriormente. Las estrategias puras son listadas usando \textit{backtracking} y para obtener el valor de pago se recorre el árbol mediante \textit{dfs}, pero sólo expandiendo los nodos que corresponden a la estrategia dada.

\subsection{Estrategia de comportamiento a partir estrategia mixta}
\label{subsec:EC-EM}
Una vez encontrado un equilibrio de Nash para la forma normal en un juego en forma extensiva originalmente, es deseable ser capaz de encontrar la estrategia de comportamiento respectiva. Debido a que más de un árbol pueden generar la misma matriz de pagos en forma normal, es necesario que tanto el árbol como la matriz de pagos sean proporcionados de forma explícita. Por otra parte, es necesario listar nuevamente todas las estrategias puras (de la misma forma que se listaron para encontrar la forma normal), para poder hacer la correspondencia entre cada estrategia pura del jugador $1$ y cada fila de la matriz, y cada estrategia pura del jugador $2$ y cada columna.

Para encontrar la distribución de probabilidad en cada conjunto de información se utiliza \textit{dfs} por cada estrategia mixta de cada jugador para encontrar, para cada conjunto de información $I$ y cada acción $a \in A(I)$, la probabilidad de alcanzar $I$ ($\pi(I)$) y la probabilidad de elegir la acción $a$ dado que $I$ es alcanzado. Luego, es utilizada la ecuación \ref{eq:mixta-a-comportamiento} para obtener cada una de las distribuciones de probabilidad.